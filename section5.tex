\section{\lr{05 Maximal submodule}}

\textbf{قضیه.} فرض کنید $M$ یک $R$-مدول ناصفر و \textcolor{red}{مولد متناهی} باشد. در این صورت هر زیرمدول سرهٔ $M$ مشمول در یک زیرمدول ماکسیمال $M$ خواهد بود. لذا هر $R$-مدول با تولید متناهی حداقل یک زیرمدول ماکسیمال دارد.

\textit{اثبات.} فرض کنید $K$ زیرمدولی سره از $M$ باشد. فرض کنید:
\[
    \mathcal{A} = \{ N \mid K \subseteq N \subsetneq M \}
\]
و آن را با رابطهٔ شمول در نظر بگیرید. واضح است که
$K \in \mathcal{A}$،
پس
$\mathcal{A} \neq \emptyset$.

چنانچه
$\mathcal{T} \neq \emptyset$
یک زنجیر از اعضای
$\mathcal{A}$
باشد، بوضوح
\[
    \mathcal{L} = \bigcup_{N \in \mathcal{T}} N
\]
یک کران بالای
\(\mathcal{T}\)
در $\mathcal{A}$ است (چرا؟).

اگر
\(\mathcal{L} = M\)
آنگاه چون
\(M\)
با تولید متناهی است پس مثلاً
\[
    M = \langle \alpha_1, \alpha_2, \dots, \alpha_r \rangle
\]
و لذا
$N_1, \dots, N_r$
از اعضای
$\mathcal{T}$
یافت می‌شود که
$\alpha_i \in N_i$
در نتیجه
$\exists j$
که
\[
    \forall i; \quad  \alpha_i \in N_j
\]
(چرا؟).

پس
$M = N_j$، که تناقض است.
بنابراین
$\mathcal{A}$
دارای عنصر ماکسیمال است.
یعنی زیرمدول ماکسیمال برای
\(M\)
وجود دارد.

\hrulefill

\textbf{نتیجه.} قبلاً دیده بودیم که در حلقه جابجایی و یک‌دار
هر ایده‌آل مشمول در یک ایده‌آل ماکسیمال است. چون
\(R = (1)\)
پس با تولید متناهی است و لذا از حکم فوق نتیجه به‌دست می آید.



\hrulefill

\textbf{قرارداد.} از این پس، همواره $R$-مدول‌ها را $R$-مدول چپ و یک‌دار درنظر می‌گیریم.

\hrulefill

\textbf{تعریف.} $R$-مدول $M \neq (0)$ را \textbf{ساده} یا \textcolor{red}{تحویل‌ناپذیر} نامند هرگاه تنها زیرمدول‌های آن $(0)$ و $M$ باشند.

\hrulefill

\textit{مثال.} میدان $\mathbb{F}$ به عنوان $\mathbb{F}$-مدول.

\hrulefill

\textit{تذکر.}
درمورد مدول‌های (با طول متناهی؟)، مدولهای ساده بلوک‌های سازندهً آنهاهستند. چیزی شبیه به گروه‌های ساده در نظریه گروه‌‌ها.

\hrulefill


\textit{تذکر.} واضح است اگر $M$ یک $R$-مدول ساده باشد و $0 \neq a \in M$ آنگاه
\[
    M = Ra.
\]

\hrulefill

\textit{مثال.} $\mathbb{Z}_p$ به عنوان $\mathbb{Z}_p$-مدول ساده است.

$\mathbb{Z}_p$ به عنوان $\mathbb{Z}$-مدول.

\hrulefill


\textit{مثال.} در حلقهٔ
جابجایی و یک‌دار
$R$، ایده‌آل مینیمال $I$ به عنوان
یک
$R$-مدول
ساده است.
چون هر
$R$-زیر مدول
زیر مدول آن در واقع یک ایده‌آل است.


\hrulefill

\textit{مثال.}
در حلقهٔ
$R$
اگر
$I$
یک ایده‌آل ماکسیمال باشد،
\(\frac{R}{I}\)
به‌عنوان
$R$-مدول
یک
$R$-مدول ساده
است.

\hrulefill


\textit{تذکر.} هر $R$-مدول لزوماً شامل
یک
$R$-زیر مدول
ساده نیست، مانند $\mathbb{Z}$ به عنوان $\mathbb{Z}$-مدول.

\hrulefill

\textbf{قضیه.} فرض کنید $M$ یک $R$-مدول ساده  (تحویل‌ناپذیر) باشد. در این صورت ایده‌آل چپ ماکسیمال $I$ از $R$ موجود است که:
\[
    M \cong \frac{R}{I} \quad (\text{-مدولی}R)
\]
\textit{اثبات.} اگر $0 \neq a \in M$,
آنگاه
\(a \in Ra \leqslant M\)
پس
$M = Ra$
چون
\(M\)
ساده است. اما چون
$M = Ra$
پس
\(M\)
دوری است. و لذا
\(M \cong \frac{R}{I}.\)
که
$I = \mathrm{Ann}_R(a)$ .
اما
\(M\)،
زیرمدول غیربدیهی ندارد، پس
\(\frac{R}{I}\)
هم به‌عنوان
$R$-مدول
ساده است و لذا ایده‌آلی مانند
\(J\)
وجود ندارد که
\(I \lneqq J \lneqq R\)
یعنی
\(I\)
ایده‌آلی ماکسیمال
\(R\)
است.

\hrulefill

\textit{تذکر.} اگر $N$ زیرمدول ماکسیمال $M$ باشد آنگاه $\frac{M}{N}$ یک $R$-مدول ساده است.
چرا که بنابر قضیهً تناظر، تنها زیرمدول‌های $\frac{M}{N}$ به شکل $\frac{K}{N}$ هستند که $N \subseteq K \subseteq M$.

\hrulefill

\textbf{قضیه.} فرض کنید $M = \sum_{i \in I} M_i$ که در آن $M_i$ها $R$-مدول‌های \textcolor{red}{ساده}  \(M\) هستند. اگر $K$ زیرمدولی از $M$ باشد در این صورت زیرمجموعهٔ $J \subseteq I$ وجود دارد به طوری که
\(\sum_{j \in J} M_j\)
یک مجموع مستقیم است و بعلاوه:
\[
    M = K \oplus \sum_{j \in J} M_j.
\]

\hrulefill

\textit{اثبات.} فرض کنید
\[
    \mathcal{A} = \left\{ I' \subseteq I \ \middle|
    \begin{aligned}
         & \text{یک مجموعهٔ مستقیم باشد و} \sum_{i \in I'} M_i \\
         & \sum_{i \in I'} M_i \cap K = \{0\}
    \end{aligned} \right\}.
\]
توجه که اگر
$I' = \emptyset$
آنگاه
$\sum_{i \in I'} M_i = 0$.
پس به‌وضوح
$\mathcal{A} \neq \emptyset$
زیرا
$\emptyset \in \mathcal{A}$.
رابطهٔ ترتیبی را شمول درنظر می‌گیریم که رابطه‌ای جزئاً مرتب می‌شود. حالا زنجیر ناتهی
$\mathcal{T}$
را در
$\mathcal{A}$
درنظر بگیرید. پس
$\mathcal{T}$
کاملاً مرتب است.
قرار می‌دهیم
\[
    S = \bigcup_{I' \in \mathcal{T}} I'
\]
به وضوح
$\forall I' \in \mathcal{T}$، $I' \subseteq S$.
پس کران  بالایی برای
\(\mathcal{T}\)
است. ثابت می‌کنیم که
\(S\)
در
\(\mathcal{A}\)
قرار دارد.




\textbf{شرط اول.}



اگر
$\sum_{i \in S} M_i$
مجموعهٔ مستقیم نباشد، پس
$ m_{i_1}, \dots, m_{i_n}$
با
\(i_1 \in I_1 \cdots i_n \in I_n\)
وجود دار که
$m_{i_1} + \dots + m_{i_n} = 0$
و
\(\{i_1, \cdots, i_n\} \subseteq S\).
اما چون در زنجیر هستیم
\(\alpha\)ای
هست که
\(\{i_1, \cdots, i_n\} \subseteq I_\alpha\)
و
\(I_\alpha\)
در
\(mathcal(A)\)
است. پس از
$m_{i_1} + \dots + m_{i_n} = 0$
نتیجه می‌شود تمام
\(m_{i_j}\)
ها صفرند.

\textbf{شرط دوم.}\( K \cap \sum_{i \in S} M_i = 0.\)

چنانچه
$x \in \left(\sum_{i \in S} M_i\right) \cap K$
دراین‌صورت
\[
    \exists m_{i_j}; \quad x = m_{i_1} + \dots + m_{i_n}.
\]
و دقیقا شبیه بحث بالا به تناقض می‌رسیم. پس بنابر لم زورن
\(mathcal(A)\)
دارای عنصری ماکسیمال مانند
\(J\)
است.

\textbf{ادعا:}
\begin{enumerate}[label = \Roman*.]
    \item $\sum_{i \in J} M_i$ مجموع مستقیم است.
    \item$M = K \oplus \sum_{i \in J} M_i$.
\end{enumerate}
به وضوح چون $K \cap \sum_{i \in J} M_i = 0$ پس مجموع این دو، مجموع مستقیم است.
نهایتاً ثابت می‌کنیم به ازای هر
\(\beta \in I\)،
\[
    M_\beta \subseteq N = K \oplus \sum_{i \in J} M_i.
\]



برای هر $\beta \in I$ دقت می‌کنیم که چون $M_\beta$ ساده است و $M_\beta \cap N$ یک $R$-زیرمدول $M_\beta$ است، پس $M_\beta \cap N =(0)$ یا $M_\beta \cap N =  M_\beta$.
چنانچه
$M_\beta \cap N =  M_\beta$
یعنی
\(M_\beta \subseteq N\).
اما خواهیم دید که
$M_\beta \cap N =(0)$
امکان‌پذیر نیست.

چنانچه
\(\beta\)
ای باشد که
$M_\beta \cap N =(0)$
نشان می‌دهیم که
\(\beta \cup J\)
نیز در
\(\mathcal{A}\)
قرار داد که با ماکسیمال بودن
\(J\)
در تناقض است.


دقت کنیم که
$M_\beta \cap N =(0)$
نتیجه می‌دهد که
$M_\beta \cap K =(0)$
و بعلاوه
$M_\beta \cap \sum_{i \in J} M_i = 0$.
از طرف دیگر چنانچه
\[
    \left(\sum_{i \in J \cup \{\beta\}} M_i\right) \cap K \neq 0.
\]

فرض کنید $x$ در اشتراک باشد،
در این‌صورت
$x\in K$
و ضمناً
\[
    x \in \sum_{i \in J \cup \{\beta\}} M_i = \left(\sum_{i \in J} M_i\right)  + M_\beta
\]
پس
$x = y + m_\beta$
که در آن
\(y \in \sum_{i \in J} M_i\)
و
\(m_\beta \in M_\beta\)
اما در این صورت
\[
    m_\beta = x - y \in K \oplus \sum_{i \in J} M_i = N
\]
پس
\[
    m_\beta \in  N \cap M_\beta  = (0)
\]
و لذا
\(m_\beta = 0 \)
پس
\[
    x=y \in \Sigma_{i\in J} M_i
\]
و
\(x \in K\)
یعنی
\[
    x \in  \sum_{i \in J} M_i \cap K  = (0)
\]
یعنی
\(x = 0\)
پس حکم ثابت شد.


\hrulefill

\textbf{نتیجه.}
اگر $M = \sum_{i \in I} M_i$ مجموع خانواده‌ای از زیرمدول‌های سادهٔ $M$ باشد آنگاه
زیرمجموعهٔ
$J \subseteq I$ وجود دارد که:
\[
    M = \bigoplus_{i \in I} M_i.
\]



\textit{اثبات.}
در قضیهً قبل
فرض کنید
\( k = (0) \).

\textbf{تعریف.} فرض کنید \( M \) یک \( R \)-مدول باشد.
در این صورت زیرمدول
\( M_1 \)
از
\( M \)
را یک جمعوند مستقیم
\( M \)
نامیم، هرگاه زیرمدول
\( M_2 \)
از
\( M \)
موجود باشد که
\( M = M_1 \oplus M_2 \)
(اصطلاحاً گوییم \( M_1 \) در \( M \) مکمل دارد و \( M_2 \) را یک مکمل \( M_1 \) در \( M \) می‌نامیم).



\textbf{تعریف.} مدول \( M \) را \textit{مکمل‌پذیر} نامیم هرگاه هر زیرمدول آن دارای مکمل باشد.

\textbf{مثال.} فضاهای برداری مکمل‌پذیرند.

\textbf{مثال.}
\(K_4\)
به‌عنوان
\( \mathbb{Z} \)-مدول مکمل‌پذیر است.

\textbf{مثال.} در \( \mathbb{Q} \) به‌عنوان \( \mathbb{Z} \)-مدول، \( \mathbb{Z} \) مکمل ندارد (چرا؟)

\textbf{مثال.} در \( \mathbb{Q} \) به‌عنوان \( \mathbb{Z} \)-مدول، هیچ زیرمدول ناصفر مکمل ندارد. (چرا؟)

\[
    Q = M \oplus N
    \qquad \ldots
\]

