\section{\lr{13 Length of a Module}}




\textbf{طول یک مدول}

\textbf{تعریف:} دنبوره‌ای از زیرمدول‌های $M$ به شکل
\[
    0 = M_0 \subsetneq M_1 \subsetneq M_2 \subsetneq \cdots \subsetneq M_n = M
\]
را یک \textit{سری} به طول $n$ می‌نامیم.

\textbf{تعریف (تصفیه سری):} $\{ M_i' \}_{i=1}^m$ را تصفیه‌ای از $\{ M_i \}_{i=1}^n$ گوییم هرگاه
\[
    \{ M_i \}_{i=1}^m \supseteq \{ M_i' \}_{i=1}^n.
\]

\textbf{تعریف:} در سری
\[
    0 = M_0 \subsetneq M_1 \subsetneq \cdots \subsetneq M_n = M
\]
هر خارج‌قسمت $M_{i+1} / M_i$ یک \textit{عامل سری} گفته می‌شود. حال اگر هر عامل یک $R$-مدول ساده باشد، سری را یک \textit{سری ترکیبی} برای $M$ می‌نامیم.

\textbf{مثال:}
اگر $\{ M_i \}_{i=1}^n$ خانواده‌ای از $R$-مدول‌های ساده باشد، آنگاه
\[
    0 \subsetneq M_1 \subsetneq M_1 \oplus M_2 \subsetneq \cdots \subsetneq \bigoplus_{i=1}^n M_i
\]
یک سری ترکیبی برای $\bigoplus_{i=1}^n M_i$ است.

\textbf{مثال:}
\[
    0 \subsetneq \langle 4 \rangle \subsetneq \langle 2 \rangle \subsetneq \mathbb{Z}_8
\]
یک سری ترکیبی برای $\mathbb{Z}_8$ به‌عنوان $\mathbb{Z}$-مدول است.

\textbf{مثال:}
$\mathbb{Z}$ به‌عنوان $\mathbb{Z}$-مدول سری ترکیبی ندارد.


\textbf{تعریف:} فرض کنید $M$ یک $R$-مدول باشد که حداقل یک سری ترکیبی دارد.
طول کوتاه‌ترین سری ترکیبی $M$ را یا $\ell(M)$ خالص می‌گوییم و آن را \textit{طول مدول $M$} گوییم.

اگر سری ترکیبی برای $M$ وجود نداشته باشد، $\ell(M) = \infty$ تعریف می‌کنیم.
طبیعی است که در غیر این‌صورت $M$ با طول متناهی است.

\textbf{قضیه:} اگر $M$ یک $R$-مدول با طول متناهی باشد، در این صورت برای هر زیرمدول $N \subseteq M$ داریم:
\[
    \ell(N) \leq \ell(M).
\]

\textbf{اثبات:} اگر $\ell(M) = n$، در این صورت
\[
    0 = M_0 \subsetneq M_1 \subsetneq \cdots \subsetneq M_n = M
\]
وجود دارد که عواملش ساده‌اند. حال فرض کنید $N$ زیرمدولی سره از $M$ باشد. ثابت می‌کنیم $\ell(N) \leq \ell(M)$.

با اشتراک گرفتن داریم:
\[
    0 = M_0 \cap N \subseteq M_1 \cap N \subseteq \cdots \subseteq M_n \cap N = M \cap N = N.
\]
زنجیری از زیرمدول‌های $N$ است که تعداد جملات آن برابر با $n$ است، ولی ممکن است بعضی از جملات برابر باشند.

تعریف می‌کنیم:
\[
    \varphi : M_i \cap N \longrightarrow \frac{M_i}{M_{i-1}},
    \quad
    \alpha \longmapsto \alpha + M_{i-1}.
\]
بنابراین قضیه اول ایزومرفیسم نتیجه می‌دهد.


\[
    \frac{M_i \cap N}{\ker \varphi} \cong \operatorname{Im} \varphi \subseteq \frac{M_i}{M_{i-1}}
\]
که در آن
\[
    \ker \varphi = M_{i-1} \cap N.
\]
چون $\frac{M_i}{M_{i-1}}$ ساده است، پس $\operatorname{Im} \varphi = 0$ یا $\operatorname{Im} \varphi \cong \frac{M_i}{M_{i-1}}$.

نتیجه می‌گیریم:
\[
    \forall i \quad \frac{M_i \cap N}{M_{i-1} \cap N} \cong \frac{M_i}{M_{i-1}} \ \text{یا} \ 0.
\]
بنابراین $\ell(N) \leq \ell(M)$.

ادعا می‌کنیم که تساوی $\ell(N) = \ell(M)$ تنها در حالتی رخ می‌دهد که $M_i \cap N = M_i$ برای همهٔ $i$.

\textit{اثبات:}
برای $i=0$ بدیهی است که $M_0 \cap N = M_0$.

فرض استقرا: اگر $M_{i-1} \cap N = M_{i-1}$، آنگاه
\[
    \frac{M_i \cap N}{M_{i-1} \cap N} \cong \frac{M_i \cap N}{M_{i-1}} \cong \frac{M_i}{M_{i-1}}.
\]
پس $M_i \cap N \subseteq M_i$ و هر $x \in M_i$ داریم:
\[
    x + M_{i-1} = t + M_{i-1} \quad \text{برای بعضی } t \in M_i \cap N
\]
که به این معنی است:
\[
    x - t \in M_{i-1} \subseteq M_{i-1} \cap N \subseteq N,
\]
و چون $t \in N$، نتیجه می‌شود $x \in N$ و لذا $x \in M_i \cap N$.

پس $M_i \cap N = M_i$ و حکم استقرا کامل است.

ولذا $M_i \subseteq M_i \cap N$ پس
\[
    M_i = M_i \cap N.
\]
در این صورت
\[
    M_n = M_n \cap N = M \cap N = N
\]
و نتیجه می‌شود $N = M$.

\textbf{قضیه وجود هولدر.}
اگر $M$ یک $R$-مدول با طول $n$ باشد، در این صورت هر سری ترکیبی $M$ دارای طول $n$ است.
بعلاوه هر سری از زیرمدول‌های $M$ تصفیهای دارد که یک سری ترکیبی $M$ است.

\textbf{اثبات:} داریم $\ell(M) = n$.
ادعا می‌کنیم طول هر سری از زیرمدول‌های $M$ حداکثر $\ell(M)$ است.

فرض کنید
\[
    0 \subsetneq M_1 \subsetneq M_2 \subsetneq \cdots \subsetneq M_k = M
\]
یک سری ترکیبی $M$ باشد. با قضیهٔ قبلی:
\[
    \ell(M_1) < \ell(M_2) < \cdots < \ell(M_k) = \ell(M) = n
\]
که هر بار حداقل $+1$ اضافه می‌شود. پس $k \leq n$.

اما طبق تعریف طول سری، طول کوتاه‌ترین سری ترکیبی $n$ است، پس $k \geq n$.
نتیجه: $n \leq k \leq n$ و لذا $k = n$.


نتیجهٔ آخر:
ساده‌نبودن $\frac{M_i}{M_{i-1}}$ یعنی
\[
    0 \neq \frac{N}{M_{i-1}} \subsetneq \frac{M_i}{M_{i-1}}
\]
و این با $M_{i-1} \subsetneq N \subsetneq M_i$ باعث می‌شود طول سری یکی افزایش یابد و این خلاف تعریف یک سری ترکیبی است.

\noindent\rule{\linewidth}{0.4pt}

\textbf{نتیجه:}
اگر $M$ و $N$ دو $R$-مدول ساده باشند و
\[
    M^m \cong_R N^n
\]
آنگاه $m = n$ و $M \cong N$.

\textit{اثبات:}
سری ترکیبی:
\[
    0 \subsetneq M \subsetneq M \oplus M \subsetneq \cdots \subsetneq M^m
\]
با توجه به یکریختی $\varphi : M^m \to N^n$ به‌عنوان $R$-مدول، داریم:
\[
    0 \subsetneq \varphi(M) \subsetneq \varphi(M \oplus M) \subsetneq \cdots \subsetneq \varphi(M^m) = N^n
\]
یک سری برای $N^n$ است و هر عامل آن ساده است، زیرا مثلاً برای بعضی $T$ داریم:
\[
    \varphi^{-1}(T) = M \quad \text{یا} \quad \varphi^{-1}(T) = M \oplus M
\]
(عامل‌ها ایزومرف هستند).

پس یک سری ترکیبی برای $N^n$ است. با توجه به:
\[
    0 \subsetneq N \subsetneq N \oplus N \subsetneq \cdots \subsetneq N^n
\]
نتیجه می‌شود $m = n$.

\[
    \operatorname{Hom}_R(M^m, N) \cong \bigoplus_{i=1}^m \operatorname{Hom}_R(M, N)
\]
اگر $M \not\cong N$، آنگاه $\operatorname{Hom}_R(M, N) = 0$ که تناقض است.

\textbf{قضیه.}
$M$ یک $R$-مدول است. آنگاه
\[
    M \ \text{نوتری و آرتینی} \ \Longleftrightarrow \ \ell(M) < \infty.
\]

\textbf{اثبات:}
($\Rightarrow$) اگر $\ell(M) < \infty$، طول هر سری متناهی است. $\checkmark$

($\Leftarrow$) اگر $M$ نوتری و آرتینی باشد، $M$ دارای زیرمدول مینیمال است (دلیل ساده: مینیمال بودن در حالت آرتینی) مانند $M_1$.
اگر $M = M_1$ که تمام شد، وگرنه $\frac{M}{M_1}$ آرتینی و نوتری است، پس مینیمال دارد مانند $M_2 / M_1$ و الی آخر.

با تکرار این فرایند، سری
\[
    0 \subsetneq M_1 \subsetneq M_2 \subsetneq \cdots \subsetneq M_k \subsetneq \cdots
\]
خواهیم داشت که چون نوتری است متوقف می‌شود و لذا $M_t = M$ که به طول $t$ است و در نتیجه $\ell(M) < \infty$.

\noindent\rule{\linewidth}{0.4pt}

\textbf{نتیجه:}
اگر $M$ یک $R$-مدول با طول متناهی باشد، آیا $\frac{M}{N}$ با طول متناهی است؟

\[
    \ell(M) < \infty \ \Rightarrow \ M \ \text{نوتری و آرتینی} \ \Rightarrow \ N \ \text{و} \ \frac{M}{N} \ \text{نوتری و آرتینی} \ \Rightarrow \ \ell(N) < \infty \ \text{و} \ \ell\left(\frac{M}{N}\right) < \infty.
\]

\noindent\rule{\linewidth}{0.4pt}

\textbf{حقیقت:}
فرض کنید $M$ یک $R$-مدول و $f : M \to M$ یک $R$-همریختی باشد.

(الف) اگر $M$ آرتینی باشد، آنگاه $\exists n$ که $\operatorname{Im} f^n + \ker f^n = M$.

(ب) اگر $M$ نوتری باشد، آنگاه $\exists n$ که $\operatorname{Im} f^n \cap \ker f^n = 0$.


\textbf{اثبات:}

(الف) چون $M$ آرتینی است:
\[
    \operatorname{Im} f \supseteq \operatorname{Im} f^2 \supseteq \operatorname{Im} f^3 \supseteq \cdots
\]
ایستا می‌شود. پس $\exists n$ که:
\[
    \operatorname{Im} f^n = \operatorname{Im} f^{n+i} \quad (\forall i \geq 0).
\]
در نتیجه:
\[
    \operatorname{Im} f^n = \operatorname{Im} f^{n+1}.
\]
حال برای هر $x \in M$، چون $f^n(x) \in \operatorname{Im} f^n = \operatorname{Im} f^{n+1}$،
$\exists y \in M$ که $f^n(x) = f^{n+1}(y) = f(f^n(y))$.
پس:
\[
    x - f^n(y) \in \ker f^n,
\]
و لذا:
\[
    x = f^n(y) + \big(x - f^n(y)\big) \in \operatorname{Im} f^n + \ker f^n.
\]
بنابراین:
\[
    M = \operatorname{Im} f^n + \ker f^n.
\]

(ب) به‌طور مشابه چون $M$ نوتری است:
\[
    \ker f \subseteq \ker f^2 \subseteq \ker f^3 \subseteq \cdots
\]
ایستا می‌شود. پس $\exists n$ که:
\[
    \ker f^n = \ker f^{n+i} \quad (\forall i \geq 0),
\]
و در نتیجه:
\[
    \ker f^n = \ker f^{2n}.
\]
حال اگر $x \in \ker f^n \cap \operatorname{Im} f^n$،
$\exists y$ که $x = f^n(y)$ و $f^n(x) = 0$.
از $x = f^n(y)$ و $f^n(x) = 0$ نتیجه می‌شود:
\[
    f^{2n}(y) = 0 \ \Rightarrow \ y \in \ker f^{2n} = \ker f^n,
\]
پس $f^n(y) = 0$ و بنابراین $x = 0$.


\textbf{قضیه (لم فیتینگ).}  
اگر $M$ یک $R$-مدول با طول $n$ باشد و $f : M \to M$ یک $R$-همریختی، آنگاه:
\[
M = \operatorname{Im} f^n \ \oplus \ \ker f^n.
\]

\textbf{اثبات:}  
از $\ell(M) < \infty \Rightarrow M$ نوتری و آرتینی نتیجه می‌شود که:  

$\exists n_1$:  
\[
\operatorname{Im} f^{n_1} + \ker f^{n_1} = M.
\]  

$\exists n_2$:  
\[
\operatorname{Im} f^{n_2} \cap \ker f^{n_2} = 0.
\]  

اگر $m \geq \max\{n_1, n_2\}$ بگیریم، آنگاه:
\[
\operatorname{Im} f^m = \operatorname{Im} f^{n_1}, \quad \ker f^m = \ker f^{n_2}.
\]
پس:
\[
M = \operatorname{Im} f^m \ \oplus \ \ker f^m.
\]
