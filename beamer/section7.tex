\section{\lr{07 Hom}}


\begin{frame}
    \begin{theorem}
        فرض کنید \( R \) حلقه‌ای یک‌دار و نیمه‌ساده است. در این‌صورت \( R \) با جمع مستقیم تعداد متناهی ایده‌آل چپ مینیمال برابر است.
    \end{theorem}

    \textbf{اثبات.} دقت کنید که \( R \) بعنوان \( R \)-مدول نیمه‌ساده است و ضمناً هر زیرمدولی از \( R \) ایده‌آل چپ
    \( R \)
    درواقع یک
    است.
    بعلاوه، زیرمدول ساده است، هرگاه ایده‌آل چپ مینیمال باشد.
    اما
    \( R \) نیمه‌ساده‌ است، پس
    \[
        R = \bigoplus_{j \in J} I_j
    \]

\end{frame}

\begin{frame}
    که \( I_j \)ها ایده‌آل‌های چپ مینیمال \( R \) هستند. اما چون \( 1_R \in R \) پس
    \[
        \exists n : 1_R = r_1 + r_2 + \dots + r_n \quad (r_j \in I_{i_j})
    \]
    در نتیجه هر \( r \in R \) داریم
    \[
        r = r 1_R = r r_1 + r r_2 + \dots + r r_n \in \sum_{j=1}^n I_{i_j}
    \]
    و چون \( \oplus \) جمع مستقیم است(چرا؟)، پس
    \[
        R = \bigoplus_{j=1}^n I_{i_j}.
    \]
    \hfill\(\qed\)

\end{frame}


\begin{frame}
    \begin{definition}
        فرض کنید \( M \) و \( N \) دو \( R \)-مدول باشند. در این‌صورت مجموعه‌ی تمام \( R \)-همریختی‌ها از \( M \) به \( N \) را با نماد
        \[
            Hom(M,N)
        \]
        نشان می‌دهیم.

    \end{definition}
\end{frame}


\begin{frame}
    \begin{remark}
        \( Hom(M,N) \) با جمع معمولی \( R \)-همریختی‌ها یک گروه آبلی است، زیرا
        \[
            f,g \in Hom(M,N) \implies f+g \in Hom(M,N).
        \]


        اما اگر \( R \) حلقه‌ای جابجایی باشد، آنگاه \( Hom(M,N) \) یک \( R \)-مدول چپ نیز خواهد بود که ضرب اسکالر را به شکل زیر تعریف می‌کنیم:
        \[
            \forall r \in R, \quad f \in Hom(M,N) \implies rf:M\to N, \quad(rf)(m) := r f(m)
        \]
        و چون \( R \) جابجایی است، به ازای هر
        \(rf \in Hom(M,N)\)
        و
        \(r' \in R\)
        داریم
        \[
            (rf)(r'm) = r f(r'm) = r r' f(m) = r' r f(m) = r'(rf)(m).
        \]

    \end{remark}


\end{frame}



\begin{frame}
    \begin{definition}
        اگر \( M = N \) باشد، آنگاه
        \[
            End_R(M) := Hom(M,M)
        \]
        حلقه‌ی \( R \)-خودریختی‌های \( M \) نامیده می‌شود.


    \end{definition}
    \begin{remark}
        \( End_R(M) \) با عمل جمع و ترکیب توابع تشکیل یک حلقه‌  می‌دهد.
    \end{remark}

\end{frame}

\begin{frame}
    \begin{lemma}[{لم شُر:}]
        \begin{enumerate}
            \item اگر \( M \) یک \( R \)-مدول ساده باشد، در این‌صورت \( End_R(M) \) یک حلقه‌ی تقسیم است.
            \item فرض کنید \( M \) و \( N \) دو \( R \)-مدول ساده باشند، در این‌صورت:
                  \[
                      M \cong N \iff Hom(M,N) \neq \{0\}.
                  \]
        \end{enumerate}
    \end{lemma}


    \textbf{اثبات.}
    \begin{tblock}{}
        \begin{enumerate}
            \item فرض کنید
                  \[
                      0 \neq f \in End_R(M)
                  \]
                  دراین‌صورت
                  \(ker (f) \leqslant M\)
                  و چون
                  \( M \) ساده است، \(ker (f) = \{0\}\) یا
                  \(ker (f) = M\) که دومی ممکن نیست. پس
                  \(ker (f) = \{0\}\) و \( f \) یک‌به‌یک است. از طرفی
                  \(Im (f) \leqslant M\) و چون \( M \) ساده است،
                  \(Im (f) = M\)، پس
                  \( f \)
                  وارون‌پذیر
                  است.
                  بعلاوه
                  \( f^{-1} \)
                  یک
                  \(R\)-همریختی
                  است. پس
                  \( End_R(M) \)
                  یک حلقهٔ تقسیم است.
            \item با استدلال مشابه بالا.(چرا؟)
        \end{enumerate}
    \end{tblock}

\end{frame}


\begin{frame}
    \begin{theorem}
        فرض کنید
        \(M_1\)
        و
        \(M_2\)
        و
        \(N\)
        سه
        \(R\)-مدول
        باشند، در این صورت
        \begin{enumerate}
            \item \[
                      Hom(N, M_1 \oplus M_2) \cong Hom(N,M_1) \oplus Hom(N,M_2)
                  \]
                  که این همریختی،‌ یک همریختی
                  \(\mathbb{Z}\)-مدولی
                  است، و درصورتی که
                  \(R\)
                  حلقه‌ای جابجایی باشد، یک همریختی
                  \(R\)-مدولی
                  است.
            \item \[
                      Hom(M_1 \oplus M_2, N) \cong Hom(M_1,N) \oplus Hom(M_2,N)
                  \]
                  که این همریختی،‌ یک همریختی
                  \(\mathbb{Z}\)-مدولی
                  است، و درصورتی که
                  \(R\)
                  حلقه‌ای جابجایی باشد، یک همریختی
                  \(R\)-مدولی
                  است.
        \end{enumerate}
    \end{theorem}
\end{frame}

\begin{frame}
    \textbf{اثبات 1.}   تعریف می‌کنیم:
    \[
        \psi : Hom(N, M_1 \oplus M_2) \to Hom(N, M_1) \oplus Hom(N, M_2)
    \]
    \[
        f \mapsto (\pi_1 \circ f, \pi_2 \circ f)
    \]
    که در آن \( \pi_1, \pi_2 \) توابع تصویر روی مؤلفه‌ی اول و دوم هستند، یعنی:
    \[
        \pi_1 : Hom(N,M_1) \oplus Hom(N,M_2) \to Hom(N,M_1), \quad (a,b) \mapsto a
    \]
    \[
        \pi_2 : Hom(N,M_1) \oplus Hom(N,M_2) \to Hom(N,M_2), \quad (a,b) \mapsto b
    \]
    واضح است که
    \(\pi_1\)
    و
    \(\pi_۲\)
    \(R\)-همریختی
    هستند. چرا خوش‌تعریف است؟
    و می‌توان دید که \( \pi_1, \pi_2 \) \( R \)-همریختی هستند. (چرا؟)

\end{frame}



\begin{frame}
    \textbf{\(R\)-همریختی:}
    \[
        \begin{aligned}
            \psi(f+g) = & (\pi_1(f+g), \pi_2(f+g))               \\
            =           & (\pi_1(f)+\pi_1(g), \pi_2(f)+\pi_2(g)) \\
            =           & \psi(f) + \psi(g)
        \end{aligned}
    \]
    \[
        \left.
        \begin{aligned}
            \pi_1(rf) = r\pi_1(f), \\
            \pi_2(rf) = r\pi_2(f)
        \end{aligned}
        \right\}
        \implies \psi(rf) = r \psi(f)
    \]

\end{frame}


\begin{frame}
    \textbf{یک‌به‌یک بودن:}
    اگر \( f \in ker \psi \) آنگاه
    \[
        \psi(f) = 0 \implies
        \begin{cases}
            \pi_1 f = 0, \\
            \pi_2 f = 0.
        \end{cases}
    \]
    اگر \( f \neq 0 \) یعنی
    \[
        \exists n \in N : f(n) = (m_1,m_2) \neq (0,0)
    \] پس
    \[
        \pi_1 f \neq 0\text{ یا }\pi_2 f \neq 0
    \]
    پس \( f=0 \)،
    تناقض.


\end{frame}

\begin{frame}

    \textbf{پوشا بودن:}
    فرض کنید
    \[
        (g_1, g_2) \in Hom(N,M_1) \oplus Hom(N,M_2).
    \]
    تعریف می‌کنیم:
    \[
        \forall n; \;f(n) := (g_1(n), g_2(n)),
    \]
    ثابت می‌شود که
    \( f \in Hom(N, M_1 \oplus M_2)\)
    و
    \(\psi(f) = (g_1,g_2)\).(چرا؟)

\end{frame}

\begin{frame}
    \textbf{اثبات 2.}
    توابع زیر را تعریف می‌کنیم:
    \[
        \iota_1 : M_1 \to M_1 \oplus M_2, \quad m_1 \mapsto (m_1, 0)
    \]
    \[
        \iota_2 : M_2 \to M_1 \oplus M_2, \quad m_2 \mapsto (0, m_2)
    \]
    \[
        \psi : Hom(M_1 \oplus M_2, N) \to Hom(M_1,N) \oplus Hom(M_2,N)
    \]
    \[
        f \mapsto (f \circ \iota_1, f \circ \iota_2)
    \]
    \hfill\(\qed\)

    \begin{exercise}
        ادامه حل را کامل کنید.
    \end{exercise}

\end{frame}


\begin{frame}
    \begin{exercise}
        اگر \( \{M_i\}_{i\in I} \)ها \( R \)-مدول باشند و \( N \) هم \( R \)-مدول باشد:
        \begin{enumerate}
            \item
                  \[
                      Hom\left( \bigoplus_{i \in I} M_i, N \right) \cong \prod_{i \in I} Hom(M_i, N)
                  \]
            \item
                  \[
                      Hom\left( N, \prod_{i \in I} M_i \right) \cong \prod_{i \in I} Hom(N, M_i)
                  \]
                  (همریختی‌ها \(\mathbb{Z}\)-مدولی و اگر \( R \) جابجایی باشد \( R \)-مدولی‌اند)
        \end{enumerate}
    \end{exercise}


\end{frame}


\begin{frame}
    \begin{remark}
        اگر \( M \) یک \( R \)-مدول باشد، \(Hom(R,M)\) یک \( R \)-مدول چپ است (بدون نیاز به جابجایی \( R \)).
        با تعریف ضرب زیر:
        \[
            \forall r \in R, \; f \in Hom(R,M), \quad (r \cdot f)(a) := f(ar) \quad (\forall a \in R)
        \]
        به ازای هر
        \(a_1, a_2, r, s, a \in R\)
        داریم:
        \[
            (r \cdot f)(a_1+a_2) = f((a_1+a_2)r) = f(a_1r) + f(a_2r) = (r \cdot f)(a_1) + (r \cdot f)(a_2)
        \]
        \[
            (r \cdot f)(sa) = f((sa)r) = f(s(ar)) = s f(ar) = s (r \cdot f)(a)
        \]
        بررسی خواص
        \(R\)-مدولی
        بودن
        \(Hom(R,M) \) \(\cdots\)



    \end{remark}
\end{frame}


\begin{frame}
    \begin{theorem}
        اگر \( R \) حلقه‌ی یک‌دار و \( M \) یک \( R \)-مدول یکانی باشد آنگاه
        \[
            M \modulecong Hom(R, M)
        \]
    \end{theorem}
    \textbf{اثبات.} تعریف می‌کنیم:
    \[
        \psi : Hom(R, M) \to M, \quad f \mapsto f(1)
    \]
    \(\psi\) یک \( R \)-همریختی  است:
    \[
        \psi(f+g) = (f+g)(1) = f(1) + g(1) = \psi(f) + \psi(g)
    \]
    \[
        \psi(rf) = (rf)(1) = f(1r) = f(r) \stackrel{?}{=} r f(1) = r \psi(f)
    \]


\end{frame}

\begin{frame}
    \(\psi\) یک‌به‌یک است: اگر
    \[
        f \in ker(\psi) \implies\psi(f) = 0
    \]
    آنگاه
    \[
        f(1) = 0 \implies \forall r \in R, f(r) = r f(1) = 0 \implies f=0.
    \]
    \(\psi\) پوشا است: برای هر \( m \in M \)، تعریف می‌کنیم:
    \[
        \left.
        \begin{aligned}
            f_m(1) := m, \\
            f_m(r) := r m
        \end{aligned}
        \right\}
        \implies
        \begin{cases}
            f_m(r+r') = f_m(r) + f_m(r') \\
            f_m(sr) = sf(r)
        \end{cases}
    \]
    بنابراین\( f_m \in Hom(R,M) \) و \(\psi(f_m) = m\).

    \hfill\(\qed\)
\end{frame}



