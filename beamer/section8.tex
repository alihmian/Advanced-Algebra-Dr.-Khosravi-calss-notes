\section{\lr{08  direct sum}}


\begin{frame}
    \begin{theorem}
        فرض کنید $R$ حلقه‌ای نیمه‌ساده و یک‌دار باشد. در این صورت هر $R$-مدول ساده با ایده‌آل مینیمالی از $R$ یکریخت است.
    \end{theorem}
    \textbf{اثبات.}
    اگر
    \(M\)
    یک
    \(R\)-مدول
    مدول باشد، در این صورت
    \(a\in M\)
    وجود دارد که
    \[
        M = Ra \cong \frac{R}{Ann_R(a)} \cong \frac{R}{I}
    \]

    اما چون $R$ نیمه‌ساده است،
    پس
    مکمل پذیر است. ولذا
    \[
        \exists I' \leqslant R; \quad R = I \oplus I' \implies I \modulecong \frac{R}{I'} \modulecong M
    \]

\end{frame}
\begin{frame}
    \textbf{روش دوم.}
    با استفاده از قضایای قبل.
    فرض کنید $R = \bigoplus_{i=1}^n I_i$ که $I_i$ها ایده‌آل‌های مینیمال $R$ هستند و فرض کنید $M$ یک $R$-مدول ساده باشد. آنگاه بنابر قضیهٔ قبل
    \[
        M \cong Hom_R(R, M) = Hom_R \left( \bigoplus_{i=1}^n I_i , M \right)
        = \bigoplus_{i=1}^n Hom_R(I_i, M)
    \]
    اما $M$ و $I_i$ها ساده‌اند پس بنابر لم شور هر یک از $Hom_R(I_i, M)$ یا صفر است یا $M \cong I_i$. نتیجه اینکه اگر همه صفر باشند $M = 0$ (چرا؟). در غیر این صورت $M \cong I_i$ که $I_i$ یک ایده‌آل مینیمال $R$ است.

    نتیجه: تعداد متناهی ایده‌آل مینیمال از هر حلقهٔ نیمه‌ساده وجود دارد که در هر $R$-مدول ساده، با یکی از آن‌ها یک‌ریخت است.


\end{frame}




\begin{frame}
    \begin{remark}
        واضح است اگر $M$, $N$ دو $R$-مدول باشند، داریم:

        \[
            Hom_R(\{0\}, N) = \{0\}, \quad Hom_R(M, \{0\}) = \{0\}
        \]

    \end{remark}



    \begin{question}
        اگر $Hom_R(M, N) = \{0\}$، آیا $M$ یا $N$ حتماً باید صفر باشند؟
    \end{question}

\end{frame}


\begin{frame}
    \begin{theorem}{}
        فرض کنید $M$ یک $R$-مدول و $\{M_i\}_{i \in I}$ خانواده‌ای از $R$-مدول‌ها باشد.


        \[
            M \cong \bigoplus_{i \in I} M_i
        \]
        اگر و تنها اگر
        به‌ازای هر
        \(i\in I\)،
        {\(R\)-همریختی }های
        \(\lambda_i : M_i \to M\)
        موجود باشند، به‌طوری که، برای هر
        {\(R\)-مدول }
        \(\mathcal{X}\)
        و هر
        {\(R\)-همریختی }
        \((\forall i\in I) \; f_i : M_i \to \mathcal{X}\)
        بتوان
        {\(R\)-همریختی }
        یکتای
        \(\varphi: M \to \mathcal{X}\)
        را یافت، به طوری که به ازای هر
        \(i\in I\)،
        داشته باشیم
        \(\varphi\lambda_i=f_i\).

            {
                \large
                \[
                    \begin{tikzcd}[ampersand replacement=\&]
                        M_i \arrow{r}{\exists\lambda_i} \arrow[swap]{d}{f_i} \& M \arrow[dashed]{dl}{\exists! \varphi} \\
                        \mathcal{X} \&
                    \end{tikzcd}
                \]
            }

    \end{theorem}

\end{frame}



\begin{frame}
    \textbf{اثبات. }
    ابتدا فرض کنید
    $M \cong \bigoplus_{i \in I} M_i$.
    در این وضعیت به ازای هر \(i \in I\) نگاشت‌های طبیعی
    \[
        \lambda_i: M_i \to M, \quad m_i \mapsto \{m_j'\} \; s.t \;
        \begin{cases}
            m_j' = m_i & \; j = i    \\
            m_j' = 0   & \; j \neq i \\
        \end{cases}
    \]
    را درنظر بگیرید که به‌وضوح یک به یک هستند.
    نشان می‌دهیم شرط طرف دوم برقرار است.

    پس فرض کنید $\mathcal{X}$ یک $R$-مدول دلخواه و
    برای هر $i \in I$،
    $R$-همریختی های
    $f_i : M_i \to \mathcal{X}$
    مفروض باشند.
    نگاشت‌های
    $\pi_i : M \to M_i$
    را تصویر روی مؤلفهٔ
    \(i\)-ام
    در نظر بگیرید:
    \[
        \pi_i : M \to M_i, \quad \{m_j\} \mapsto m_i
    \]


\end{frame}


\begin{frame}
    و نگاشت $\varphi$ را به صورت زیر تعریف می‌کنیم:
    \[
        \begin{aligned}
            \varphi : M = \bigoplus_{i \in I} M_i \to \mathcal{X}, \quad \varphi = \sum_{i \in I} f_i \circ \pi_i \\
            \forall m \in M; \quad \varphi(m) := \sum_{i \in I} f_i(\pi_i(m))
        \end{aligned}
    \]

    آیا تعریف مشکل دار؟
    خیر، چون فقط تعداد متناهی تا از مؤلفه‌ها می‌توانند ناصفر باشند.

    اما در این صورت برای هر $a_j \in M_j$ و $j \in I$ داریم:
    \[
        \varphi  (\lambda_j(a_j))= \sum_{i \in I} f_i \circ \pi_i (\lambda_j(a_j))  = f_j(a_j)
    \]
    یعنی
    \[
        \forall j \in I, \quad \varphi \circ \lambda_j = f_j
    \]
\end{frame}

\begin{frame}
    \textbf{یکتایی؟}
    فرض کنید
    $R$-همریختی
    \(\psi\)
    هم همین شرایط را داشته باشد،‌یعنی
    \[
        \forall j \in I, \quad \psi \circ \lambda_j = f_j
    \]
    از آن‌جا که
    \[
        \sum_{i \in I} \lambda_i \circ \pi_i = \operatorname{id}_M
    \]
    داریم
    \[
        \psi\left( \sum_{i \in I} \lambda_i \circ \pi_i \right) = \psi(\operatorname{id}_M) = \psi
    \]
    از طرفی
    \(\psi\)
    یک
    $R$-همریختی
    است، لذا
    \[
        \psi
        = \psi\left( \sum_{i \in I} \lambda_i \circ \pi_i \right)
        = \sum_{i \in I} \psi (\lambda_i \circ \pi_i)
        = \sum_{i \in I} \psi \circ \lambda_i  (\pi_i)
        = \sum_{i \in I} f_i \circ \pi_i
        = \varphi
    \]
\end{frame}


\begin{frame}
    \textbf{برعکس:}
    فرض کنید
    $R$-همریختی
    های
    \(\lambda_i:M_i\to M\)
    موجودند، به‌طوری که خواص مطرح شده را دارند. چون برای هر
    $R$-مدول،
    \(\mathcal(X)\)
    شرط قضیه برقرار است، فرض می‌کنیم
    \(\mathcal{X} = \oplus_{i\in I}M_i\)
    و
    $R$-همریختی
    های
    \(f_i:M_i\to \mathcal{X}\)
    را هم نگاشت طبیعی درنظر می‌گیریم.

    \large
    \[
        \begin{tikzcd}[column sep=large, row sep=large, ampersand replacement=\&]
            \& M_i \arrow[r, "\lambda_i"] \arrow[ddl, "\lambda_i" swap] \arrow[d, "f_i" swap] \& M \arrow[dashed, dl, "\exists !\varphi"] \arrow[ddll, "\operatorname{id}_M" , bend left]\\
            \& \mathcal{X} = \oplus_{i \in I} M_i \arrow[dashed, dl, "\exists !\psi"]\& \\
            M \& \&
        \end{tikzcd}
    \]


\end{frame}


\begin{frame}

    {
        \large
        \[
            \begin{tikzcd}[column sep=large, row sep=large, ampersand replacement=\&]
                \& M_i \arrow[r, "\lambda_i"] \arrow[ddl, "\lambda_i" swap] \arrow[d, "f_i" swap] \& M \arrow[dashed, dl, "\exists !\varphi"] \arrow[ddll, "\operatorname{id}_M" , bend left]\\
                \& \mathcal{X} = \oplus_{i \in I} M_i \arrow[dashed, dl, "\exists !\psi"]\& \\
                M \& \&
            \end{tikzcd}
        \]
    }
    \textbf{از عبارات فوق:}
    \[
        \begin{cases}
            \varphi \circ \lambda_i = f_i \quad \forall i \\
            \psi \circ f_i = \lambda_i \quad \forall i
        \end{cases}
        \implies \psi \circ \varphi \circ \lambda_i = \lambda_i \quad \forall i
    \]

    چون $\operatorname{id}_M \circ \lambda_i = \lambda_i$ برای هر $i$، بنابراین از یکتایی داریم $\psi \circ \varphi = \operatorname{id}_M$ و در نتیجه
    \(\psi\)
    وارون‌پذیر و در نتیجه یک
    $R$-یک‌ریختی است. یعنی:
    \[
        M \cong \bigoplus_{i \in I} M_i
    \]

    \hfill\(\qed\)

\end{frame}


\begin{frame}
    به‌صورت مشابه ثابت می‌شود که:
    \begin{theorem}
        فرض کنید $M$ یک $R$-مدول و $\{M_i\}_{i \in I}$ خانواده‌ای از $R$-مدول‌ها باشد. آنگاه
        \[
            M \cong \bigoplus_{i \in I} M_i
        \]
        اگر و تنها اگر
        به‌ازای هر
        \(i\in I\)،
        {\(R\)-همریختی }های
        \(\pi_i : M_i \to M\)
        موجود باشند، به‌طوری که، برای هر
        {\(R\)-مدول }
        \(\mathcal{X}\)
        و هر
        {\(R\)-همریختی }
        \((\forall i\in I) \; f_i : M_i \to \mathcal{X}\)
        بتوان
        {\(R\)-همریختی }
        یکتای
        \(\varphi: \mathcal{X} \to M\)
        را یافت، به طوری که به ازای هر
        \(i\in I\)،
        داشته باشیم
        \(\pi_i\varphi=f_i\).
            {
                \large
                \[
                    \begin{tikzcd}[ampersand replacement=\&]
                        \&  M \arrow{d}{\pi_i}\\
                        \mathcal{X} \arrow[dashed]{ur}{\exists! \varphi} \arrow[swap]{r}{f_i} \&  M_i
                    \end{tikzcd}
                \]
            }
    \end{theorem}

    \textbf{اثبات.}(تمرین)
    \hfill \(\qed\)
\end{frame}

