\section{\lr{10 Notherian and Artinian modules}}




\begin{frame}
    \normalsize
    \begin{remark}[خارج از درس]
        $\mathbb{R}$
        به‌عنوان
        $\mathbb{Z}$-مدول
        آزاد نیست.
        (چرا که $\mathbb{Q}$ به‌عنوان $\mathbb{Z}$-مدول آزاد نیست.)
        [با کمک قضیهٔ ددکیند یا قضیهٔ \lr{Nielsen - Schreier}]
        (زیرگروه هر گروه آبلی آزاد، آزاد است.)
        $\mathbb{R}$
        به‌عنوان
        $\mathbb{Q}$-مدول
        آزاد است.
        (پایهٔ \lr{Hammel})


    \end{remark}
\end{frame}

\begin{frame}
    \begin{lemma}
        اگر $M$ یک $R$-مدول آزاد با پایه‌ای $n$ عضوی مانند $\{m_1, \dots, m_n\}$ باشد آنگاه:
        \[
            M \modulecong R^n
        \]
    \end{lemma}
    \textbf{اثبات.}
    \[
        \psi: R^n \to M, \quad (\alpha_1, \dots, \alpha_n) \mapsto  \sum_{i=1}^n \alpha_i m_i
    \]
    \[\cdots\]
    \hfill\(\qed\)

\end{frame}



\begin{frame}
    \begin{theorem}
        فرض کنید $M$ یک $R$-مدول آزاد و با تولید متناهی باشد. در این صورت هر پایه $M$ تعداد متناهی عضو دارد.
    \end{theorem}

    \textbf{اثبات.} فرض کنید $\{e_i\}_{i \in I}$ یک پایه برای $M$ باشد، اما چون $M$ با تولید متناهی است، پس
    \[
        \exists m_1, \dots, m_n; \quad M = \langle m_1, \dots, m_n \rangle
    \]
    هر $m_j$ را می‌توان به صورت ترکیب خطی متناهی از $e_i$ها نوشت:
    \[
        m_1 = r_{11} e_{i_1} + r_{12} e_{i_2} + \dots + r_{1k_1} e_{i_{k_1}}
    \]
    \[
        \vdots
    \]
    \[
        m_n = r_{n1} e_{n_1} + r_{n2} e_{n_2} + \dots + r_{n k_n} e_{n_{k_n}}
    \]
    مجموعهٔ تمام $e_i$های ظاهر شده را به صورت $T$ در نظر بگیرید. $T$ متناهی است و $M = \langle T \rangle$. چون $\{e_i\}$ پایه است، $T$ مستقل خطی است و پس $T$ یک پایه متناهی است.

    \hfill\(\qed\)
\end{frame}



\begin{frame}
    \begin{theorem}
        فرض کنید $M$ یک $R$-مدول آزاد و با تولید متناهی باشد. در این صورت هر دو پایهٔ $M$ تعداد برابری عضو دارد.
    \end{theorem}


    \textbf{اثبات.} فرض کنید:
    \[
        S_1 = \{ e_i \}_{i=1}^n, \quad S_2 = \{ e_j' \}_{j=1}^m
    \]
    دو پایه برای $M$ باشند. آنگاه:
    \[
        M \cong R^n, \quad M \cong R^m \quad \implies  R^n \modulecong R^m
    \]

    در این صورت
    {$R$-همریختی یک به یک و پوشا}
    \( \varphi :R^m \to R^n \)
    و
    {$R$-همریختی یک به یک و پوشا}
    \(\psi :R^n \to R^m \quad \)
    وجود دارد به‌نحوی که
    \[
        \left.
        \begin{aligned}
            A_{n \times m} = [\varphi]_{S_1, S_2} \\
            B_{m \times n} = [\psi]_{S_2, S_1}
        \end{aligned}
        \right\}
        \; s.t \quad
        \begin{aligned}
            AB = I_{n \times n}, \\
            BA = I_{m \times m}
        \end{aligned}
    \]
\end{frame}



\begin{frame}
    حال $\bar{A} = \begin{bmatrix} A & 0 \end{bmatrix}$ و $\bar{B} = \begin{bmatrix} B \\ 0 \end{bmatrix}$ را تعریف می‌کنیم. آنگاه:
    \[
        \left.
        \begin{aligned}
            \bar{A} \bar{B} = AB = I_n, \\
            \bar{B} \bar{A} =
            \begin{bmatrix}
                I_m & 0 \\
                0   & 0
            \end{bmatrix}
        \end{aligned}
        \right\}
        \implies
        det(\bar{A} \bar{B}) = det(\bar{B} \bar{A}) \implies 0 = 1
    \]
    \hfill
    \(\qed\)

\end{frame}


\section{مدول‌های نوتری و آرتینی}

\begin{frame}

    \begin{definition}
        $R$-مدول $M$ را یک \textit{مدول نوتری} نامیم هرگاه هر زنجیر صعودی از $R$-زیرمدول‌های $M$ متوقف شود.
        \textbf{مثال.} $\mathbb{Z}$ به عنوان $\mathbb{Z}$-مدول نوتری است زیرا تمام زیرمدول‌های آن به شکل $\langle m \rangle$ هستند و:
        \[
            n \mid m \ \Leftrightarrow\ \langle m \rangle \subseteq \langle n \rangle
        \]
        ولذا هر زنجیر صعودی از زیرمدول‌ها متوقف می‌شود.
    \end{definition}

    \begin{definition}
        $R$-مدول $M$ را یک \textit{مدول آرتینی} نامیم هرگاه هر زنجیر نزولی از زیرمدول‌های $M$ متوقف شود.
    \end{definition}
\end{frame}


\begin{frame}

    \begin{example}
        $\mathbb{Z}$ به عنوان $\mathbb{Z}$-مدول آرتینی نیست زیرا:
        \[
            \mathbb{Z} \supsetneq \langle 2 \rangle \supsetneq \langle 4 \rangle \supsetneq \langle 8 \rangle \supsetneq \dots
        \]
        و هیچگاه متوقف نمی‌شود.

    \end{example}

    \begin{example}
        هر گروه آبلی متناهی به عنوان $\mathbb{Z}$-مدول هم نوتری است و هم آرتینی.
    \end{example}




\end{frame}

\begin{frame}
    \begin{example}
        به عنوان $\mathbb{Z}$-مدول، $\mathbb{Q}$ نه نوتری است و نه آرتینی زیرا:
        \[
            \mathbb{Q} \supsetneq \langle 2 \rangle \supsetneq \langle 4 \rangle \supsetneq \langle 8 \rangle \supsetneq \dots
        \]
        و
        \[
            \langle \tfrac12 \rangle \subsetneq \langle \tfrac14 \rangle \subsetneq \langle \tfrac18 \rangle \subsetneq \dots
        \]
    \end{example}

    \begin{example}
        هر فضای برداری با بعد متناهی هم نوتری است و هم آرتینی.
    \end{example}

\end{frame}
\begin{frame}
    \begin{definition}
        فرض کنید $p$ عددی اول باشد. تعریف کنید:
        \[
            M = \left\{ \frac{a}{p^n} \ \middle| \ n \in \mathbb{N},\ a \in \mathbb{Z} \right\}
        \]
        $(M, +)$ یک گروه آبلی است.  بعلاوه:
        \[
            (\mathbb{Z}, +) \subsetneq M \subsetneq (\mathbb{Q}, +)
        \]
        گروه خارج‌قسمتی $\frac{M}{\mathbb{Z}}$ را در نظر بگیرید. این گروه را $\mathbb{Z}_{p^\infty}$ می‌نامیم.

    \end{definition}

\end{frame}

\begin{frame}
    \begin{example}
        $\mathbb{Z}_{p^\infty}$ به عنوان $\mathbb{Z}$-مدول آرتینی است ولی نوتری نیست.
    \end{example}
    \textbf{اثبات.}
    برای اثبات، تعریف کنید:
    \[
        G_n = \left\{ \frac{a}{p^n} + \mathbb{Z} = \overline{\frac{a}{p^n}} \in \mathbb{Z}_{p^\infty} \ \middle| \ a \in \mathbb{Z} \right\}
    \]
    ادعا:


    \begin{tblock}{}
        \begin{enumerate}
            \item هر $G_n$ زیرگروهی از $\mathbb{Z}_{p^\infty}$ است،
            \item به‌ازای هر ‌‌\(‌n\) داریم، \(G_n \subsetneq G_{n+1} \)،
            \item \(\mathbb{Z}_{p^\infty} = \bigcup_{n=1}^\infty G_n\)،
            \item  هر زیرمدول ناصفر $\mathcal{N} \leqslant \mathbb{Z}_{p^\infty}$ برابر با $G_n$ برای یک $n$ است.
        \end{enumerate}

    \end{tblock}

\end{frame}

\begin{frame}
    برای این منظور ابتدا ثابت می‌کنیم که اگر
    \(\overline{\frac{a}{p^n}} \in \mathcal{N}\)
    هرگاه
    \(p \nmid a\)
    در این صورت
    \(\overline{\frac{1}{p^n}} \in \mathcal{N}\)
    \begin{tblock}{}
        \[
            \begin{aligned}
                (a, p) = 1 \  & \implies\ (a, p^n) = 1 \                                   \\
                              & \implies\ \exists r, s \in \mathbb{Z}, \ ra + sp^n = 1     \\
                              & \implies \overline{\frac{a}{p^n}} \in \mathcal{N}          \\
                              & \implies\ r \cdot \overline{\frac{a}{p^n}} \in \mathcal{N} \\
                              & \implies\ \overline{\frac{1 - sp^n}{p^n}} \in \mathcal{N}  \\
                              & \implies\ \overline{\frac{1}{p^n}} - s \in \mathcal{N}     \\
                              & \implies \overline{\frac{1}{p^n}} \in \mathcal{N}          \\
                              & \implies\ G_n \subseteq \mathcal{N}.
            \end{aligned}
        \]

    \end{tblock}
\end{frame}


\begin{frame}

    اگر بزرگ‌ترین $n$ را در نظر بگیریم که به‌ازای یک ‌\(a\)، $\overline{\frac{a}{p^n}} \in \mathcal{N}$ آنگاه \(G_n \subseteq \mathcal{N}\) و اگر $\mathcal{N} \neq G_n$، آنگاه:
    \[
        \exists m > n; \quad \ \overline{\frac{a}{p^m}} \in \mathcal{N}, \quad (a, p) = 1
    \]
    که تناقض است.
    و اگر چنین
    \(n\)ای
    وجود نداشته‌باشد
    $$\mathcal{N} = \mathbb{Z}_{p^\infty}.$$

    \textbf{نتیجه:} فقط زنجیر زیر در $\mathbb{Z}_{p^\infty}$ وجود دارد:
    \[
        G_0 \subsetneq G_1 \subsetneq G_2 \subsetneq G_3 \subsetneq \dots
    \]
    که نشان می‌دهد نوتری نیست ولی آرتینی هست.

    \hfill\(\qed\)
\end{frame}

\begin{frame}
    \begin{theorem}
        اگر $M$ یک $R$-مدول یکانی باشد، گزاره‌های زیر معادل‌اند:
        \begin{enumerate}
            \item $M$ یک $R$-مدول نوتری است.
            \item هر زیرمدول $M$ با تولید متناهی است.
            \item هر مجموعهٔ ناتهی از زیرمدول‌های $M$ دارای یک عضو ماکسیمال است.
        \end{enumerate}

    \end{theorem}
    \textbf{اثبات:} \textbf{(1) \(\impliedby \) (2)}
    فرض کنید $N \leq M$.
    اگر $N = 0$ آنگاه نتیجه برقرار است.
    اگر $N \neq 0$، آنگاه $ x_1 \in N$ ناصفر وجود دارد  به طوری که اگر
    \(N = \langle x_1 \rangle \) تمام است.
    در غیر این صورت $ x_2 \in N \setminus \langle x_1 \rangle$ ناصفر وجود دارد.
    به همین ترتیب اگر $N = \langle x_1, x_2 \rangle$، تمام است وگرنه $\exists x_3 \in N \setminus \langle x_1, x_2 \rangle$ و ادامه دادن این فرآیند:
    \[
        \langle 0 \rangle \subsetneq\langle x_1 \rangle \subsetneq \langle x_1, x_2 \rangle \subsetneq \dots
    \]
    زنجیر صعودی بی‌نهایت می‌دهد که با نوتری بودن در تناقض است، پس این فرآیند متوقف می‌شود.


\end{frame}




\begin{frame}

    \textbf{(2) \(\impliedby\) (3):}
    فرض کنید $\{ M_i \}_{i \in I}$ مجموعه‌ای از زیرمدول‌های $M$ باشد. اگر $M_1$ ماکسیمال نباشد، $M_1 \subsetneq M_2$ وجود دارد  و به همین ترتیب:
    \[
        M_1 \subsetneq M_2 \subsetneq M_3 \subsetneq \dots
    \]
    قرار دهید
    \[
        \mathcal{N} = \bigcup_{i=1}^{+infty} M_i
    \]
    در این صورت
    \(\mathcal{N}\)
    زیر مدولی از
    \(M\)
    و
    کران بالای تمام
    \(M_i\)
    ها است. طبق فرض
    \(n_k, \cdots , n_1\)
    وجود دارد به‌نحوی که
    \[
        \mathcal{N} = \langle n_1, \cdots , n_k \rangle
    \]
    لذا هر
    \(n_i\)
    در
    \(M_{j_i}\)
    قرار خواهد گرفت.
    اگر قرار دهید
    \[
        k = max\{j_i \mid 1 \leqslant i \leqslant n\}
    \]
    در این صورت
    \(\mathcal{N} = M_k\).





\end{frame}


\begin{frame}
    \textbf{(3) \(\impliedby\) (1):}
    فرض کنید
    \[
        M_1 \subseteq M_2 \subseteq M_3 \subseteq \dots
    \]
    یک زنجیر صعودی از زیرمدول‌ها باشد.
    اگر
    \[
        \bigcup_{i=1}^\infty M_i
    \]
    را در نظر بگیریم، این یک زیرمدول $M$ است که ماکسیمال در این زنجیر وجود دارد، پس زنجیر متوقف می‌شود.
    \hfill\(\qed\)

\end{frame}

\begin{frame}
    \begin{definition}
        $R$-مدول $M$ را \textbf{ متناهیاً تولید شده} گوییم هرگاه
        چنانچه برای
        \(R\)-زیرمدول‌های
        \(\{M_i\}_{i\in I}\)
        داشته باشیم
        \(M = \sum_{i\in I} M_i\)
        بتوان زیرمجموعه‌ٔ متناهی
        \(J \subseteq I\)
        را یافت به‌نحوی که
        \(M = \sum_{i\in J} M_i\).
        (معادل بودن با تعریف قبل؟)

    \end{definition}
    \begin{definition}
        $R$-مدول $M$ را \textbf{ متناهیاً هم‌تولید شده} گوییم هرگاه
        چنانچه برای
        \(R\)-زیرمدول‌های
        \(\{M_i\}_{i\in I}\)
        داشته باشیم
        \( \bigcap_{i\in I} M_i = 0\)
        بتوان زیرمجموعه‌ٔ متناهی
        \(J \subseteq I\)
        را یافت به‌نحوی که
        \( \bigcap_{i\in J} M_i = 0\).

    \end{definition}
\end{frame}
