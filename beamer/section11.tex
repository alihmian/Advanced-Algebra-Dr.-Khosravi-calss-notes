\section{\lr{11 Notherian and Artinian modules (2)}}

\begin{frame}
    \begin{example}
        $\mathbb{Z}$ به عنوان $\mathbb{Z}$-مدول با تولید متناهی است.
        ولی متناهیاًهم‌تولید شده نیست.
        \[
            \bigcap_{p\ \text{اول}} pZ = 0 \quad \implies \quad ?
        \]

    \end{example}
    \begin{example}
        $\mathbb{Z}_p^\infty$ متناهیاًهم‌تولید شده است ولی با تولید متناهی نیست.
    \end{example}
\end{frame}

\begin{frame}
    \begin{theorem}
        برای هر $R$-مدول $M$ موارد زیر معادلند:
        \begin{enumerate}
            \item $M$ آرتینی است.
            \item هر مدول خارج‌قسمتی $M$  متناهیاًهم‌تولید شده است.
            \item هر مجموعهٔ ناتهی از زیرمدول‌های $M$ دارای عنصر مینیمالی است.
        \end{enumerate}
    \end{theorem}
    \textbf{اثبات.} (1) $\impliedby$ (2):
    فرض کنید $N \leqslant M$ و $\frac{M}{N}$ را در نظر بگیرید. اگر
    \[
        \bigcap_{i\in I} \frac{M_i}{N} = 0 \implies \bigcap_{i\in I} M_i = N.
    \]
    ثابت می‌کنیم تعدادی متناهی از اعضای
    $I$
    هستند که
    \[
        \bigcap_{1 \leqslant i \leqslant k} M_i = N
    \]
\end{frame}



\begin{frame}{ادامهٔ اثبات}
    برای این منظور دقت کنید:
    \[
        M_1 \supsetneq M_1 \cap M_2 \supsetneq M_1 \cap M_2 \cap M_3 \supsetneq \dots
    \]
    ولی در این صورت چون $M$ آرتینی است پس:
    \[
        \exists k \quad s.t \quad \bigcap_{i=1}^k M_i = \bigcap_{i\in I} M_i = N
    \]
    پس حکم ثابت شد یعنی $\frac{M}{N}$ متناهیاًهم‌تولید‌شده‌است.

\end{frame}


\begin{frame}{ادامهٔ اثبات}

    (2) $\impliedby$ (3):
    خانواده $\{M_i\}_{i\in I}$ از زیرمدول‌های $R$-مدول $M$ را در نظر بگیرید.
    \begin{tblock}{}
        \begin{itemize}
            \item [] اگر \(M_1\) مینیمال باشد.\(\checkmark\)
            \item [] در غیر این‌صورت، \(M_2 \subsetneq M_1\) وجود دارد. اگر \(M_2\) مینیمال باشد.\(\checkmark\)
            \item [] در غیر این‌صورت، \(M_3 \subsetneq M_2\) وجود دارد. اگر \(M_3\) مینیمال باشد.\(\checkmark\)
            \item [] \[\vdots\]
        \end{itemize}
    \end{tblock}

    با تکرار این فرایند،‌ زنجیرهٔ
    \[
        \dots  \subsetneq  M_3 \subsetneq M_2 \subsetneq M_1
    \]
    بدست می‌آید.

\end{frame}

\begin{frame}{ادامهٔ اثبات}
    قرار دهید
    \[
        N = \bigcap_{i=1}^k M_i \leqslant M.
    \]
    در این صورت در
    \(\frac{M}{N}\)
    داریم
    \(\bigcap_{i \in I} \frac{M_i}{N} = 0\).
    و چون
    \(\frac{M}{N}\)
    متناهیاًهم ‌تولیدشده است،‌ پس
    \[
        \exists k \quad s.t \quad \bigcap_{i = 1}^{k} \frac{M_i}{N} = 0
    \]
    یعنی
    \(N = M_k\)
    چون تو در تو بود و لذا
    \(N = M_k = M_{k+1}\)

    \bigskip
    (3) $\impliedby$ (1):
    واضح است.

\end{frame}


\begin{frame}
    \begin{theorem}
        اگر $N \leqslant M$،

        \[
            \text{ نوتری باشند.  } \frac{M}{N} \text{  و } N \iff  \text{ نوترى است  } M
        \]

    \end{theorem}
    \textbf{اثبات.}
    اگر
    \(M\)
    نوترى
    باشد،‌ در اینصورت زنجیرهٔ
    \[
        N_1 \subseteq N_2 \subseteq N_3 \subseteq \dots
    \]
    از زیرمدول‌های
    \(N\)
    در واقع زنجیره‌ای از زیرمدول‌های
    \(M\)
    هم هستند. پس متوقف می‌شود.

    به صورت مشابه هر زنجیره از زیرمدول‌های
    \(\frac{M}{N}\)
    به شکل
    \[
        \frac{M_1}{N} \subsetneq \frac{M_2}{N} \subsetneq \frac{M_3}{N} \subsetneq \dots
    \]
    است که از نوتری بودنِ
    \(M\)
    متوقف می‌شود.

\end{frame}


\begin{frame}{ادامهٔ اثبات}
    \textbf{برعکس.}
    \(N\)
    و
    \(\frac{M}{N}\)
    نوتری هستند. ثابت می‌کنیم هر زیرمدول
    \(K\)
    از
    \(M\)
    با تولید متناهی است.

    از آن‌جا که
    \(\frac{K+N}{N}\)
    زیر مدولی از
    \(\frac{M}{N}\)
    است و
    \(\frac{M}{N}\)
    نوتری است،‌
    \(\frac{K+N}{N}\)
    با تولید متناهی است. با کمک قضیهٔ دوم یکریختی:
    \[
        \frac{K+N}{N} \cong \frac{K}{K \cap N}
    \]
    و درنتیجه به‌َعنوان
    \(R\)-مدول
    \[
        \frac{K}{K \cap N} = \langle k_1 + K \cap N, \dots, k_\ell + K \cap N \rangle \implies K = \langle k_1, \dots, k_\ell \rangle + K \cap N
    \]
    از آنجا که
    \( K \cap N\)
    زیرمدولی از
    \(N\)
    است،‌ پس با تولید متناهی است. لذا
    \[
        K \cap N = \langle n_1, \dots, n_m \rangle
    \]
    و بنابراین
    \[
        K = \langle n_1, \dots, n_m, k_1, \dots, k_\ell \rangle
    \]
    با تولید متناهی است و
    \(M\)
    نوتری است.
    $M$ نوترى $\implies$ نتیجه واضح است.
    \hfill\(\qed\)
\end{frame}

\begin{frame}
    \begin{theorem}
        اگر $N\le M$، از آرتینی بودنِ $M$ نتیجه می‌شود $N$ و $\dfrac{M}{N}$ آرتینی‌اند (دقیقاً همان اثباتِ قبل).

    \end{theorem}
    برعکس نیز برقرار است، ولی ابتدا یک لم را ثابت می‌کنیم.
    \begin{lemma}
        فرض کنید $N_1,N_2,N_3$ زیرمدول‌های یک $R$-مدولِ $M$ باشند به‌طوری‌که
        \[
            \left.
            \begin{aligned}
                N_2\subseteq N_1, \\
                N_1+N_3=N_2+N_3,  \\
                N_1\cap N_3=N_2\cap N_3.
            \end{aligned}
            \right\}
            \implies
            N_1=N_2.
        \]
    \end{lemma}

    \textbf{اثبات.}
    \[
        N_1
        = N_1\cap (N_1+N_3)
        = N_1\cap (N_2+N_3)
        = (N_1\cap N_2)+(N_1\cap N_3)
        = N_1+(N_1\cap N_3)
        = N_2
    \]
    \hfill\(\qed\)

\end{frame}



\begin{frame}
    \begin{theorem}
        اگر $N\leqslant M$ و $N$ و $\dfrac{M}{N}$ آرتینی باشند، آنگاه $M$ آرتینی است.
    \end{theorem}
    \textbf{اثبات.} یک زنجیرهٔ نزولی از زیرمدول‌های $M$ را در نظر بگیرید
    \[
        M_1 \supsetneq M_2 \supsetneq M_3 \supsetneq \cdots .
    \]
    در این‌صورت
    \[
        M_1 + N \supsetneq M_2 + N \supsetneq M_3 + N \supsetneq \cdots .
    \]
    و لذا
    \[
        \frac{M_1+N}{N}\ \supsetneq\ \frac{M_2+N}{N}\ \supsetneq\ \frac{M_3+N}{N}\ \supsetneq\ \cdots
    \]
    در $\dfrac{M}{N}$ نزولی است؛
\end{frame}



\begin{frame}

    چون $\dfrac{M}{N}$ آرتینی است، $\exists k$ به‌طوری‌که برای هر $i\ge 0$
    \[
        \frac{M_k+N}{N}=\frac{M_{k+i}+N}{N}\qquad\Longrightarrow\qquad M_k+N=M_{k+i}+N.
    \]
    از طرفی زنجیرهٔ
    \[
        M_1\cap N \ \supsetneq\ M_2\cap N \ \supsetneq\ M_3\cap N \ \supsetneq\ \cdots
    \]
    در $N$ نزولی است؛ چون $N$ آرتینی است، $\exists t$ به‌طوری‌که برای هر $j\ge 0$
    \[
        M_t\cap N = M_{t+j}\cap N.
    \]
    اکنون اگر $s\ge max\{k,t\}$  ، برای هر $i\ge 0$ داریم
    \[
        \left.
        \begin{aligned}
            M_{s+i}\subseteq M_s, \\
            M_{s+i}+N=M_s+N,      \\
            M_{s+i}\cap N=M_s\cap N.
        \end{aligned}
        \right\}
        \xRightarrow[]{\footnotesize\text{از لم قبل}} \forall i; \; M_s = M_{s+i} \implies \text{آرتینی است} M.
    \]
    \hfill\(\qed\)

\end{frame}

\begin{frame}
    \begin{theorem}
        اگر $\{M_i\}_{i\in I}$ خانواده‌ای از $R$-مدول‌های ناصفر باشد، آنگاه:
        \[
            \left.
            \begin{aligned}
                \text{ متناهی است،  } I \\
                \text{ نوتری(آرتینی) است. } M_i \text{ هر  }
            \end{aligned}
            \right\}
            \iff
            \text{  نوترى (آرتینی) است.  } \bigoplus_{i\in I} M_i
        \]
    \end{theorem}
    \textbf{اثبات.}
    اگر $\bigoplus_{i\in I} M_i$ آرتینی باشد و $I$ نامتناهی باشد، دنبالهٔ
    \[
        \bigoplus_{i\in I} M_i \supsetneq \bigoplus_{i\in I\setminus\{i_1\}} M_i \supsetneq \bigoplus_{i\in I\setminus\{i_1,i_2\}} M_i \supsetneq \cdots
    \]
    بی‌نهایت نزولی تشکیل می‌دهد که متوقف نمی‌شود، تناقض.

    اگر
    \(\bigoplus_{i\in I} M_i \)
    نوتری و
    \(I\)
    نامتناهی باشد،‌ می‌توان نوشت
    \[
        \{0\} \subsetneq M_1 \subsetneq M_1 \oplus M_2 \subsetneq \cdots
    \]
    که تناقض است.

\end{frame}



\begin{frame}
    \textbf{برعکس.}

    حال اگر $I$ متناهی و هر $M_i$ نوتری (آرتینی) باشد، داریم:
    \[
        \frac{M_1 \oplus M_2 }{M_1} \modulecong M_2
    \]
    پس اگر
    \(M_1\)
    و
    \(M_2\)
    نوتری (آرتینی) باشد، در این‌صورت
    \(M_1 \oplus M_2\)
    نوتری (آرتینی) خواهد بود.
    با تکرار
    \(k\)
    مرحله، حکم ثابت می‌شود:
    \[
        \bigoplus_{i=1}^k M_i = \left(\bigoplus_{i=1}^{k-1} M_i\right) \bigoplus M_k
    \]
    \hfill\(\qed\)

\end{frame}

\begin{frame}
    \begin{theorem}
        فرض کنید $M=\sum_{i\in I} M_i$ که $I$ متناهی و $M_i$ها زیرمدول‌های $M$ هستند.
        آنگاه:
        \[
            \text{ نوترى (آرتینی) است. } M  \iff \forall i;\; \text{ نوترى (آرتینی) است. } M_i
        \]
    \end{theorem}
    \textbf{اثبات.}
    اگر $M$ نوترى (آرتینی) باشد، چون $M_i \leqslant M$، نتیجه می‌شود $M_i$ نوترى (آرتینی) است.

\end{frame}


\begin{frame}
    \textbf{برعکس،} فرض کنید هر $M_i$ نوترى (آرتینی) باشد. در این صورت نگاشت
    \[
        \begin{aligned}
            \varphi:\ \bigoplus_{i=1}^n M_i     \quad & \to      \quad \sum_{i=1}^n M_i   \\
            (m_1,m_2,\dots,m_n)                 \quad & \mapsto  \quad m_1+m_2+\cdots+m_n
        \end{aligned}
    \]
    یک هم‌ریختی پوشا است و داریم
    \[
        \frac{\bigoplus_{i=1}^n M_i}{ker(\varphi)} \ \cong\ \sum_{i=1}^n M_i.
    \]
    اگر هر $M_i$ نوترى (آرتینی) باشد، حاصل‌جمع مستقیم متناهی $\bigoplus_{i=1}^n M_i$ نیز نوترى (آرتینی) است.
    چون خارج‌قسمت یک مدول نوترى (آرتینی) نیز نوترى (آرتینی) است، نتیجه می‌شود $\sum_{i=1}^n M_i$ نوترى (آرتینی) است.

    \hfill\(\qed\)
\end{frame}

\begin{frame}
    \begin{theorem}
        اگر $M$ یک $R$-مدول ساده باشد، موارد زیر معادلند:
        \begin{enumerate}
            \item $M$ با تولید متناهی است.
            \item $M$ آرتینی است.
            \item $M$ نوترى است.
        \end{enumerate}
    \end{theorem}

    \textbf{اثبات.}
    (1) $\impliedby$ (2): بدیهی است چون هر مدول با بُعد متناهی آرتینی است.
\end{frame}

\begin{frame}
    (2) $\impliedby$ (1): اگر $M$ آرتینی باشد و ساده، داریم

    \[
        \text{نیمه‌ساده است.} M
        \implies
        \left.
        \begin{aligned}
            (\text{ ساده}M_i) \;M = \bigoplus_{i\in I} M_i \\
            M = \langle \alpha_1,\alpha_2, \cdots , \alpha_k \rangle
        \end{aligned}
        \right\}
        \implies
        \text{ متناهی است.(چرا؟)  } I
    \]
    هر
    \(\alpha_i\)
    مجموعی متناهی از تعدادی عناصر ناصفر
    \(M_i\)
    ها است، بنابر این اندیس‌های ظاهر شده (؟)

    اما
    \[
        \text{ساده} M_i \implies \text{نوتری و آرتینی} M_i \implies \text{متناهی} I \implies \text{نوتری و آرتینی} M
    \]
\end{frame}
\begin{frame}
    (2) \(\impliedby\) (3)
    \[
        \left.
        \begin{aligned}
            (\text{ ساده}M_i) \;M = \bigoplus_{i\in I} M_i \\
            \text{آرتینی است.} \;M
        \end{aligned}
        \right\}
        \implies
        \text{متناهی} \; I
        \implies
        \text{نوتری} \; M.
    \]

    (3) \(\impliedby\) (1)
    واضح است.(قضیه)

    \hfill
    \(\qed\)
\end{frame}


\begin{frame}
    \begin{theorem}
        هر $R$-مدول آرتینی ناصفر دارای حداقل یک زیرمدول ساده است.
    \end{theorem}
    \textbf{اثبات.} یک زنجیرهٔ نزولی از زیرمدول‌ها بگیرید:
    \[
        M_1 \supsetneq M_2 \supsetneq M_3 \supsetneq \dots
    \]
    به‌علت آرتینی بودن، این زنجیره متوقف می‌شود و مینیمم به‌دست می‌آید که ساده است.
    \hfill
    \(\qed\)
\end{frame}



\begin{frame}
    \begin{theorem}
        اگر $R$ یک حلقهٔ یک‌دار باشد، موارد زیر معادلند:
        \begin{enumerate}
            \item $R$ به عنوان $R$-مدول چپ، نوترى (آرتینی) است.
            \item هر $R$-مدول با تولید متناهی، نوترى (آرتینی) است.
        \end{enumerate}
    \end{theorem}
    \textbf{اثبات.} (1) $\impliedby$ (2): اگر $M=\langle m_1,\dots,m_n\rangle$، نگاشت
    \[
        \varphi: R^n \to M,\quad (r_1,\dots,r_n) \mapsto r_1m_1+\dots+r_nm_n
    \]
    یک هم‌ریختی پوشا است، بنابراین  $M \cong R^n/ker(\varphi)$.
    چون $R$ نوترى (آرتینی)
    است،‌ بنابراین
    $R^n$
    نوترى (آرتینی)
    است و خارج‌قسمت یک مدول نوترى (آرتینی) نیز نوترى (آرتینی) است، $M$ نوترى (آرتینی) خواهد بود.

    (2) $\impliedby$ (1):
    با قرار دادن
    \(R = 1\)
    حکم بدست می‌آید.
    \hfill
    \(\qed\)
\end{frame}




\begin{frame}
    \begin{theorem}
        اگر
        $R$
        حلقه‌ای
        یک‌دار و به‌عنوان
        $R$-مدول
        چپ آرتینی است.
        اگر
        $M=0$
        یکریختی
        $R$-مدول
        دلخواه باشد،
        آنگاه $M$ حداقل یک زیرمدول ساده دارد.
    \end{theorem}
    \textit{اثبات.} چون $M\neq 0$، یک $0\neq x \in M$ انتخاب کنید و بگذارید $N=Rx$. این $N$ یک زیرمدول غیرصفر و با تولید متناهی است و به‌علت آرتینی بودن، زنجیرهٔ نزولی از زیرمدول‌های $N$ متوقف می‌شود، بنابراین یک زیرمدول مینیمم (ساده) دارد.
\hfill
    \(\qed\)
\end{frame}





