% \section{\lr{04 Modules}}

% \subsection*{}


% \begin{definition}[مدول]
%     فرض کنید $R$ یک حلقه باشد.
%     $(M,+)$ یک گروه آبلی باشد، همانند فضاهای برداری  یک ضرب اسکالر به‌صورت زیر تعریف می‌کنیم:
%     \[
%         \forall r \in R,\ \forall m \in M; \quad r.m \in M
%     \]

%     که در خواص زیر صدق کند:
%     $$\forall m_1, m_2 \in M,\ \forall r \in R;\quad r(m_1 + m_2) = rm_1 + rm_2$$
%     $$\forall r, s \in R,\ \forall m \in M;\quad (r+s)m = rm + sm$$
%     $$\forall r, s \in R,\ \forall m \in M;\quad (rs)m = r(sm)$$


%     در این صورت $M$ را یک
%     \textbf{$R$-مدول چپ}
%     می‌نامیم.
%     به‌صورت مشابه،
%     \textbf{$R$-مدول راست}
%     تعریف می‌شود.

% \end{definition}

% \begin{definition}[$R$-مدول چپ یکانی]
%     اگر حلقه $R$
%     یکدار باشد
%     و $M$،
%     یک
%     $R$-مدول چپ
%     باشد که
%     \[
%         \forall m \in M;\quad 1.m = m
%     \]
%     آنگاه $M$ را یک
%     $R$-مدول چپ یکانی
%     می‌گویند.

% \end{definition}


% \begin{example}
%     هر فضای برداری بر میدان $F$ یک $F$-مدول چپ یکانی است.
% \end{example}


% \begin{example}
%     هر حلقه دلخواه $R$، با ضرب:
%     \[
%         \forall r, s \in R:\quad r.s = rs
%     \]
%     یک $R$-مدول چپ می‌شود.
% \end{example}



% \begin{example}
%     هر گروه آبلی یک $\mathbb{Z}$-مدول چپ است.
% \end{example}


% \[
%     n \cdot a =
%     \begin{cases}
%         a + \cdots + a    & n > 0 \\
%         0                 & n = 0 \\
%         -(a + \cdots + a) & n < 0
%     \end{cases}
% \]



% \begin{example}
%     هر گروه آبلی $(M,+)$ را می‌توان با تعریف \(r \cdot m = 0\) ، به یک $R$-مدول تبدیل کرد.
% \end{example}

% \begin{example}
%     $\mathbb{R}[x]$ به‌عنوان یک $\mathbb{R}$-مدول چپ.
% \end{example}


% \begin{example}
%     $\mathbb{R}$ به‌عنوان $\mathbb{Z}$-مدول.
% \end{example}

% \begin{exercise}
%     $\mathbb{Z}$ به‌عنوان $\mathbb{R}$-مدول؟
% \end{exercise}


% \begin{remark}
%     اگر $R$ جابجایی باشد، هر $R$-مدول چپ را می‌توان به‌صورت $R$-مدول راست درنظر گرفت، با تعریف:
%     \[
%         m \cdot r := r \cdot m
%         \quad
%     \]
% \end{remark}
% (شرط جابجایی نیاز بود؟)


% \begin{remark}
%     اگر $\varphi: R \to S$ یک هم‌ریختی حلقه‌ای باشد و $M$ یک $S$-مدول چپ باشد، آنگاه $M$، یک $R$-مدول چپ هم هست، به‌صورت:
%     \[
%         r \cdot m := \varphi(r) \cdot m
%     \]
% \end{remark}


% \begin{definition}[زیرمدول]
%     $M$ یک $R$-مدول چپ است، زیرگروه $(N,+) \subseteq (M,+)$ را یک زیرمدول $M$ گوییم هرگاه:
%     \[
%         \forall r \in R,\ \forall n \in N;\quad r \cdot n \in N
%     \]
% \end{definition}

% \begin{example}
%     \(\{0\}\)
%     و
%     \(M\)
%     زیرمدول‌های بدیهی هستند
% \end{example}


% \begin{definition}[{مدول خارج‌قسمتی}]
%     اگر $N$ یک زیرمدول از $M$ باشد،
%     در این صورت
%     $N$
%     زیرگروه نرمال
%     $M$
%     است و لذا همدسته‌ها در
%     $M/N$
%     تشکیل گروه جمعی می‌دهند.
%     تعریف می‌کنیم:
%     \[
%         r \cdot (m + N) := r \cdot m + N
%     \]
%     در این صورت $M/N$ یک $R$-مدول چپ می‌شود. (چرا؟)
% \end{definition}

% آیا برای هر زیرمدول صحیح است؟


% \begin{definition}[{$R$-هم‌ریختی}]
%     اگر $M_1$ و $M_2$ دو $R$-مدول چپ باشند، آنگاه نگاشت
%     \[
%         \varphi : M_1 \longrightarrow M_2
%     \]
%     یک \textbf{$R$-هم‌ریختی} نامیده می‌شود هرگاه:
%     \begin{align*}
%         \text{(1)}\quad & \forall m, m' \in M_1;\ \varphi(m + m') = \varphi(m) + \varphi(m')            \\
%         \text{(2)}\quad & \forall m \in M_1,\ \forall r \in R;\ \varphi(r \cdot m) = r \cdot \varphi(m)
%     \end{align*}
%     به همین نحو
%     \textbf{$R$-تک‌ریختی}،
%     \textbf{$R$-بروریختی}و
%     \textbf{$R$-یک‌ریختی}
%     تعریف می‌شود.

% \end{definition}


% \begin{theorem}
%     اگر $\varphi : M \to M'$ یک $R$-هم‌ریختی باشد، آنگاه:
%     \begin{itemize}
%         \item $ker \varphi$ یک زیرمدول از $M$ است.
%         \item $Im  \varphi$ یک زیرمدول از $M'$ است.
%     \end{itemize}

% \end{theorem}


% \begin{theorem}[{قضیه اول هم‌ریختی مدول‌ها}]
%     اگر $\varphi : M \to M'$ یک $R$-هم‌ریختی باشد، آنگاه:
%     \[
%         \frac{M}{ker \varphi} \modulecong Im  \varphi
%     \]
% \end{theorem}


% \begin{theorem}[{قضیه دوم هم‌ریختی مدول‌ها}]
%     اگر $N$ و $K$ دو زیرمدول از $R$-مدول $M$ باشند، در این‌صورت:
%     \[
%         \frac{N + K}{N} \modulecong \frac{K}{N \cap K}
%     \]


% \end{theorem}

% \begin{proof}
%     {تعریف می‌کنیم:}
%     \[\varphi : K \to \frac{N + K}{N}, \quad k \mapsto k + N\]

% \end{proof}

% \begin{lemma}
%     اگر $\{ M_i \}_{i \in I}$ خانواده‌ای از زیرمدول‌های $R$-مدول $M$ باشند، آنگاه
%     \(\bigcap_{i \in I} M_i\)
%     نیز زیرمدول $M$ است.
% \end{lemma}


\begin{frame}
    \begin{definition}
        اگر $M$ یک $R$-مدول باشد و $X \subseteq M$ آنگاه کوچکترین $R$-زیرمدول $M$ که شامل $X$ باشد را زیرمدول تولیدشده توسط $X$ می‌نامند و آن را با $\langle X \rangle$ نشان می‌دهند.
        اگر $X = \{ a \}$، می‌نویسیم $\langle a \rangle$.

        اگر
        \(X = \emptyset\)
        به‌وضوح
        \(\langle X \rangle = \{0\}\).
        اگر
        \(X = {a}\)
        آنگاه
        \(\langle a \rangle\)
        یک زیرمدول دوری
        \(M\)
        نامیده می‌شود. اگر
        \(X\)
        متناهی باشد
        $\langle X \rangle$
        با تولید متناهی نامیده می‌شود.
    \end{definition}

    \begin{lemma}
        با فرضیات فوق داریم:
        \[
            \langle X \rangle = \bigcap_{\substack{N \leqslant M ,  X \subseteq N}} N
        \]
    \end{lemma}


\end{frame}


\begin{frame}
    \begin{example}
        در $\mathbb{Z}$ به‌عنوان $\mathbb{Z}$-مدول:
        \[
            \langle n \rangle = n \mathbb{Z}
        \]

    \end{example}

    \begin{example}
        در $\mathbb{Z}_{30}$ به‌عنوان $\mathbb{Z}$-مدول:
        \[
            \langle \overline{5} \rangle = \{ \overline{0}, \overline{5}, \overline{10}, \overline{15}, \overline{20}, \overline{25} \}
        \]
        \[
            \langle \overline{4} \rangle = \{ \overline{0}, \overline{4}, \overline{8}, \overline{12}, \overline{16}, \overline{20}, \overline{24}, \overline{28} \}
        \]
    \end{example}


\end{frame}

\begin{frame}

    \begin{definition}
        اگر $R$ حلقه جابجایی و یکدار و $M$ یک $R$-مدول باشد، برای $a \in M$ تعریف می‌کنیم:
        \[
            Ra = \{ ra \mid r \in R \}
        \]
        که یک زیرمدول از $M$ خواهد بود.

    \end{definition}

\end{frame}

\begin{frame}
    \begin{remark}
        آیا $\langle a \rangle = Ra$ همواره برقرار است؟
    \end{remark}


    \[
        Ra = \{ ra \mid r \in R \}
        \qquad ,
        \langle a \rangle = \{ ra + na \mid r \in R,\ n \in \mathbb{Z} \}
    \]

    ولی اگر حلقه
    یکدار باشد،‌ یکی هستند.
    به عنوان مثال حلقهٔ غیر یکدار
    \(R = (5\mathbb{Z}, +, \cdot)\)
    را در نظر بگیرید.
    به وضوح
    \(\mathbb{Z}\)
    یک
    \(R\)-مدول
    است. حالا اگر
    \(a = 3\)
    \[
        Ra = \{ ra \mid r \in R \} = 15\mathbb{Z}
    \]
    در حالی که
    \(\langle 3 \rangle = 3 \mathbb{Z}\). البته در ادامه همواره حلقه های را جابجایی و یکدار فرض می‌کنیم.

    \begin{remark}
        با
        تعریف
        \[
            \varphi : R \to Ra, \quad r \mapsto ra
        \]
        به یک
        $R$-بروریختی
        می‌رسیم.

    \end{remark}

\end{frame}



\begin{frame}
    \begin{theorem}
        فرض کنید $R$ حلقه‌ای یکدار باشد. در این صورت،
        \(R\)-مدول
        یکانی
        \(M\)
        دوری است اگر و تنها اگر
        \(M \cong \frac{R}{I}\)
        که در آن
        \(I\)
        یک ایده‌آل چپ
        \(R\)
        است.
        [در واقع \(I = Ann_R(a) = \{ r \in R \mid ra = 0 \}\)]
    \end{theorem}

    \[
        M = \langle a \rangle \Rightarrow \frac{R}{Ann_R(a)} \modulecong M = Ra
    \]

\end{frame}


\begin{frame}
    \textbf{اثبات.}
    چون $R$ یکدار است و $M$ یکانی است، پس
    \[
        \exists a \in R,\quad M = Ra
    \]

    در این صورت
    \[
        \varphi : R \longrightarrow M = Ra, \quad r \longmapsto ra
    \]
    $\varphi$ یک $R$-بروریختی  است. بنابراین:
    \[
        \frac{R}{ker \varphi} \modulecong Im  \varphi = M
    \]
    و واضح است که
    \[
        ker \varphi = Ann_R(a)
    \]


\end{frame}

\begin{frame}

    \textbf{برعکس.}
    اگر $M \cong \frac{R}{I}$ که $I$ یک ایده‌آل چپ از $R$ باشد، فرض کنید:

    \[
        \varphi : \frac{R}{I} \longrightarrow M, \quad \varphi(r + I) = \varphi(r)
    \]
    $R$-یکریختی
    موردنظر باشد.
    در این صورت با فرض $a = \varphi(1 + I)$ داریم
    \[
        \varphi(r + I) = \varphi(r(1 + I)) = r \varphi(1 + I) = ra
    \]
    و لذا $M$ توسط $\{a\}$ تولید می‌شود.

\end{frame}


\begin{frame}
    \begin{definition}
        اگر $M$ یک $R$-مدول و $\{M_i\}_{i \in I}$ زیرمدول‌های آن باشند، مجموع زیرمدول‌ها را به‌صورت زیر تعریف می‌کنیم:

        \[
            \sum_{i \in I} M_i := \left\{ a_1 + a_2 + \cdots + a_n \mid n \in \mathbb{N},\ a_j \in M_{i_j} \right\}
        \]



    \end{definition}

    \begin{lemma}
        واضح است که
        \[
            \sum_{i \in I} M_i = \left\langle \bigcup_{i \in I} M_i \right\rangle
        \]
    \end{lemma}

\end{frame}

\begin{frame}
    \begin{definition}
        فرض کنید $\{M_i\}_{i \in I}$ خانواده‌ای از $R$-مدول‌ها باشند.
        در این صورت دنباله‌هایی به شکل $\{m_i\}_{i \in I}$ را در نظر می‌گیریم که در آن $m_i \in M_i$. سپس عملیات‌های زیر را تعریف می‌کنیم:

        \[
            \{m_i\}_{i \in I} + \{m_i'\}_{i \in I} := \{m_i + m_i'\}_{i \in I}
            \quad ; \quad
            r \cdot \{m_i\}_{i \in I} = \{r m_i\}_{i \in I}
        \]
        با این جمع و ضرب تعریف‌شده، یک $R$-مدول بدست می‌آید که آن را حاصل‌ضرب مستقیم $M_i$ها می‌خوانند و با نماد
        \[
            \prod_{i \in I} M_i
        \]
        نمایش می‌دهیم.

    \end{definition}

\end{frame}



\begin{frame}
    حال اگر فقط دنباله‌هایی به شکل
    $\{M_i\}_{i \in I}$
    را در نظر بگیریم که فقط تعداد متناهی از عناصر آن‌ها
    ناصفرند،
    به زیرمجموعه‌ای از قبلی می‌رسیم که خود یک
    $R$-مدول
    است و آن را
    حاصل‌جمع مستقیم
    $M_i$ ها
    گویند و با علامت
    \(\bigoplus_{i \in I} M_i\)
    نمایش داده می‌شود.

    واضح است که
    $\bigoplus_{i \in I} M_i$
    زیرمدولی از
    $\prod_{i \in I} M_i$
    است.
    اگر $I$ متناهی باشد:
    \[
        \prod_{i \in I} M_i = \bigoplus_{i \in I} M_i.
    \]

\end{frame}

\begin{frame}
    \begin{remark}
        حال اگر فرض کنیم $\{M_i\}_{i \in I}$ خانواده‌ای از زیرمدول‌های یک $R$-مدول $M$ باشند، در این صورت با تعریف
        \[
            \varphi : \bigoplus_{i \in I} M_i \longrightarrow \sum_{i \in I} M_i \; , \; \{m_i\}_{i \in I} \longmapsto \sum_{i \in I} m_i
        \]
        \[
        \]
        یک $R$-همریختی پوشا می‌رسیم و لذا:
        \[
            \sum_{i \in I} M_i \ \cong_{\text{همریختی}-R} \ \frac{\bigoplus_{i \in I} M_i}{ker \varphi}.
        \]

    \end{remark}

    اما در این صورت $ker \varphi = \{0\}$ به چه معنا است؟
    \[
        \begin{aligned}
            ker \varphi = \{0\} \ \iff & \text{نمایش  منحصر بفرد}                                                            \\
            \iff                        & \left( m_{i_1} + \cdots + m_{i_k} = 0 \ \Rightarrow \ \forall j : m_{i} = 0 \right) \\
            \iff                        & \quad M_i \ \cap \ \sum_{\substack{j \in I                                          \\ j \neq i}} m_j \ = \ \{0\}.
        \end{aligned}
    \]

\end{frame}

\begin{frame}

    \begin{theorem}[\lr{Internal direct sum}]
        فرض کنید $M$ یک $R$-مدول است و $\{M_i\}_{i \in I}$ خانواده‌ای از زیرمدول‌های $M$ باشند. در این صورت $M$ را \textbf{مجموع مستقیم} زیرمدول‌های $M_i$ می‌نامند هرگاه:
        \[
            M = \sum_{i \in I} M_i
        \]
        و هر عنصر $M$ را بتوان به‌صورت
        منحصر بفردی به شکل مجموع عناصری از
        \(M_i\)
        نوشت.
        در این صورت می‌نویسیم
        \[
            M = \bigoplus_{i \in I} M_i.
        \]

    \end{theorem}

\end{frame}


\begin{frame}
    \begin{corollary}
        اگر $\{M_i\}_{i \in I}$ خانواده‌ای از $R$-زیرمدول‌های $M$ باشند، گزاره‌های زیر معادلند:
        \begin{enumerate}
            \item[(1)]
                  \[
                      M = \bigoplus_{i \in I} M_i
                  \]
            \item[(2)]
                  \[
                      M_i \ \cap \ \sum_{\substack{j \in I \\ j \neq i}} M_j \ = \ \{0\} \quad (\forall i \in I)
                  \]
        \end{enumerate}
    \end{corollary}

    \textbf{اثبات.}
    اگر
    \[
        a \in M_i \ \cap \ \sum_{\substack{j \in I \\ j \neq i}} M_j
    \]
    پس
    \[
        a = m_i = m_{t_1} + m_{t_2} + \cdots + m_{t_k}
    \]
\end{frame}

\begin{frame}

    در این صورت:
    \[
        m_i - m_{t_1} - m_{t_2} - \cdots - m_{t_k} = 0 + 0 + \cdots + 0
    \]
    و با توجه به نمایش یکتا
    $$a = m_i = 0$$

    برعکس، اگر نمایش یک عنصر به‌ دو صورت توشته شود:
    \[
        m_{t_1} + m_{t_2} + \cdots + m_{t_k} \ = \ m'_{s_1} + m'_{s_2} + \cdots + m'_{s_r}
    \]
    با یکی گرفتن این اندیس‌ها نتیجه می‌گیریم که
    \[
        m_{t_1} - m'_{t_1}
        = (m'_{t_2} - m_{t_2}) + \cdots + (m'_{t_k} - m_{t_k})
    \]
    که
    \[
        m_{t_1} - m'_{t_1} \in M_{t_1},
        \quad
        (m'_{t_2} - m_{t_2}) + \cdots + (m'_{t_k} - m_{t_k})
        \in \sum_{\substack{j \in I \\ j \neq t_1}} M_j
    \]
    لذا صفر می‌شود و پس $m_{t_1} = m'_{t_1}$ و همین‌طور برای بقیهٔ اعضا.

\end{frame}

\begin{frame}
    \begin{remark}
        فرض کنید
        \(R\)
        حلقه‌ای جابجایی و یک‌دار است.
        اگر $M$ یک $R$-مدول و $X$ یک مجموعهٔ مولد $M$ باشد آنگاه خواهیم داشت:
        \[
            M = \sum_{x \in X} Rx.
        \]

    \end{remark}


    \begin{definition}
        زیرمدول $N$ از $R$-مدول $M$ را \textbf{ماکسیمال} گوییم هرگاه $N \neq M$ و اگر
        \[
            N \subseteq K \subseteq M
        \]
        که $K$ یک $R$-زیرمدول $M$ است، آنگاه
        \[
            K = N \quad \text{یا} \quad K = M.
        \]

    \end{definition}



\end{frame}

\begin{frame}
    \begin{example}
        $\mathbb{Z}$ به عنوان $\mathbb{Z}$-مدول و
        \[
            N = p\mathbb{Z}
        \]
        که $p$ عددی اول است.

    \end{example}
\end{frame}