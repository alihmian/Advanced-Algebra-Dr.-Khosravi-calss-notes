\section{\lr{06 simple and semisimple modules}}


\begin{frame}
    \begin{theorem}
        فرض کنید \( M \) یک \( R \)-مدول مکمل‌پذیر باشد و \( N \) زیرمدولی از \( M \) باشد. در این صورت \( N \) و \( M/N \) هر دو مکمل‌پذیرند.

    \end{theorem}


    \textbf{اثبات.} فرض کنید \( K \leq M \) زیرمدولی از \( N \) باشد و چون \( M \) مکمل‌پذیر است، پس
    \[
        \exists \mathcal{L} \leq M \text{ که } M = K \oplus \mathcal{L} .
    \]
    در این صورت
    \[
        N = N \cap M = N \cap (K \oplus \mathcal{L} ) \mathrel{\overset{{{?}}}{=}} (N \cap K) \oplus (N \cap \mathcal{L} ).
    \]


\end{frame}



\begin{frame}
    ثابت  می‌کنیم:
    فرض کنید
    \[
        x \in (N \cap K) \oplus (N \cap L),
    \]
    در این صورت
    \[
        x = x_1 + x_2, s.t
        \left.
        \begin{aligned}
            x_1 \in N \cap K \\
            x_2 \in N \cap L
        \end{aligned}
        \right\}
        \implies
        \begin{cases}
            x = x_1 + x_2 \in N, \\
            x = x_1 + x_2 \in K \oplus L
        \end{cases}
        \implies x \in N \cap (K \oplus L).
    \]

\end{frame}

\begin{frame}
    برعکس: اگر
    \[
        x \in N \cap (K \oplus L),
    \]
    در این صورت \( x \in N \) و \( x = k + \ell \) که \( k \in K \), \( \ell \in L \).
    از آنجا که \( x \in N \), و \( k \in K \leqslant N\), لذا
    \[
        \ell = x - k \in N \implies \ell \in N \cap L
    \]
    \[
        \implies k \in N \cap K = K, \quad \ell \in N \cap L \implies x \in (N \cap K) \oplus (N \cap L)
    \]

    پس این دو مجموعه برابرند، لذا
    \[
        N = (K \cap N) \oplus (L \cap N) =  K \oplus (L \cap N)
    \]
    یعنی
    \( K \)
    در
    \( N \)
    مکمل‌پذیر است.



\end{frame}


\begin{frame}
    اما  بجای اینکه نشان دهیم \( M/N \) مکمل‌پذیر است، با توجه به مکمل‌پذیری \( M \) زیرمدول \( N' \) وجود دارد که
    \[
        M = N \oplus N'.
    \]
    تعریف می‌کنیم
    \[
        \varphi : M \to N', \quad m = n + n' \longmapsto n'.
    \]

    در این صورت \( \varphi \) یک \( R \)-همریختی است و \( ker \varphi = N \) پس
    \[
        M/N \modulecong N'
    \]
    حالا چون \( N' \) مکمل‌پذیر است، نتیجه می‌گیریم \( M/N \) مکمل‌پذیر است.
    \hfill 
    \(\qed\)
\end{frame}


\begin{frame}
    \begin{question}
        اگر \( N \) و \( M/N \) مکمل‌پذیر باشند، آیا الزاماً نتیجه می‌دهد \( M \) مکمل‌پذیر است؟

    \end{question}
    \textbf{مثال نقض.}

    حلقه‌ی زیر را در نظر بگیرید:
    \[
        S = \frac{R[x]}{\langle x^2 \rangle}
    \]
    مدول \( \frac{S}{(x)} \) ساده است، و \( (x) \) هر دو
    \(S\)-مدول‌های
    تحویل‌ناپذیرند .
    اما
    \(S\)
    به‌عنوان
    \( S \)-مدول
    نیمه‌ساده نیست چون \( \langle x \rangle \) تنها زیرمدول ماکسیمال آن است.

    \hfill 
    \(\qed\)
\end{frame}


\begin{frame}
    \begin{theorem}
        هر مدول مکمل‌پذیر ناصفر شامل یک زیرمدول ساده است.
    \end{theorem}
    \textbf{اثبات.}
    درواقع کافی است که حکم را برای
    \(R\)-مدول‌های
    دوری ثابت کنیم زیرا
    \[
        0 \neq a \in M \implies 0 \neq Ra \leqslant M.
    \]
    و ضمناً
    \(Ra\)
    هم مکمل‌پذیر است. اما می‌دانیم
    \[
        Ra \modulecong \frac{R}{\mathop{Ann}(a)} = \frac{R}{I}
    \]
    که
    \(I\)
    یک ایده‌آل چپ
    \(R\)
    است. پس در ادامه فرض کنید
    \(M = Ra\)
    بنابر بحث‌های قبل ایده‌آل ماکسیمال چپ
    \(J\)
    از
    \(R\)
    یافت می‌شود که
    \(I \subseteq J\)
    و لذا
    \( \frac{J}{I} \lneq \frac{R}{I} \)
    که نتیجه می‌دهد
    \(\frac{J}{I}\)
    یک
    \(R\)-زیرمدول
    ماکسیمال
    \(\frac{R}{I}\)
    است، پس
    \(M\)
    هم زیرمدولی نظیر
    \(\frac{J}{I}\)
    دارد
    (مانند \(N\))
    ولی
    \(M\)
    مکمل‌پذیر است. پس
    \[
        \exists N' \leq M \quad \text{و} \quad M = N \oplus N'
    \]
    اما
    \( M/N \cong N' \)
    و
    \( M/N \)
    ساده است. پس
    \( N' \)
    ساده است.

    \hfill 
    \(\qed\)

\end{frame}




\begin{frame}
    \begin{theorem}
        فرض کنید \( M \) یک \( R \)-مدول و \( M = \sum_{i \in I} M_i \) که \( M_i \)ها زیرمدول‌های ساده \( M \) هستند. اگر
        \( N \leqslant M \)
        ثابت کنید که
        \(J \subseteq I\)
        یافته می‌شود که
        \[
            N = \bigoplus_{j \in J} M_j
        \]
    \end{theorem}
    \textbf{اثبات.}
    \[
        \exists J_0 \subseteq I \quad s.t \quad M = \bigoplus_{i \in J_0} M_i.
    \]


\end{frame}


\begin{frame}
    بعلاوه
    \( N \leqslant M \)
    پس وجود دارد
    \( J_1 \subseteq J_0 \)
    که
    \[
        M = N \oplus \left( \bigoplus_{i \in J_1} M_i \right)
    \]
    اما از طرفی
    \[
        M = \left( \bigoplus_{i \in J_1} M_i \right) \oplus \left( \bigoplus_{i \in J_0 \setminus J_1} M_i \right)
    \]
    و شبیه بحث قبل
    \[
        N \cong \frac{M}{\bigoplus_{i \in J_1} M_i} \cong \bigoplus_{i \in J_0 \setminus J_1} M_i
    \]
    \hfill 
    \(\qed\)
\end{frame}

\begin{frame}
    با توجه به بحث‌های بالا داریم:

    \begin{theorem}
        \( M \) یک \( R \)-مدول است. موارد زیر معادلند:
        \begin{enumerate}
            \item \( M = \sum_{i \in I} M_i \) که \( M_i \)ها ساده‌اند.
            \item \( M = \bigoplus_{i \in J} M_i \) و \( M_i \)ها ساده‌اند.
            \item \( M \) مکمل‌پذیر است.
        \end{enumerate}

    \end{theorem}
    \textbf{اثبات.}
    \begin{enumerate}
        \item [ (1) \(\Leftarrow\) (2)] قبلاً بحث شد.
        \item [ (2) \(\Leftarrow\) (3)] قبلاً بحث شد.


    \end{enumerate}



\end{frame}

\begin{frame}

    (3) \(\Leftarrow\) (1) اگر \( M \neq 0 \) چون مکمل‌پذیر، بنابراین حتماً یک زیرمدول ساده دارد.

    تعریف می‌کنیم:
    \[
        N = \sum_{\substack{P \leqslant M \\  \text{ ساده} P}} P,
    \]
    \begin{claim}
        \[N = M\]
    \end{claim}

    \textbf{برهان خلف:}
    اگر \( N \lneq M \)، آنگاه
    \[
        \exists N' \leqslant M \quad \text{و} \quad M = N \oplus N'
    \]
    اما \( N' \) مکمل‌پذیر (چرا؟) و بنابراین شامل زیرمدول ساده \( P' \) است. در این صورت \( P' \leqslant N \cap N'\)، پس \( M = N \).
    \hfill 
    \(\qed\)
\end{frame}


\begin{frame}
    \begin{definition}
        \( R \)-مدول \( M \) را \textit{نیمه‌ساده (یا کاملاً تجزیه‌پذیر)} می‌نامیم هرگاه در یکی از شرایط قضیه‌ی قبل صدق کند.

    \end{definition}
    \begin{theorem}
        حلقه‌ی یک‌دار \( R \) را در نظر بگیرید که  به عنوان \( R \)-مدول نیمه‌ساده است. در این صورت هر \( R \)-مدول نیمه‌ساده است.
    \end{theorem}
    \textit{اثبات.} اگر \( M \) دوری باشد،
    \[
        \text{نیمه‌ساده} R \implies \text{مکمل‌پذیر} \implies \exists I' \leqslant R ; \; R = I \oplus I' \implies \frac{R}{I} \cong I'
    \]
    \[
        \exists a; \quad M = Ra \implies M \cong \frac{R}{I}
    \]


\end{frame}


\begin{frame}
    بنابراین
    \[
        M \cong I' \text{مکمل‌پذیر}
        \implies M \text{نیمه‌ساده}
    \]
    اما در حالت کلی
    \[
        M = \sum_{a \in M} Ra \implies M \text{نیمه‌ساده}
    \]
    چرا که
    \(Ra\)
    مجموعی از ساده‌ها و
    \(\sum_{a \in M} Ra\)
    مجموعی از ساده‌ها است.

    \hfill 
    \(\qed\)
\end{frame}

\begin{frame}
    \begin{definition}
        حلقه \( R \) را \textit{نیمه‌ساده‌ی } گوییم هرگاه بعنوان \( R \)-مدول نیمه‌ساده باشد.
    \end{definition}

    \begin{theorem}
        حلقه‌ی ناتهی و نیمه‌ساده‌ی \( R \) با جمع مستقیم تعداد متناهی ایده‌آل چپ مینیمال یکتاست.
    \end{theorem}





\end{frame}

\begin{frame}
    \textbf{اثبات.} یک \( R \)-زیرمدول سادهٔ \( R \) بعنوان \( R \)-مدول در واقع
    همان
    ایده‌آل چپ مینیمال است، پس
    \[
        R = \bigoplus_{i \in J} I_i
    \]
    که \( I_i \)ها ایده‌آل چپ مینیمال \( R \) هستند. اما \( 1 \in R \)
    \[
        1 = r_1 + r_2 + \dots + r_n
    \]
    که
    \[
        \forall m ; \; r_m \in I_{j_m}
    \]
    پس
    \[
        \forall r \in R ; \; r = r r_1 + r r_2 + \dots + r r_n
    \]
    به‌ طوری که
    \(rr_n \in I_{i_n}, \cdots, rr_1 \in I_{i_1}\)،
    پس
    \[
        R = \sum_{m=1}^n I_{j_m}
    \]
    و بعلاوه جمع مستقیم است(چرا؟)
    و لذا حکم ثابت شد.


    \hfill 
    \(\qed\)
\end{frame}