\section{\lr{03 Introduction}}

\begin{frame}
    \begin{exercise}
        فرض کنید \( R \) حلقه‌ای جابجایی و یکدار است. ثابت کنید موارد زیر معادلند:
        \begin{enumerate}
            \item \( R \) فقط یک ایده‌آل اول دارد.
            \item هر عنصر  \( R \) یا یکه است یا عنصری پوچتوان است.
            \item \( \dfrac{R}{nil(R)} \) یک میدان است.
        \end{enumerate}
    \end{exercise}

    \begin{definition}
        حلقه‌ای را که دقیقا یک ایده‌آل ماکسیمال داشته باشد را یک
        \textbf{حلقهُ موضعی}
        نامند.
    \end{definition}


\end{frame}

\begin{frame}
    \begin{example}
        میدان
        \(\mathbb{F} \)
        یک حلقهُ موضعی است،
        چرا که در هر میدان ایده‌آل نابدیهی وجود ندارد.
    \end{example}

    \begin{example}
        اگر
        \(p\)
        عددی اول باشد،‌ در این صورت حلقه های زیر نیز موضعی هستند
        \[\frac{\mathbb{Z}}{p\mathbb{Z}}, \quad\frac{\mathbb{Z}}{p^2\mathbb{Z}}, \quad\frac{\mathbb{Z}}{p^3\mathbb{Z}}, \quad\cdots,\quad \frac{\mathbb{Z}}{p^n\mathbb{Z}}\]
    \end{example}


    \begin{example}
        حلقهُ
        \(\mathbb{C}[[x]]\)،
        یعنی مجموعهٔ تمام عناصر
        \(\Sigma_{i=0}^{+\infty} c_ix^i\)
        که در آن
        \(c_i \in \mathbb{C}\)
        است، با ضرب معمولی موضعی است.
    \end{example}

\end{frame}

\begin{frame}
    \begin{example}
        حلقه زیر:
        \[
            R = \left\{ \frac{m}{n} \in \mathbb{Q} \mid \text{ فرد است } n \right\}
        \]
        حلقه‌ای موضعی است چرا که تنها ایده‌آل ماکسیمال آن ایده‌آل زیر است:
        \[
            \mathfrak{m} = \left\{ \frac{m}{n} \in R \mid \text{ زوج است } m \right\}
        \]
    \end{example}


\end{frame}

\begin{frame}
    \begin{theorem}
        \(R\)
        حلقه ای موضعی است اگر و تنها اگر برای هر
        \(r\in R\)
        عنصر
        \(r\)
        یا
        \(1-r\)
        عنصری یکه باشد.
    \end{theorem}

    \textbf{اثبات.}
    اگر
    \(R\)
    حلقه‌ای
    موضعی باشد،‌ فقط یک ایده‌آل اول مانند
    \(m\)
    دارد، از طرفی هر عنصر
    \(a \notin m\)
    عنصری یکه است زیرا
    \(m \subsetneq m + (a) = R\).
    لذا اگر
    \(r \in R\)
    از آنجا که
    \(r \notin m\)
    یا
    \(1-r \notin m\)، پس
    \(r \)
    یا
    \(1-r \)
    یکه است.

    حال، اگر
    \(M\)
    و
    \(M'\)
    دو ایده‌آل ماکسیمال
    \(R\)
    باشند؛ فرض کنید
    \(a \in M' \setminus M\)
    پس
    \[
        R = M + (a) \implies \exists m \in M, \exists r \in R; \quad 1 = m + ra \implies m = 1 - ra
    \]
    از طرفی
    \[
        a \in M' \implies ra \in M' \implies \text{ یکه است } 1 - ra
    \]
    پس
    \(m \in M\)
    عنصری یکه بوده، که در تناقض با ماکسیمال بودن
    \(M\)
    است.
\hfill \(\qed\)
\end{frame}

\begin{frame}
    \begin{corollary}
        \(\mathbb{R}[x]\)
        حلقه ای موضعی نیست،‌ چرا که
        \(x\)
        و
        \(1-x\)
        هر دو غیر یکه هستند.
    \end{corollary}

    \begin{corollary}
        \begin{enumerate}
            \item فرض کنید
                  \(R\)
                  یک حلقه و
                  \(\mathfrak{m}  \neq (1)\)
                  ایده‌آلی از
                  \(R\)
                  باشد، به طوری که هر
                  \(x \in R \setminus \mathfrak{m}\)
                  یکه‌ای از
                  \(R\)
                  باشد. در این صورت
                  \(R\)
                  موضعی است و
                  \(\mathfrak{m}\)
                  ایده‌آل ماکسیمال آن است.
                  \label{part1}
            \item \(R\)
                  یک حلقه و
                  \(\mathfrak{m}\)
                  ایده‌آل ماکسیمال
                  \(R\)
                  باشد، به طوری که به ازای هر عنصر
                  \(m \in \mathfrak{m}\)،
                  \(1+m\)
                  یکه‌ای در
                  \(R\)
                  باشد. در این صورت
                  \(R\)
                  موضعی است.
                  \label{part2}
        \end{enumerate}
    \end{corollary}


    \textbf{اثبات قسمت ۲.}
    اگر
    \(a \in R \setminus \mathfrak{m}\)
    در این صورت
    \(\mathfrak{m} + (a) = R\)
    پس
    \[
        \exists t \in \mathfrak{m}, \exists r \in R; \quad t + ra = 1 \implies ra = 1 - t = 1 + (-t) \implies ra \text{ یکه } \implies a \text{ یکه }
    \]
    بنابر
    قسمت
    1
    حلقهُ
    \(R\)
    موضعی است.
    \hfill \(\qed\)
\end{frame}

\begin{frame}
    \begin{definition}
        \[
            \mathcal{N}(R) = \bigcup_{M \in Max(R)} M
        \]
    \end{definition}


    \begin{remark}
        اجتماع 3 ایده‌آل ممکن است یک ایده‌آل نباشد. مثل \( \mathbb{Z}_2 \times \mathbb{Z}_2 \).
    \end{remark}

\end{frame}

\begin{frame}
    \begin{theorem}
        موارد زیر هم‌ارزند:
        \begin{enumerate}
            \item \(\mathcal{N}(R) = \mathcal{J}(R)\)
            \item \(\mathcal{N}(R)\) یک ایده‌آل از \(R\) است.
            \item \(R\) موضعی است.
        \end{enumerate}
    \end{theorem}

    \textbf{اثبات.}
    
        \begin{itemize}
            \item (1) $\Leftarrow$ (2):  واضح\(\checkmark\)
            \item (3) $\Leftarrow$ (1): واضح\( \checkmark\)
            \item (2) $\Leftarrow$ (3): اگر \( M \subsetneq \mathcal{N}(R) \) و \( M \) ماکسیمال باشد، آنگاه اگر \( R \) موضعی نباشد،
                  \( \mathcal{J}(R) = \mathcal{N}(R) = R\).
                  از طرفی
                  \(R = \mathcal{J}(R) \subseteq M \subseteq R\)،
                  بنابر این
                  \(M = R\)،
                  که در تناقض با ماکسیمال بودن
                  \(M\)
                  است.
        \end{itemize}

        \hfill \(\qed\)
\end{frame}