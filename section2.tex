\section{\lr{02 Zorn's Lemma}}

\begin{theorem}
    فرض کنید $R$ حلقه‌ای جابجایی و یکدار باشد و $I$ ایده‌آلی سره از $R$ باشد. در این صورت \textbf{ایده‌آل اول} $P$ وجود دارد که شامل $I$ است و به‌علاوه هیچ ایده‌آل اولی بین $I$ و $P$ وجود ندارد.
\end{theorem}
\begin{proof}
    فرض کنید
    \[
        \mathcal{A} = \left\{ Q \;\middle|\; I \subseteq Q,\; Q \in \mathrm{Spec}(R) \right\}.
    \]
    توجه کنید که $\mathcal{A} \ne \emptyset$.

    رابطه‌ی $\leq$ را عکس شمول در نظر می‌گیریم یعنی $B \leq C \Leftrightarrow C \subseteq B$، که رابطه‌ای جزئاً مرتب بر $\mathcal{A}$ است.

    فرض کنید $\mathcal{T}$ یک زنجیر در $\mathcal{A}$ باشد. ثابت می‌کنیم که $\mathcal{T}$ در $\mathcal{A}$ کران بالا دارد. بگذارید
    \[
        \mathcal{L} = \bigcap_{B \in \mathcal{T}} B.
    \]

    در این صورت $\mathcal{L} $ کران بالای $\mathcal{T}$ است. نشان می‌دهیم که $\mathcal{L}  \in \mathcal{A}$ است.

    اگر $ab \in \mathcal{L} $ و $a \notin \mathcal{L} $، آنگاه:
    \[
        \exists B \in \mathcal{T}; \quad  a \notin B.
    \]
    اما در این صورت
    \[
        \forall C \in \mathcal{T}; \quad b \in C.
    \]
    در نتیجه \(b \in \mathcal{L}\) و لذا \(\mathcal{L} \in \mathcal{A}\).
    پس شرایط لم زورن برقرار است، لذا \(\mathcal{A}\) دارای عنصری ماکسیمال مانند \(P\) است.
    مفهوم ماکسیمال بودن در این شرایط...

\end{proof}



\begin{definition}
    ایده‌آل $P$ در قضیه‌ی قبل را یک \textbf{ایده‌آل اول مینیمال $I$} نامیده و با $\operatorname{Min}(I)$ نمایش می‌دهند.
\end{definition}
\begin{assignment}
    فرض کنید $R$ حلقه‌ای جابجایی و یکدار باشد و $I$ ایده‌آلی از $R$ باشد    بعلاوه فرض کنید  $I \subseteq P$   موجود است که $I \subseteq P$. ثابت کنید ایده‌آل اول مینیمالی از \(I\) مانند \(Q\) وجود دارد که  $I \subseteq Q \subseteq P$.

\end{assignment}

\begin{tcolorbox}[colback=blue!5!white, colframe=blue!75!black, title={یادآوری}]

    \begin{itemize}
        \item اگر $\varphi: R \to S$ یک هم‌ریختی حلقه‌ای باشد و $I$ ایده‌آلی از $S$، آنگاه $\varphi^{-1}(I)$ ایده‌آلی از $R$ است.
        \item اگر $J$ ایده‌آلی از $R$ باشد،لزوما \(\varphi(J)\) ایده‌آل  \(S\) نیست.
    \end{itemize}



\end{tcolorbox}

\begin{tcolorbox}[colback=blue!5!white, colframe=blue!75!black, title={اصطلاحات رایج}]
    $\varphi^{-1}(I) = I^c$ را \textbf{انقباض  \LTRfootnote{contraction} $I$} و $\varphi(J)S = J^e$ ({ ایده‌آل تولید شده توسط \(\varphi(J)\)}) را \textbf{ توسیع  \LTRfootnote{extension} $J$} گویند.
\end{tcolorbox}

\begin{theorem}
    اگر $P \in \mathrm{Spec}(S)$ آنگاه $P^c \in \mathrm{Spec}(R)$.
\end{theorem}

\begin{proof}
    اگر $ab \in P^c$ ولی $a \notin P^c$، آنگاه $b \in P^c$.
\end{proof}


\begin{theorem}[\lr{(Prime Avoidance Theorem)}]
    فرض کنید $P_1, \dots, P_n$ (\(n \geq 2\)) ایده‌آل‌هایی از حلقه جابجایی و یک‌دار $R$ باشند، به‌طوری‌که حداقل $n-2$ تا از آن‌ها اول باشند. فرض کنید $I \subseteq R$ زیرمجموعه‌ای باشد که تحت جمع و ضرب بسته است و $I \subseteq \bigcup_{i=1}^n P_i$. در این صورت، $\exists\, 1 \le t \le n$ چنان‌که $I \subseteq P_t$.

\end{theorem}


\begin{proof}

    \textbf{اثبات با استقرا بر $n$ حکم ثابت می‌شود.}

    \textbf{پایه:} اگر $n = 2$، فرض کنیم $I \subseteq P_1 \cup P_2$ ولی $I \not\subseteq P_1$ و $I \not\subseteq P_2$، آنگاه $\exists\, x \in I \setminus P_1$ و $\exists\, y \in I \setminus P_2$. چون $x+y \in I$ ولی $x+y \notin P_1 \cup P_2$، تناقض.

    \textbf{گام استقرا:} فرض کنید حکم برای $n = k$ برقرار است.
    حال $n = k+1$ را در نظر بگیرید.

    فرض کنید $I \subseteq \bigcup_{i=1}^{k+1} P_i$
    و حداقل
    \(k-1\)
    تا از
    \(P_i\)
    ها اولند.
    با برهان خلف فرض کنید حکم صحیح نباشد، پس
    \[
        \forall 1 \le j \le k+1; \quad I \not\subseteq \bigcup_{\substack{i=1\\i \ne j}}^{k+1} P_i
    \]
    برای هر $j$، عضوی $a_j \in I$ موجود است که $a_j \notin \bigcup_{\substack{i=1\\i \ne j}}^{k+1} P_i$.
    از آنجا که
    $I \subseteq \bigcup_{i=1}^{k+1} P_i$
    پس
    به ازای هر
    \(1 \le j \le k+1\)
    بایستی
    \(a_j \in P_j\). ضمناً فرض کنید که
    \(P_{k+1}\)
    اول باشد.
    حال عنصر
    \[
        b = a_1 a_2 \cdots a_k + a_{k+1}
    \]
    را در نظر بگیرید. نشان می‌دهیم
    $b \notin \bigcup_{i=1}^{k+1} P_i$.

    \begin{itemize}
        \item
              از آنجا که
              \(a_{k+1} \in P_{k+1}\)، اگر
              \(b\in P_{k+1}\)
              در این صورت
              \(a_1a_2\cdots a_k = b - a_{k+1} \in P_{k+1}\).
              و از آنجا که
              \(P_{k+1}\)
              اول است، حداقل یک
              \(1 \le j \le k\)
              وجود دارد که
              \(a_j \in P_{k+1}\)، تناقض.
        \item به همین ترتیب، به ازای \(1 \le j \le k\) از آنجا که
              \(a_{j} \in P_{j}\)
              و
              \( P_{j}\)
              یک ایده‌آل است
              \(a_1 a_2 \cdots a_k \in P_{j}\)
              ، اگر
              \(b\in P_{k+1}\)
              در این صورت
              \(a_{k+1}  = b - a_1a_2\cdots a_k  \in P_{k+1}\)، تناقض.
    \end{itemize}

    پس داریم $b \in I$ ولی $b \notin \bigcup_{i=1}^{k+1} P_i$، که این با فرض اولیه $I \subseteq \bigcup_{i=1}^{k+1} P_i$ در تناقض است.

    \textbf{در نتیجه:} باید $I \subseteq P_t$ برای یک $1 \le t \le k+1$ باشد.

\end{proof}

\begin{definition}[ رادیکال پوچ $R$]
    اگه $R$
    یک حلقه باشد، در این صورت
    رادیکال‌ پوچ
    حلقه
    $R$
    یا
    نیل‌رادیکال $R$
    به صورت زیر تعریف می‌شود:
    $$\mathrm{nil}(R) = \bigcap_{P \in \mathrm{Spec}(R)} P$$

\end{definition}

\begin{definition}[  جیکبسون رادیکال  $R$]
    اگه $R$
    یک حلقه باشد، در این صورت
    جیکبسون رادیکال $R$
    به صورت زیر تعریف می‌شود:
    $$\mathcal{J}(R) = \bigcap_{M \in \mathrm{Max}(R)} M$$

\end{definition}

\begin{remark}
    اگر $R$ حلقه‌ای جابجایی و یکدار باشد،
    از آنجا که $\mathrm{Max}(R) \subseteq \mathrm{Spec}(R)$ داریم:
    \[
        \mathrm{nil}(R) \subseteq \mathcal{J}(R)
    \]
\end{remark}

\begin{theorem}
    اگر $R$ حلقه‌ای جابجایی و یکدار باشد، در این صورت
    $$\mathrm{nil}(R) = \{ x \in R \mid \exists n \in \mathbb{N} ; \; x^n = 0 \} = R \text{ مجموعهٔ عناصر پوچ‌توان  } $$

\end{theorem}

\begin{proof}
    فرض کنید $x^n = 0 $، در این صورت
    $x^n = 0 \in P$
    برای هر $P \in \mathrm{Spec}(R)$.
    از آنجا که
    \(P\)
    اول است،
    آنگاه $x \in P$ برای هر $P \in \mathrm{Spec}(R)$، پس:
    \[
        x \in \bigcap_{P \in \mathrm{Spec}(R)} P = \mathrm{nil}(R)
    \]
    بنابراین:
    \[
        R \text{مجموعهٔ عناصر پوچ‌توان } \subseteq \mathrm{nil}(R)
    \]
    حال
    اگر $x \in \mathrm{nil}(R)$، آنگاه $x \in P$ برای هر $P \in \mathrm{Spec}(R)$.
    فرض کنیم $\forall n,\ x^n \ne 0$.
    بگذارید:
    \[
        S = \{x^n \mid n \in \mathbb{N} \cup \{0\} \}
    \]
    $S$
    بستهُ ضربی است و
    $0 \notin S$.
    پس
    \(\{0\} \cap S = \emptyset\). بنابراین ایده‌آل اول
    $Q$
    موجود است که
    \(\{0\} \subseteq Q\)
    و
    $S \cap Q = \emptyset$
    پس
    \[
        x \notin Q \implies x \notin \mathrm{nil}(R)
    \]
    که تناقض است،ٔ پس وجود دارد \(n\) که \(x^n = 0\).

\end{proof}

\begin{assignment}
    فرض کنید
    \(R\)
    حلقه‌ای جابجایی و یکدار است، در این صورت:
    $$x \in \mathcal{J}(R) \iff \forall y \in R, \text{ یکه است } R \text{ در } \ 1 - xy  $$
\end{assignment}

\begin{assignment}
    فرض کنید
    \(R\)
    حلقه‌ای جابجایی و یکدار است، در این صورت
    \(\frac{R}{\mathrm{nil}(R)}\)
    هیچ عنصر پوچ‌توان ناصفری ندارد.
\end{assignment}