\section{ مرور جبر کارشناسی}


\begin{definition}[  نیم‌گروه]
    فرض کنید \( S \) مجموعه‌ای ناتهی باشد و یک عمل دوتایی \(\cdot\) روی \( S \) را در نظر بگیرید.
    اگر این عمل شرکت‌پذیر باشد یعنی:
    \[
        \forall a, b, c \in S \quad (a \cdot b) \cdot c = a \cdot (b \cdot c)
    \]
    آنگاه \( S \) را همراه با عمل \(\cdot\) یک \textbf{نیم‌گروه \LTRfootnote{Semigroup}} می‌نامند.

\end{definition}



\begin{definition}[گروه]
    فرض کنید مجموعه ناتهی \( G \) همراه با عمل دوتایی \(\cdot\) در شرایط زیر صدق کند:
    \[\forall a, b, c \in G \quad a \cdot (b \cdot c) = (a \cdot b) \cdot c\]
    \[\exists e \in G \; \forall a \in G \quad a \cdot e = e \cdot a = a \quad \text{(وجود عنصر همانی)}\]
    \[\forall a \in G \; \exists b \in G \quad a \cdot b = b \cdot a = e \quad \text{(وجود عنصر وارون)}\]
    در این صورت \( (G, \cdot) \) را یک \textbf{گروه} می‌نامند.

\end{definition}
\begin{remark}
    غالبا  بجای \( a \cdot b \) می‌نویسیم \( ab \).

\end{remark}
\begin{definition}[  گروه آبلی]
    اگر به ازای هر \( a, b \in G \) داشته باشیم:
    \[
        a \cdot b = b \cdot a
    \]
    گروه را \textbf{آبلی یا جابجایی} می‌نامند.
\end{definition}


\begin{definition}[  زیرگروه]
    اگر \( (G, \cdot) \) یک گروه باشد و \( H \subseteq G \) زیرمجموعه‌ای ناتهی از \( G \) باشد که خودش نیز تحت عمل \(\cdot\) یک گروه باشد، آنگاه \( H \) را یک \textbf{زیرگروه} \( G \) می‌نامند و می‌نویسیم:
    \[
        H \leq G
    \]

\end{definition}

\begin{theorem}[  محک فشرده]

    زیرمجموعه ناتهی \( H \subseteq G \) زیرگروه \( G \) است اگر و تنها اگر:
    \[
        \forall a, b \in H \quad a \cdot b^{-1} \in H
    \]
\end{theorem}


\begin{example}
    \( (\mathbb{Z}, +) \)، گروه اعداد صحیح با عمل جمع را در نظر بگیرید.
    زیرمجموعه‌ی \( 2\mathbb{Z} = \{ 2k \mid k \in \mathbb{Z} \} \) یعنی اعداد زوج، با همان عمل جمع، یک زیرگروه از \( \mathbb{Z} \) است، بنابراین:
    \[
        (2\mathbb{Z}, +) \leq (\mathbb{Z}, +)
    \]
\end{example}

\begin{proof}
    برای اثبات اینکه \( (2\mathbb{Z}, +) \) زیرگروه است، از محک فشرده استفاده می‌کنیم:
    باید نشان دهیم اگر \( a, b \in 2\mathbb{Z} \) باشند، آنگاه \( a - b \in 2\mathbb{Z} \).

    ازآنجا که \( a = 2m \) و \( b = 2n \)  به ازای \( m, n \in \mathbb{Z} \)، آنگاه:
    \[
        a - b = 2m - 2n = 2(m - n)
    \]
    که چون \( m - n \in \mathbb{Z} \)، پس \( a - b \in 2\mathbb{Z} \).

    بنابراین، \( a + (-b) \in 2\mathbb{Z} \) و با استفاده از محک فشرده نتیجه می‌گیریم که \( (2\mathbb{Z}, +) \) زیرگروه \( (\mathbb{Z}, +) \) است.
\end{proof}

\begin{example}
    \( (\mathbb{Z}, +) \)، گروه اعداد صحیح با عمل جمع را در نظر بگیرید.
    برای هر \( n \in \mathbb{N} \)، مجموعه \( n\mathbb{Z} = \{ nk \mid k \in \mathbb{Z} \} \) با همان عمل جمع، زیرگروه \( \mathbb{Z} \) است:
    \[
        (n\mathbb{Z}, +) \leq (\mathbb{Z}, +)
    \]
\end{example}

\begin{proof}
    برای \( a, b \in n\mathbb{Z} \)، داریم \( a = nk \)، \( b = nl \)  به ازای \( k, l \in \mathbb{Z} \)، پس:
    \[
        a - b = nk - nl = n(k - l) \in n\mathbb{Z}
    \]
    بنابراین، \( n\mathbb{Z} \) تحت تفاضل بسته است و زیرگروه \( \mathbb{Z} \) می‌باشد.
\end{proof}


\begin{example}
    \( (\mathbb{N}, \times) \) گروه نیست.

    زیرا برای مثال عدد \( 2 \in \mathbb{N} \) هیچ عضو معکوسی نسبت به ضرب در \( \mathbb{N} \) ندارد.
    یعنی عدد طبیعی‌ای وجود ندارد که \( 2 \times x = 1 \) را برآورده کند.

    بنابراین، شرط وجود عنصر معکوس برای همه اعضا برقرار نیست و \( (\mathbb{N}, \times) \) گروه نیست.
\end{example}
\begin{definition}
    برای هر میدان \( F \)، مجموعه‌ی \( F^* \) به صورت زیر تعریف می‌شود:
    \[
        F^* = F \setminus \{0\}
    \]
    یعنی مجموعه‌ی تمام اعضای ناصفر \( F \). این مجموعه تحت عمل ضرب، یک گروه تشکیل می‌دهد.

    به طور خاص:
    \begin{itemize}
        \item \( \mathbb{Q}^* = \mathbb{Q} \setminus \{0\} \): اعداد گویا ناصفر
        \item \( \mathbb{R}^* = \mathbb{R} \setminus \{0\} \): اعداد حقیقی ناصفر
        \item \( \mathbb{C}^* = \mathbb{C} \setminus \{0\} \): اعداد مختلط ناصفر
    \end{itemize}
    هر یک از این مجموعه‌ها با عمل ضرب، یک گروه ضربی می‌سازند.
\end{definition}


\begin{definition}
    اگر \( F \) یک میدان \LTRfootnote{Field} باشد، آنگاه \( M_n(F) \) مجموعه‌ی تمام ماتریس‌های \( n \times n \) با درایه‌هایی از \( F \) است:
    \[
        M_n(F) = \{ A = (a_{ij}) \mid 1 \leq i, j \leq n,\; a_{ij} \in F \}
    \]
    روی \( M_n(F) \) می‌توان اعمال مختلفی تعریف کرد، مانند جمع ماتریسی و ضرب ماتریسی.

    معمولاً \( (M_n(F), +) \) یک گروه آبلی است (نسبت به جمع ماتریسی) و \( (M_n(F), \cdot) \) یک نیم‌گروه است (نسبت به ضرب ماتریسی، ولی بسته به \( F \) و \( n \) ممکن است گروه نباشد زیرا ماتریس‌های ناتبدیل وارون ندارند).
\end{definition}

\begin{example}
    در ادامه چند مثال از زیرگروه‌ها آورده شده است:
    \begin{align*}
        (\mathbb{Z}, +)       & \leq (\mathbb{Q}, +)                                        \\
        (\mathbb{Q}, +)       & \leq (\mathbb{R}, +)                                        \\
        (\mathbb{Q}^*, \cdot) & \leq (\mathbb{R}^*, \cdot)                                  \\
        (M_n(\mathbb{Q}), +)  & \leq (M_n(\mathbb{R}), +)                                   \\
        (\mathbb{C}, +)       & \quad \text{(گروه جمعی اعداد مختلط)}                        \\
        (\mathbb{C}^*, \cdot) & \quad \text{(گروه ضربی اعداد مختلط ناصفر)}                  \\
        (M_n(\mathbb{C}), +)  & \quad \text{(گروه ماتریس‌های \( n \times n \) مختلط با جمع)}
    \end{align*}
\end{example}
\begin{definition}
    برای عدد صحیح \( n \geq 2 \)، مجموعه‌ی باقیمانده‌های صحیح پیمانه‌ای را به صورت زیر تعریف می‌کنیم:
    \[
        \mathbb{Z}_n = \{ \overline{0}, \overline{1}, \overline{2}, \ldots, \overline{n-1} \}
    \]
    که در آن \( \overline{a} \) نمایانگر کلاس پیمانه‌ای \( a \) نسبت به \( n \) است، یعنی:
    \[
        \overline{a} = \{ b \in \mathbb{Z} \mid b \equiv a \pmod{n} \}
    \]

    عمل جمع روی \( \mathbb{Z}_n \) به صورت زیر تعریف می‌شود:
    \[
        \overline{i} + \overline{j} := \overline{i + j}
    \]

    به این ترتیب، \( (\mathbb{Z}_n, +) \) یک گروه آبلی متناهی است.
\end{definition}

\begin{proof}[اثبات خوش‌تعریف بودن عمل جمع]
    فرض کنید \( \overline{i} = \overline{i'} \) و \( \overline{j} = \overline{j'} \).
    آنگاه به ازای \( k, \ell \in \mathbb{Z} \) داریم:
    \[
        i = i' + kn \quad \text{و} \quad j = j' + \ell n
    \]
    بنابراین:
    \[
        i + j = i' + j' + (k + \ell)n \Rightarrow i + j \equiv i' + j' \pmod{n}
    \]
    یعنی:
    \[
        \overline{i + j} = \overline{i' + j'}
    \]
    پس:
    \[
        \overline{i} + \overline{j} = \overline{i + j} = \overline{i' + j'} = \overline{i'} + \overline{j'}
    \]
    بنابراین، عمل جمع خوش‌تعریف است.
\end{proof}
\begin{definition}[زیرگروه نرمال]
    زیرگروه \( N \) از گروه \( G \) را \textbf{نرمال} گویند اگر:
    \[
        \forall g \in G \quad gNg^{-1} \subseteq N
    \]
    که معادل است با:
    \[
        gNg^{-1} = \{ gng^{-1} \mid n \in N \} \subseteq N
    \]
    در این صورت می‌نویسیم \( N \trianglelefteq G \).

\end{definition}
\begin{example}
    تمام زیرگروه‌های آبلی نرمال هستند.
\end{example}
\begin{example}
    گروه دیهدرال \( D_n = \langle a, b \mid a^n = b^2 = e,\; bab^{-1} = a^{-1} \rangle \) را در نظر بگیرید.

    اعضای این گروه عبارت‌اند از:
    \[
        \{ e, a, a^2, \ldots, a^{n-1},\; b, ab, a^2b, \ldots, a^{n-1}b \}
    \]

    زیرگروه مولد \( a \) که برابر است با:
    \[
        \langle a \rangle = \{ e, a, a^2, \ldots, a^{n-1} \}
    \]
    یک زیرگروه نرمال در \( D_n \) است، یعنی:
    \[
        \langle a \rangle \trianglelefteq D_n
    \]
\end{example}

\begin{proof}
    کافی است نشان دهیم که \( b \langle a \rangle b^{-1} \subseteq \langle a \rangle \).
    برای هر \( a^k \in \langle a \rangle \) داریم:
    \[
        b a^k b^{-1} = (bab^{-1})^k = (a^{-1})^k = a^{-k} \in \langle a \rangle
    \]
    پس \( b \langle a \rangle b^{-1} \subseteq \langle a \rangle \) و از آنجا که \( D_n \) توسط \( a \) و \( b \) تولید شده، برای همه عناصر \( g \in D_n \) داریم \( g \langle a \rangle g^{-1} \subseteq \langle a \rangle \).
    بنابراین، \( \langle a \rangle \) زیرگروه نرمال در \( D_n \) است.
\end{proof}

\begin{theorem}
    اگر \( H \leq G \) و \( [G : H] = 2 \) آنگاه \( H \trianglelefteq G \).
\end{theorem}

\begin{proof}
    چون اندیس برابر ۲ است، \( G \) دقیقاً دو هم‌دسته‌ی راست دارد: \( H \) و \( G \setminus H \).
    حال اگر \( g \in H \) باشد، آنگاه \( gH = H = Hg \).
    و اگر \( g \in G \setminus H \) باشد، آنگاه \( gH = G \setminus H = Hg \).
    پس برای همه \( g \in G \) داریم \( gH = Hg \)، یعنی همه‌ی هم‌دسته‌های چپ و راست با هم برابرند و \( H \) نرمال است.
\end{proof}

\begin{definition}[هم‌دستهٔ راست]
    اگر \( H \leq G \) و \( g \in G \)، هم‌دستهٔ راست \( H \) نسبت به \( g \) به صورت زیر تعریف می‌شود:
    \[
        Hg = \{ hg \mid h \in H \}
    \]
\end{definition}




\begin{example}
    مثال. در گروه $D_{2n}$ با فرض $H = \langle a \rangle$ دو هم‌دستهٔ متمایز داریم:
    \[
        H \quad \text{و} \quad Hb
    \]
    که در واقع دو مجموعهٔ متمایزند:
    \[
        \{e, a, \ldots, a^{n-1} \} \neq \{e, a, \ldots, a^{n-1} \}b
    \]
\end{example}



\begin{proof}
    هر عضو از \(D_{2n}\) یا به صورت دوران \(a^k\) است یا به صورت بازتاب \(a^k b\) برای \(0\le k\le n-1\)، بنابراین:
    \[
        D_{2n}=H\;\dot\cup\;Hb.
    \]
    در نتیجه، \([D_{2n}:H]=2\) و بنابراین حداکثر دو هم‌دستهٔ راست داریم: \(H\) و \(Hb\).

    برای بررسی اینکه این دو هم‌دسته متفاوتند، فرض کنید \(Hb = H\). آنگاه \(b \in H\) و در نتیجه \(b = a^m\) برای عددی صحیح \(m\) خواهد بود. اما در این صورت:
    \[
        bab^{-1}=a^m a a^{-m}=a
    \]
    که این با رابطهٔ تعریف‌شده‌ی گروه یعنی \(bab^{-1} = a^{-1}\) در تضاد است، مگر اینکه \(n=2\) باشد. بنابراین \(b\notin H\) و \(Hb\ne H\).

    (برای مورد خاص \(n=2\) داریم \(D_4=\{e,a,b,ab\}\) با روابط \(a^2=b^2=e\) و \(ab=ba\)؛ در این حالت
    \(H=\{e,a\}\) و \(Hb=\{b,ab\}\) هستند که به‌وضوح متمایزند.)
\end{proof}



\begin{lemma}
    اگر $N \triangleleft G$ باشد، آنگاه برای هر $g \in G$ داریم:
    \[
        gN = Ng
    \]
\end{lemma}


\begin{proof}
    از آنجا که \(N \triangleleft G\) است، برای هر \(g \in G\) داریم \(gNg^{-1} = N\).

    ابتدا نشان می‌دهیم \(gN \subseteq Ng\): برای هر \(n \in N\)، داریم
    \[
        gn = gn1 = gn(g^{-1}g) =(gng^{-1})\,g \in Ng
    \]
    چون \(gng^{-1} \in N\) است.

    برعکس، نشان می‌دهیم \(Ng \subseteq gN\): برای هر \(n \in N\)، داریم
    \[
        ng = g\,(g^{-1}ng) \in gN
    \]
    چون \(g^{-1}ng \in N\) است.

    در نتیجه، برای هر \(g \in G\) داریم \(gN = Ng\).
\end{proof}

\begin{remark}
    برای هر $a,b \in G$ داریم:
    \[
        Ha = Hb \iff ab^{-1} \in H
    \]
\end{remark}

\begin{proof}
    (\(\Rightarrow\)) اگر \(Ha = Hb\)، آنگاه \(a \in Hb\)، بنابراین \(a = hb\) برای عضوی \(h \in H\). پس:
    \[
        ab^{-1} = h \in H.
    \]

    (\(\Leftarrow\)) اگر \(ab^{-1} \in H\)، بنویسید \(ab^{-1} = h\) با \(h \in H\). آنگاه \(a = hb\)، و در نتیجه:
    \[
        Ha = Hhb = Hb
    \]
    چون برای هر \(h \in H\) داریم \(Hh = H\).
\end{proof}


\begin{definition}[گروه خارج‌قسمتی]
    فرض کنید $N \triangleleft G$. در این صورت، مجموعهٔ تمام همدسته‌های راست $N$ در $G$ را با نماد $G / N$ نمایش می‌دهیم و روی مجموعهٔ همدسته‌های $N$ در $G$ عمل زیر را تعریف می‌کنیم:
    \[
        Na \cdot Nb := Nab
    \]

    عنصر همانی همدستهً $N$ برابر است با:
    \[
        N = Ne = Nn \quad (n \in N)
    \]
\end{definition}

\begin{proof}[خوش‌تعریفی عمل ضرب]
    فرض کنید \(Na = Na'\) و \(Nb = Nb'\). آنگاه عناصری \(n, m \in N\) وجود دارند به‌طوری‌که
    \(a' = na\) و \(b' = mb\). پس:
    \[
        Na'b' = N(na)(mb) = N\,n\,(a m a^{-1})\,ab.
    \]
    از آنجا که \(N \triangleleft G\)، داریم \(a m a^{-1} \in N\)، و در نتیجه
    \(N\,n\,(a m a^{-1}) = N\). بنابراین:
    \[
        Na'b' = N ab.
    \]
    پس عمل \(Na \cdot Nb := Nab\) مستقل از نماینده‌های انتخاب‌شده است،
    و لذا خوش‌تعریف است.
\end{proof}





\begin{definition}[همریختی ]
    فرض کنید $G$ و $H$ دو گروه باشند و تابع $f \colon G \to H$ را در نظر بگیرید.

    می‌گوییم $f$ \textbf{همریختی} است اگر:
    \[
        \forall\, a,b \in G \quad f(ab) = f(a)f(b)
    \]

    در این صورت، مفاهیم زیر تعریف می‌شوند:

    \begin{itemize}
        \item \textbf{هستهٔ همریختی}:
              \[
                  \ker f = \left\{ a \in G \mid f(a) = e_H \right\}
              \]

        \item \textbf{برد همریختی}:
              \[
                  \operatorname{Im} f = \left\{ f(a) \mid a \in G \right\}
              \]

        \item اگر $f$ یک‌به‌یک باشد، آن‌را \textbf{تک‌ریختی} \LTRfootnote{monomorphism} می‌نامند.

        \item اگر $f$ پوشا باشد، آن‌را \textbf{برو‌ریختی} \LTRfootnote{epimorphism} می‌نامند.

        \item اگر $f$ هم‌تک‌ریختی و هم برو‌ریختی باشد، آن‌را \textbf{ایزومورفیسم} یا \textbf{یکریختی} \LTRfootnote{isomorphism} می‌نامند. در این صورت می‌نویسیم $G \cong H$.

    \end{itemize}
\end{definition}



\begin{example}
    اگر $G$ گروهی دوری از مرتبهٔ $n$ باشد یعنی $G = \langle a \mid a^n = e \rangle$، آنگاه $G \cong \mathbb{Z}_n$ ؛ یعنی گروه $G$ با گروه $\mathbb{Z}_n$ ایزومورف است.
\end{example}

\begin{example}
    اگر $G$ گروه دوران‌های حول مبدا صفحه در جهت مثلثاتی به اندازهٔ $\frac{2\pi k}{n}$ باشد که $k \in \mathbb{Z}$، این گروه $G$ با $\mathbb{Z}_n$ ایزومورف است.
\end{example}

\begin{example}
    هر گروه دوری نامتناهی با $\mathbb{Z}$ ایزومورف است.
\end{example}

\begin{definition}
    اگر
    $G$
    گروهی متناهی باشد،
    $|G|$
    را تعداد اعضای گروه
    $G$
    تعریف می‌کنیم و آن را
    \textbf{مرتبهٔ گروه} می‌نامند.
\end{definition}



\begin{example}
    هر گروه از مرتبهٔ
    $p$
    با گروه دوری $(\mathbb{Z}_p,+)$ ایزومورف است.
\end{example}




\begin{theorem}[{قضایای یکریختی}]
    \begin{description}
        \item[قضیهٔ اول یکریختی:]
              فرض کنید $f \colon G \to H$ یک همریختی گروهی باشد. در این صورت داریم:
              \[
                  \frac{G}{\ker f} \cong \operatorname{Im} f
              \]

        \item[قضیهٔ دوم یکریختی:]
              اگر $H \leq G$ و $N \trianglelefteq G$، آنگاه:
              \[
                  \frac{HN}{N} \cong \frac{H}{H \cap N}
              \]

        \item[قضیهٔ سوم یکریختی:]
              اگر
              $H, K \trianglelefteq G$
              و
              $H \leq K $، آنگاه:
              \[
                  \frac{G/H}{K/H} \cong G/K
              \]
    \end{description}
\end{theorem}


\begin{example}
    به ازای هر
    \(n \in \mathbb{Z}\)
    داریم،
    \[
        \frac{\mathbb{Z}}{n\mathbb{Z}} \cong \mathbb{Z}_n
    \]


\end{example}


\begin{proof}
    تابع $f \colon \mathbb{Z} \to \mathbb{Z}_n$ را به‌صورت $f(k) = \overline{k}$ تعریف می‌کنیم که همریختی پوشاست، زیرا:
    \[
        f(i + j) = \overline{i + j} = \overline{i} + \overline{j} = f(i) + f(j)
    \]

    در نتیجه $\ker f = n\mathbb{Z}$، و چون $f$ پوشاست، طبق قضیهٔ اول یکریختی داریم:
    \[
        \frac{\mathbb{Z}}{n\mathbb{Z}} \cong \mathbb{Z}_n
    \]

    کلاس‌های هم‌ریختی در $\mathbb{Z}/n\mathbb{Z}$ عبارتند از:
    \[
        \left\{ n\mathbb{Z},\ n\mathbb{Z}+1,\ n\mathbb{Z}+2,\ \ldots,\ n\mathbb{Z}+(n-1) \right\}
    \]
\end{proof}


\begin{remark}[یادآوری مفهوم گروه‌های خارج‌قسمتی]
    اعضا: همدسته ها

    در گروه $(\mathbb{Z}, +)$ همدسته های زیرگروه $n\mathbb{Z}$ به شکل زیر هستند:
    \[
        n\mathbb{Z} + m = \left\{ nk + m \mid k \in \mathbb{Z} \right\} \quad (m \in \mathbb{Z})
    \]

    برای نمونه:
    \begin{align*}
        n\mathbb{Z} + 0     & = n\mathbb{Z} = \left\{ \ldots, -3n, -2n, -n, 0, n, 2n, 3n, \ldots \right\}                                   \\
        n\mathbb{Z} + 1     & = \left\{ \ldots, -3n+1, -2n+1, -n+1, 1, n+1, 2n+1, 3n+1, \ldots \right\}                                     \\
                            & \vdots                                                                                                        \\
        n\mathbb{Z} + (n-1) & = \ldots                                                                                                      \\
        n\mathbb{Z} + n     & = \left\{ nk + n \mid k \in \mathbb{Z} \right\} = \left\{ n(k+1) \mid k \in \mathbb{Z} \right\} = n\mathbb{Z}
    \end{align*}
\end{remark}


\begin{definition}[گروه خودریختی‌ها]
    گروه خودریختی‌های $G$ برابر است با:
    \[
        \mathrm{Aut}(G) = \left\{ \varphi \colon G \to G \mid  \text{ همریختی یک به یک و پوشا} \varphi \right\}
    \]
\end{definition}

\begin{definition}[خودریختی‌های درونی]
    گروه خودریختی‌های داخلی $G$ برابر است با:
    \[
        \mathrm{Inn}(G) = \left\{ I_g \colon G \to G \mid g \in G \right\}
    \]
    که در آن نگاشت $I_g$ به‌صورت زیر تعریف می‌شود:
    \[
        I_g(x) = gxg^{-1}
    \]
\end{definition}

% \begin{theorem}
%     \[
%         \frac{G}{Z(G)} \cong \mathrm{Inn}(G)
%     \]
% \end{theorem}

\begin{definition}[مرکز گروه]
    مرکز گروه $G$ برابر است با:
    \[
        Z(G) = \left\{ g \in G \mid gx = xg \quad (\forall x \in G) \right\}
    \]
\end{definition}

\begin{remark}
    $G$ \textbf{آبِلی} است اگر و تنها اگر $G = Z(G)$.
\end{remark}


\begin{definition}[حلقه]
    فرض کنید $R$ مجموعه‌ای ناتهی همراه با دو عمل دوتایی $+$ و $\cdot$ باشد، به‌طوری‌که $(R, +)$ یک گروه آبلی باشد و خواص زیر برقرارند:

    برای همهٔ $a,b,c \in R$:
    \begin{align}
        a(bc)    & = (ab)c \tag{1}   \\
        a(b + c) & = ab + ac \tag{2} \\
        (b + c)a & = ba + ca \tag{3}
    \end{align}

    در این صورت $(R, +, \cdot)$ را یک \textbf{حلقه} می‌نامند.
    اگر عمل ضرب دارای عضو همانی باشد، یعنی عنصری مانند $1_R$ وجود داشته باشد چنان‌که:
    \[
        \forall a \in R \quad a 1_R = 1_R a = a
    \]
    در این صورت، $R$ را \textbf{حلقه یکدار} می‌نامند.
    اگر عمل $\cdot$ جابجایی باشد، حلقه را \textbf{حلقه جابجایی} می‌نامند.
    زیرمجموعهٔ ناتهی $S$ از حلقهٔ $(R, +, \cdot)$ را \textbf{زیرحلقه} می‌نامند اگر $S$ نسبت به دو عمل $+$ و $\cdot$ یک حلقه باشد.

\end{definition}



\begin{remark}
    در این درس همواره فرض براین است که حلقه جابجایی و یکدار باشد، مگر آن‌که خلاف آن بیان شود.
\end{remark}

\begin{example}
    \[
        (\mathbb{Z}, +, \cdot),\quad (\mathbb{Q}, +, \cdot),\quad (\mathbb{R}, +, \cdot),\quad (\mathbb{C}, +, \cdot),\quad (M_n(R), +, \cdot)
    \]
\end{example}


\begin{definition}[محک فشرده]
    چون $(S, +) \leqslant (R, +)$ زیرمجموعهٔ ناتهی $S$ از حلقهٔ $R$ یک زیرحلقه است اگر و تنها اگر:
    \[
        \forall a, b \in S \quad a - b \in S
    \]
    \[
        \forall a, b \in S \quad ab \in S
    \]

\end{definition}



\begin{remark}
    مثال: $(2\mathbb{Z}, +, \cdot)$ یک زیرحلقه از $(\mathbb{Z}, +, \cdot)$ است.
    زیرا:
    \[
        a - b \in 2\mathbb{Z}, \quad \forall a,b \in 2\mathbb{Z} \quad ab \in 2\mathbb{Z}
    \]
    به طور مشابه
    $(n\mathbb{Z}, +, \cdot)$
    یک زیر حلقه از
    $(\mathbb{Z}, +, \cdot)$
    است.
    اما این زیرحلقه‌ها هیچ‌گاه یکدار نیستند.
\end{remark}


\begin{definition}[ایده‌آل]
    زیرمجموعهٔ ناتهی $I$ از حلقهٔ $(R,+,\cdot)$ را یک \textbf{ایده‌آل} راست حلقهٔ $R$ نامند هرگاه
    \((I,+) \leqslant (R, +)\)
    و بعلاوه:
    \[
        \forall a \in I \quad \forall r \in R \quad ar \in I
    \]

    به طور مشابه می‌توان ایده‌آل چپ را نیز تعریف کرد.
    ایده آلی که هم ایده‌آل چپ و هم ایده‌آل راست باشد را
    \textbf{ایده‌آل دوطرفه} یا به‌اختصار \textbf{ایده‌آل} می‌نامند.
\end{definition}

\begin{example}
    هر ایده‌آل یک زیرحلقه است اما هر زیرحلقه یک ایده‌آل نیست. به عنوان مثال
    \begin{itemize}
        \item   $(2\mathbb{Z}, +, \cdot)$ زیرحلقه‌ای از $(\mathbb{Z}, +, \cdot)$ است ولی ایده‌آل آن نیست.
        \item  $(\mathbb{Z}, +, \cdot)$ زیرحلقه‌ای از $(\mathbb{R}, +, \cdot)$ است ولی ایده‌آل آن نیست. چرا که \( \sqrt{2} \cdot  n \not \in \mathbb{Z}\).
    \end{itemize}
\end{example}

\begin{remark}
    \begin{itemize}
        \item در حلقه یکدار
              $R$
              اگر ایده‌آل
              \(I\)
              عنصر
              \(1_R\)
              را دربر داشته باشد، در این‌صورت
              \(I = R\).
        \item در هر میدان هر ایده‌آل ناصفر دارای عنصر همانی است. لذا یک میدان دارای ایده‌آل نابدیهی نیست.
    \end{itemize}

\end{remark}


\begin{definition}[  حلقه خارج‌قسمتی]
    اگر $I$ یک ایده‌آل از حلقهٔ $R$ باشد، به ازای هر \(r \in R\) تعریف می‌کنیم:
    \[
        r + I := \{ r + a \mid a \in I \}
    \]
    مجموعهٔ تمام همدسته ها را با \(R/I\) نمایش می‌دهیم. حال عملیات عملیات جمع و ضرب  را به صورت تعریف
    \[
        (r + I) + (r' + I) := (r + r') + I
    \]
    \[
        (r + I)(r' + I) := rr' + I
    \]
    ثابت می‌شود که $(R/I, +, \cdot)$ یک حلقه است که به آن \textbf{حلقهٔ خارج‌قسمتی $R$}   گفته می‌شود.

\end{definition}

\begin{proof}[خوش تعریفی]

\end{proof}

\begin{remark}
    آیا برای زیرحلقه‌ها می‌توان حلقهٔ خارج‌قسمتی تعریف کرد؟
\end{remark}

\begin{definition}[حلقهٔ خارج‌قسمتی]
    اگر $(R, +, \cdot)$ یک حلقه باشد و $I$ یک ایده‌آل از $R$، آنگاه حلقهٔ خارج‌قسمتی $R$ نسبت به $I$ را به‌صورت $(R/I, +, \cdot)$ تعریف می‌کنیم.

\end{definition}



\begin{example}
    اگر $n \in \mathbb{N}$، آنگاه $( \frac{\mathbb{Z}}{n\mathbb{Z}}, +, \cdot)$ یک حلقهٔ خارج‌قسمتی $( \mathbb{Z}, +, \cdot)$
    است.
\end{example}


\begin{definition}[همریختی حلقه]
    فرض کنید $(R,+,\cdot)$ و $(S,+',\cdot')$ دو حلقه دلخواه باشند و $f \colon R \to S$ یک تابع باشد. $f$ را \textbf{همریختی حلقه‌ای} می‌نامند اگر:
    \[
        \forall r_1, r_2 \in R \quad f(r_1 + r_2) = f(r_1) +' f(r_2)
    \]
    \[
        \forall r_1, r_2 \in R \quad f(r_1 r_2) = f(r_1) \cdot' f(r_2)
    \]
    مشابه بحث‌های گروه، \textbf{تکریخی} ، \textbf{بروریختی} و \textbf{یکریختی} تعریف می‌شود.
\end{definition}

\begin{definition}[هسته و برد همریختی حلقه]
    اگر $f \colon R \to S$ همریختی حلقه باشد، آنگاه:

    \[
        \ker f = \left\{ r \in R \mid f(r) = 0_S \right\} \quad \text{(ایده‌آل)}
    \]
    \[
        \operatorname{Im} f = \left\{ f(r) \mid r \in R \right\}
    \]
\end{definition}

\begin{theorem}[قضایای یکریختی برای حلقه‌ها]
    \leavevmode
    \begin{itemize}
        \item \textbf{قضیهٔ اول یکریختی:} اگر $f \colon R \to S$ همریختی حلقه باشد، آنگاه:
              \[
                  \frac{R}{\ker f} \ringcong \operatorname{Im} f
              \]

        \item \textbf{قضیهٔ دوم یکریختی:}
              اگر $I $ و $J$ دو ایده‌آل از $R$ باشند، آنگاه:
              \[
                  \frac{I + J}{I} \cong \frac{J}{I \cap J}
              \]
        \item \textbf{قضیهٔ سوم یکریختی:}اگر $I \subseteq J$ دو ایده‌آل از حلقهٔ $R$ باشند، آنگاه:
              \[
                  \frac{R/J}{I/J} \ringcong \frac{R}{I}
              \]
    \end{itemize}


\end{theorem}



\begin{definition}[ایده‌آل اصلی]
    ایده‌آل $I$  از حلقهٔ جابجایی و یکدار $R$ را \textbf{اصلی} نامند هرگاه عنصری چون $a \in R$ موجود باشد به‌طوری‌که:
    \[
        I = Ra = \left\{ ra \mid r \in R \right\}
    \]
\end{definition}

\begin{definition}[حلقه با ایده‌آل‌های اصلی]
    حلقه‌ای که تمام ایده‌آل‌های آن اصلی باشند را \textbf{حلقهٔ ایده‌آل اصلی} یا \lr{PIR}  \LTRfootnote{Principal Ideal Ring}می‌نامند.
\end{definition}

\begin{definition}[عضو مقسوم‌علیه صفر ]
    عضو $a \in R$ را \textbf{مقسوم‌علیه صفر} نامند هرگاه عنصری چون $b \in R$ موجود باشد به‌طوری‌که \(ab = 0 \)

\end{definition}

\begin{definition}[دامنه صحیح]
    حلقهٔ جابجایی و یکدار که مقسوم‌علیه صفر ناصفری نداشته باشد را \textbf{دامنه صحیح} یا  \textbf{حوزهٔ صحیح}\LTRfootnote{Domain} می‌نامند.
\end{definition}

\begin{definition}[حوزه ایده‌آل اصلی]
    دامنهٔ صحیحی که تمام ایده‌آل‌های آن اصلی باشند را \textbf{حوزهٔ ایده‌آل اصلی} یا \lr{PID} \LTRfootnote{Principal Ideal Domain} می‌نامند.
\end{definition}

\begin{definition}[ایده‌آل ماکسیمال]

    در حلقهٔ
    \(R\)
    ایده‌آل سرهً
    \(I\)
    را ماکسیمال گوییم، هرگاه چنانچه
    \(J\)
    ایده‌آلی از
    \(R\)
    باشد که
    \(I \subseteq J \subseteq R\)
    آنگاه
    \(J = I\)
    یا
    \(J = R\).
    مجموعهٔ تمام ایده‌آل های ماکسیمال حلقهٔ \(R\) را با
    \(\max (R)\)
    نمایش می دهیم.


\end{definition}

\begin{example}
    $(2\mathbb{Z}, +, \cdot) \leqslant (\mathbb{Z}, +, \cdot)$
    ایده‌آلی ماکسیمال است.
\end{example}



\begin{definition}[ایده‌آل اول]
    در حلقهٔ جابجایی و یکدار $R$، ایده‌آل $P$ را \textbf{ایده‌آل اول} نامند هرگاه برای هر $a,b \in R$ اگر $ab \in P$ آنگاه $a \in P$ یا $b \in P$.
    \[\forall a, b \ in R \quad ab \in P \implies a \in P \lor b \in P\]
    مجموعهٔ تمام ایده‌آل‌های اول حلقه $R$ را با $\mathrm{Spec}(R)$ نمایش می‌دهیم.

\end{definition}


\begin{remark}
    در صورتی که حلقه جابجایی نباشد از شرط زیر استفاده می‌کنیم:
    \[
        IJ \subseteq R \implies I \subseteq R \lor J \subseteq R
    \]
\end{remark}


\begin{example}
    اگر \(p\) عددی اول باشد. آنگاه ایده‌آل \((p) = p\mathbb{Z} = \{pk : k \in \mathbb{Z}\}\) یک ایده‌آل اول در حلقهٔ \(\mathbb{Z}\) است. بعلاوه \((p)\) ایده‌آلی ماکسیمال است.
\end{example}

\begin{proof}[اول بودن]
    فرض کنید \(ab \in p\mathbb{Z}\). پس \(p \mid ab\). چون \(p\) عدد اول است،  دراین صورت
    \(p \mid a\) یا \(p \mid b\). معادل با این است که \(a \in p\mathbb{Z}\) یا \(b \in p\mathbb{Z}\).
    پس \(p\mathbb{Z}\) یک ایده‌آل اول است.
\end{proof}

\begin{proof}[ماکسیمال بودن]
    هر ایده‌آل در \(\mathbb{Z}\) اصلی است؛ یعنی به صورت \(n\mathbb{Z}\) برای عددی \(0 \leq n \in \mathbb{Z}\) نوشته می‌شود.
    اگر \(p\mathbb{Z} \subseteq I \subsetneq \mathbb{Z}\)، آنگاه \(I = n\mathbb{Z}\) و \(n \mid p\).
    چون \(p\) عدد اول است، مقسوم‌علیه‌هایش فقط \(1\) و \(p\) هستند، و چون \(I \ne \mathbb{Z}\)، باید \(n \ne 1\) باشد؛
    در نتیجه \(n = p\) و \(I = p\mathbb{Z}\).
\end{proof}


\begin{example}
    در هر \textbf{دامنه صحیح}، ایده‌آل صفر یعنی $(0)$ یک \textbf{ایده‌آل اول} است؛ زیرا اگر $ab = 0$ آنگاه $a = 0$ یا $b = 0$.
\end{example}

\begin{example}
    در حلقهٔ $\mathbb{Z}_6$، ایده‌آل $(0)$ اول هست اما ماکسیمال نیست.
    \[
        ab \in P = (0) \implies ab = 0 \xRightarrow[\text{\footnotesize{\(\mathbb{Z} \) ایده‌آل اصلی}}]{ایده‌آل اصلی} a = 0 \lor b = 0 \implies a \in (0) \lor b \in (0)
    \]
    و بعلاوه
    \[
        (0) \subsetneq (2) \subsetneq \mathbb{Z}_6
    \]
\end{example}



\begin{theorem}
    در هر حلقهٔ جابجایی و یکدار $R$ داریم:
    \[
        \mathrm{max}(R) \subseteq \mathrm{Spec}(R)
    \]
    یعنی هر ایده‌آل ماکسیمال، ایده‌آل اول نیز هست.
\end{theorem}

\begin{proof}
    فرض کنید $M$ یک ایده‌آل ماکسیمال از $R$ باشد و فرض کنید $a, b \notin M$ اما $ab \in M$.
    در این صورت، چون $a \notin M$، داریم $M + (a) = R$ و نیز $M + (b) = R$.
    پس موجودند $m_1, m_2 \in M$ و $r_1, r_2 \in R$ چنان‌که:
    \[
        m_1 + r_1 a = 1_R \quad \text{و} \quad m_2 + r_2 b = 1_R
    \]
    حالا داریم:
    \[
        1_R = (m_1 + r_1 a)(m_2 + r_2 b) = m_1 m_2 + m_1 r_2 b + r_1 a m_2 + r_1 r_2 ab
    \]
    تمام جملات بالا در $M$ قرار دارند زیرا $M$ ایده‌آل است و $ab \in M$.
    پس $1_R \in M$ که تناقض است، زیرا ایده‌آل شامل $1$ کل حلقه می‌شود.
    در نتیجه، باید داشته باشیم $a \in M$ یا $b \in M$، پس $M$ ایده‌آل اول است.
\end{proof}

\begin{remark}
    در حالت کلی، $\mathrm{max}(R) \subsetneq \mathrm{Spec}(R)$.

    \begin{example}
        شاید افزودن یک مثال خوب باشه.
    \end{example}
\end{remark}


\begin{definition}[ پوچ‌ساز  ]
    اگر $I$ ایده‌آلی از حلقهٔ $R$ باشد، \textbf{پوچ‌ساز} $I$ را به‌صورت زیر تعریف می‌کنیم:
    \[
        \operatorname{Ann}_R(I) = \left\{ r \in R \mid \forall a \in I \quad ra = 0 \right\}
    \]
\end{definition}

\begin{definition}[رابطهٔ ترتیبی]
    فرض کنید $X$ یک مجموعهٔ ناتهی و $\leq$ یک رابطه بر $X$ باشد. در این صورت، $\leq$ یک رابطهٔ ترتیبی است اگر خواص زیر برقرار باشند:

    % needs amsmath
    \[
        \begin{aligned}
            \text{\bfseries بازتابی}\colon   & \quad \forall x \in X,\; x \le x                                                                \\
            \text{\bfseries پادتقارنی}\colon & \quad \forall x_1,x_2 \in X,\; (x_1 \le x_2 \text{ و } x_2 \le x_1) \Rightarrow x_1 = x_2       \\
            \text{\bfseries تعدی}\colon      & \quad \forall x_1,x_2,x_3 \in X,\; (x_1 \le x_2 \text{ و } x_2 \le x_3) \Rightarrow x_1 \le x_3
        \end{aligned}
    \]

\end{definition}

\begin{definition}[ترتیب کلی]
    رابطهٔ $\leq$ روی $X$ را \textbf{ترتیب کلی}  یا  \textbf{تام} گویند اگر:
    \[
        \forall x_1, x_2 \in X \quad x_1 \leq x_2 \ \text{یا} \ x_2 \leq x_1
    \]
\end{definition}

\begin{definition}[زنجیر]
    مجموعهٔ ناتهی  $X$  همراه با رابطهٔ ترتیبی $\leq$ را درنظر بگیرید. زیرمجموعه‌ای از $X$ که کاملاً مرتب باشد  را \textbf{زنجیر}  \LTRfootnote{chain} می‌نامند.
\end{definition}

\begin{definition}[کران بالا]
    فرض کنید $Y \subseteq X$ یک زیرمجموعهٔ ناتهی و $X$ مجهز به رابطهٔ ترتیبی $\leq$ باشد. در این صورت، عنصر $\alpha \in X$ را \textbf{کران بالای} $Y$ می‌نامیم اگر:
    \[
        \forall a \in Y, \quad a \leq \alpha
    \]

    و عنصر $\beta$  را یک عنصر ماکسیمال \(X\) نامند هرگاه
    \[
        \forall a \in X ;\quad \beta \leq a \implies \beta = a
    \]
\end{definition}

\begin{theorem}[لِم زورن]
    اگر $X$ یک مجموعهٔ ناتهی و $\leq$ یک رابطهٔ ‌ترتیب جزئی روی $X$ باشد به‌طوری‌که
    هر زنجیر در $X$ دارای کران بالایی باشد، آنگاه $X$ حداقل دارای یک عنصر ماکسیمال است.
\end{theorem}


\begin{definition}[مجموعه بستهُ ضربی]
    زیرمجموعهٔ ناتهی $S$ از حلقهٔ جابجایی و یکدار $R$ را یک \textbf{مجموعهٔ بستهُ ضربی} نامند هرگاه:
    \[
        1_R \in S \quad \text{و} \quad \forall a, b \in S; \quad ab \in S
    \]
\end{definition}

\begin{theorem}
    فرض کنید $R$ حلقه‌ای جابجایی و یکدار است، $S$ یک زیرمجموعهٔ بستهُ ضربی و $I$ یک ایده‌آل از $R$ باشد، طوری که \(I \cap S = \varnothing\)
    در این صورت، یک \textbf{ایده‌آل اول} $P$ از $R$ وجود دارد به‌طوری‌که:
    \[
        I \subseteq P \quad \text{و} \quad P \cap S = \varnothing
    \]
\end{theorem}

\begin{proof}
    مجموعهٔ زیر را در نظر بگیرید:
    \[
        \mathcal{A} = \left\{ J \subseteq R \mid I \leqslant J \leqslant R, \ J \cap S = \varnothing \right\}
    \]
    این مجموعه ناتهی است، زیرا $I \in \mathcal{A}$.
    $\mathcal{A}$همراه با رابطهٔ شمول یک زیرمجموعهُ جزئاً مرتب است.
    حال، هر زنجیرهُ $T$ در $\mathcal{A}$ دارای کران بالایی است:
    \[
        L = \bigcup_{J \in T} J
    \]
    $L$ یک ایده‌آل است (چرا؟)، و داریم:
    \[
        I \subseteq L \quad \text{و} \quad L \cap S = \varnothing
    \]
    پس $L \in \mathcal{A}$ و $L$ کران بالای $T$ است. به کمک لم زورن، $\mathcal{A}$ دارای عضو ماکسیمال \( P\) است.
    نشان می‌دهیم که این عضو ماکسیمال، یک ایده‌آل اول است.
    فرض کنید $a,b \in R$ و $ab \in P$ ولی $a \notin P$ و $b \notin P$.
    آنگاه
    \[
        \left.
        \begin{array}{l}
            P \subsetneq P + Ra \\
            P \subsetneq P + Rb
        \end{array}
        \right\}
        \quad \xRightarrow[\text{\footnotesize{ \(P\)ایده‌آلی ماکسیمال }}]{}
        P + Ra \not\subseteq \mathcal{A} \text{ و }
        P + Rb \not\subseteq \mathcal{A}
    \]


    $P + (a)$ و $P + (b)$ ایده‌آل‌هایی بزرگ‌تر
    از $P$ و شامل \(I\) هستند که حتماً با $S$ اشتراک دارند (به‌خاطر ماکسیمال بودن $P$).
    پس موجودند $s_1 \in (P + (a)) \cap S$ و $s_2 \in (P + (b)) \cap S$.
    بنویسید:
    \[
        s_1 = m_1 + r_1 a, \quad s_2 = m_2 + r_2 b \quad \text{ به ازای }  m_1, m_2 \in P,\ r_1, r_2 \in R
    \]
    آنگاه:
    \[
        s_1 s_2 = (m_1 + r_1 a)(m_2 + r_2 b) = m_1m_2 + m_1r_2b + r_1am_2 + r_1r_2ab \in P
    \]
    و چون $S$ بستهُ ضربی است، $s_1 s_2 \in S$. پس $P \cap S \ne \varnothing$ که تناقض است.
    بنابراین، $P$ ایده‌آل اول است.
\end{proof}



\begin{corollary}
    فرض کنید $R$ حلقه‌ای جابجایی و یکدار باشد و $I$ ایده‌آلی سره از $R$ باشد.
    در این صورت، عضوی چون $M \in \mathrm{Max}(R)$ موجود است به‌طوری‌که:
    \[
        I \subseteq M
    \]
\end{corollary}

\begin{proof}
    مجموعهٔ $S := \{1_R\}$ را در نظر بگیرید. $S$ بستهُ ضربی است و $I \cap S = \varnothing$.
    پس یک ایده‌آل اول چون $P$ وجود دارد که $I \subseteq P$ و $P \cap S = \varnothing$.

    اما چون $P \in \mathcal{A}$، پس در آن عضو ماکسیمال هم هست. در نتیجه اگر $P \subseteq J \subsetneq R$ و $J$ ایده‌آل باشد،
    آنگاه
    $J \in \mathcal{A}$
    لذا
    $J = P$
    که یعنی
    $P$ ماکسیمال است.
\end{proof}


\begin{corollary}
    اگر $R$ یک حلقهٔ جابجایی و یک‌دار باشد، آن‌گاه:
    \[
        \mathrm{Spec}(R) \ne \varnothing.
    \]
\end{corollary}

\begin{corollary}
    اگر $R$ یک حلقهٔ جابجایی و یک‌دار باشد، آن‌گاه:
    \[
        \mathrm{Max}(R) \ne \varnothing.
    \]
\end{corollary}

