\section{\lr{07 Hom}}




\textbf{قضیه.} فرض کنید \( R \) حلقه‌ای یک‌دار و نیمه‌ساده است. در این‌صورت \( R \) با جمع مستقیم تعداد متناهی ایده‌آل چپ مینیمال برابر است.

\textit{اثبات.} دقت کنید که \( R \) بعنوان \( R \)-مدول نیمه‌ساده است و ضمناً هر زیرمدولی از \( R \) ایده‌آل چپ
\( R \)
درواقع یک
است.
بعلاوه، زیرمدول ساده است، هرگاه ایده‌آل چپ مینیمال باشد.
اما
\( R \) نیمه‌ساده‌ است، پس
\[
    R = \bigoplus_{j \in J} I_j
\]
که \( I_j \)ها ایده‌آل‌های چپ مینیمال \( R \) هستند. اما چون \( 1_R \in R \) پس
\[
    \exists n : 1_R = r_1 + r_2 + \dots + r_n \quad (r_j \in I_{i_j})
\]
در نتیجه هر \( r \in R \) داریم
\[
    r = r 1_R = r r_1 + r r_2 + \dots + r r_n \in \sum_{j=1}^n I_{i_j}
\]
و چون \( \oplus \) جمع مستقیم است(چرا؟)، پس
\[
    R = \bigoplus_{j=1}^n I_{i_j}.
\]

\hrulefill

\textbf{تعریف.}
فرض کنید \( M \) و \( N \) دو \( R \)-مدول باشند. در این‌صورت مجموعه‌ی تمام \( R \)-همریختی‌ها از \( M \) به \( N \) را با نماد
\[
    \mathrm{Hom}_R(M,N)
\]
نشان می‌دهیم.

تذکر: \( \mathrm{Hom}_R(M,N) \) با جمع معمولی \( R \)-همریختی‌ها یک گروه آبلی است، زیرا
\[
    f,g \in \mathrm{Hom}_R(M,N) \implies f+g \in \mathrm{Hom}_R(M,N).
\]

\hrulefill

اما اگر \( R \) حلقه‌ای جابجایی باشد، آنگاه \( \mathrm{Hom}_R(M,N) \) یک \( R \)-مدول چپ نیز خواهد بود که ضرب اسکالر را به شکل زیر تعریف می‌کنیم:
\[
    \forall r \in R, \quad f \in \mathrm{Hom}_R(M,N) \implies rf:M\to N, \quad(rf)(m) := r f(m)
\]
و چون \( R \) جابجایی است، به ازای هر
\(rf \in \mathrm{Hom}_R(M,N)\)
و
\(r' \in R\)
داریم
\[
    (rf)(r'm) = r f(r'm) = r r' f(m) = r' r f(m) = r'(rf)(m).
\]

\hrulefill

\textbf{نماد.} اگر \( M = N \) باشد، آنگاه
\[
    \mathrm{End}_R(M) := \mathrm{Hom}_R(M,M)
\]
حلقه‌ی \( R \)-خودریختی‌های \( M \) نامیده می‌شود.

تذکر: \( \mathrm{End}_R(M) \) با عمل جمع و ترکیب توابع تشکیل یک حلقه‌  می‌دهد.

\hrulefill

\textbf{لم شُر:}
\begin{enumerate}
    \item اگر \( M \) یک \( R \)-مدول ساده باشد، در این‌صورت \( \mathrm{End}_R(M) \) یک حلقه‌ی تقسیم است.
    \item فرض کنید \( M \) و \( N \) دو \( R \)-مدول ساده باشند، در این‌صورت:
          \[
              M \cong N \iff \mathrm{Hom}_R(M,N) \neq \{0\}.
          \]
\end{enumerate}




\textit{اثبات:}
\begin{enumerate}
    \item فرض کنید \( 0 \neq f \in \mathrm{End}_R(M) \)، دراین‌صورت \(\ker f \leqslant M\) و چون \( M \) ساده است، \(\ker f = \{0\}\) یا \(\ker f = M\) که دومی ممکن نیست. پس \(\ker f = \{0\}\) و \( f \) یک‌به‌یک است. از طرفی \(\mathrm{Im} f \leqslant M\) و چون \( M \) ساده است، \(\mathrm{Im} f = M\)، پس
          \( f \)
          وارون‌پذیر
          است.
          بعلاوه
          \( f^{-1} \)
          یک
          \(R\)-همریختی
          است. پس
          \( \mathrm{End}_R(M) \)
          یک حلقهٔ تقسیم است.
    \item با استدلال مشابه بالا.(چرا؟)
\end{enumerate}

\hrulefill

\textbf{قضیه:}
فرض کنید
\(M_1\)
و
\(M_2\)
و
\(N\)
سه
\(R\)-مدول
باشند، در این صورت
\begin{enumerate}
    \item \[
              \mathrm{Hom}_R(N, M_1 \oplus M_2) \cong \mathrm{Hom}_R(N,M_1) \oplus \mathrm{Hom}_R(N,M_2)
          \]
          که این همریختی،‌ یک همریختی
          \(\mathbb{Z}\)-مدولی
          است، و درصورتی که
          \(R\)
          حلقه‌ای جابجایی باشد، یک همریختی
          \(R\)-مدولی
          است.
    \item \[
              \mathrm{Hom}_R(M_1 \oplus M_2, N) \cong \mathrm{Hom}_R(M_1,N) \oplus \mathrm{Hom}_R(M_2,N)
          \]
          که این همریختی،‌ یک همریختی
          \(\mathbb{Z}\)-مدولی
          است، و درصورتی که
          \(R\)
          حلقه‌ای جابجایی باشد، یک همریختی
          \(R\)-مدولی
          است.
\end{enumerate}

\textit{اثبات \((i)\):}   تعریف می‌کنیم:
\[
    \psi : \mathrm{Hom}_R(N, M_1 \oplus M_2) \to \mathrm{Hom}_R(N, M_1) \oplus \mathrm{Hom}_R(N, M_2)
\]
\[
    f \mapsto (\pi_1 \circ f, \pi_2 \circ f)
\]
که در آن \( \pi_1, \pi_2 \) توابع تصویر روی مؤلفه‌ی اول و دوم هستند، یعنی:
\[
    \pi_1 : \mathrm{Hom}_R(N,M_1) \oplus \mathrm{Hom}_R(N,M_2) \to \mathrm{Hom}_R(N,M_1), \quad (a,b) \mapsto a
\]
\[
    \pi_2 : \mathrm{Hom}_R(N,M_1) \oplus \mathrm{Hom}_R(N,M_2) \to \mathrm{Hom}_R(N,M_2), \quad (a,b) \mapsto b
\]
واضح است که
\(\pi_1\)
و
\(\pi_۲\)
\(R\)-همریختی
هتسند. چرا خوش‌تعریف است؟
و می‌توان دید که \( \pi_1, \pi_2 \) \( R \)-همریختی هستند. (چرا؟)

\textbf{\(R\)-همریختی:}
\[
    \psi(f+g) = (\pi_1(f+g), \pi_2(f+g)) = (\pi_1(f)+\pi_1(g), \pi_2(f)+\pi_2(g)) = \psi(f) + \psi(g)
\]
\[
    \left.
    \begin{aligned}
        \pi_1(rf) = r\pi_1(f), \\
        \pi_2(rf) = r\pi_2(f)
    \end{aligned}
    \right\}
    \implies \psi(rf) = r \psi(f)
\]


\textbf{یک‌به‌یک بودن:}
اگر \( f \in \ker \psi \) آنگاه
\[
    \psi(f) = 0 \implies
    \begin{cases}
        \pi_1 f = 0, \\
        \pi_2 f = 0.
    \end{cases}
\]
اگر \( f \neq 0 \) یعنی
\[
    \exists n \in N : f(n) = (m_1,m_2) \neq (0,0)
\] پس
\[
    \pi_1 f \neq 0\text{ یا }\pi_2 f \neq 0
\]
پس \( f=0 \)،
تناقض.

\textbf{پوشا بودن:}
فرض کنید
\[
    (g_1, g_2) \in \mathrm{Hom}_R(N,M_1) \oplus \mathrm{Hom}_R(N,M_2).
\]
تعریف می‌کنیم:
\[
    \forall n; \;f(n) := (g_1(n), g_2(n)),
\]
ثابت می‌شود که
\( f \in \mathrm{Hom}_R(N, M_1 \oplus M_2)\)
و
\(\psi(f) = (g_1,g_2)\).(چرا؟)

\hrulefill

\textit{اثبات \((ii)\):}
توابع زیر را تعریف می‌کنیم:
\[
    \iota_1 : M_1 \to M_1 \oplus M_2, \quad m_1 \mapsto (m_1, 0)
\]
\[
    \iota_2 : M_2 \to M_1 \oplus M_2, \quad m_2 \mapsto (0, m_2)
\]
\[
    \psi : \mathrm{Hom}_R(M_1 \oplus M_2, N) \to \mathrm{Hom}_R(M_1,N) \oplus \mathrm{Hom}_R(M_2,N)
\]
\[
    f \mapsto (f \circ \iota_1, f \circ \iota_2)
\]
\textbf{تمرین:}
ادامه حل را کامل کنید.


\hrulefill

\textbf{تمرین.} اگر \( \{M_i\}_{i\in I} \)ها \( R \)-مدول باشند و \( N \) هم \( R \)-مدول باشد:
\begin{enumerate}
    \item
          \[
              \mathrm{Hom}_R\left( \bigoplus_{i \in I} M_i, N \right) \cong \prod_{i \in I} \mathrm{Hom}_R(M_i, N)
          \]
    \item
          \[
              \mathrm{Hom}_R\left( N, \prod_{i \in I} M_i \right) \cong \prod_{i \in I} \mathrm{Hom}_R(N, M_i)
          \]
          (همریختی‌ها \(\mathbb{Z}\)-مدولی و اگر \( R \) جابجایی باشد \( R \)-مدولی‌اند)
\end{enumerate}

\hrulefill

\textbf{تذکر:} اگر \( M \) یک \( R \)-مدول باشد، \(\mathrm{Hom}_R(R,M)\) یک \( R \)-مدول چپ است (بدون نیاز به جابجایی \( R \)).
با تعریف ضرب زیر:
\[
    \forall r \in R, \; f \in \mathrm{Hom}_R(R,M), \quad (r \cdot f)(a) := f(ar) \quad (\forall a \in R)
\]
به ازای هر
\(a_1, a_2, r, s, a \in R\)
داریم:
\[
    (r \cdot f)(a_1+a_2) = f((a_1+a_2)r) = f(a_1r) + f(a_2r) = (r \cdot f)(a_1) + (r \cdot f)(a_2)
\]
\[
    (r \cdot f)(sa) = f((sa)r) = f(s(ar)) = s f(ar) = s (r \cdot f)(a)
\]
بررسی خواص
\(R\)-مدولی
بودن
\(\mathrm{Hom}_R(R,M) \) \(\cdots\)


\hrulefill

\textbf{قضیه.} اگر \( R \) حلقه‌ی یک‌دار و \( M \) یک \( R \)-مدول یکانی باشد آنگاه
\[
    M \modulecong \mathrm{Hom}_R(R, M)
\]
\textit{اثبات:} تعریف می‌کنیم:
\[
    \psi : \mathrm{Hom}_R(R, M) \to M, \quad f \mapsto f(1)
\]
\(\psi\) یک \( R \)-همریختی  است:
\[
    \psi(f+g) = (f+g)(1) = f(1) + g(1) = \psi(f) + \psi(g)
\]
\[
    \psi(rf) = (rf)(1) = f(1r) = f(r) \stackrel{?}{=} r f(1) = r \psi(f)
\]
\(\psi\) یک‌به‌یک است: اگر
\[
    f \in \ker(\psi) \implies\psi(f) = 0
\]
آنگاه
\[
    f(1) = 0 \implies \forall r \in R, f(r) = r f(1) = 0 \implies f=0.
\]


\(\psi\) پوشا است: برای هر \( m \in M \)، تعریف می‌کنیم:
\[
    \left.
    \begin{aligned}
        f_m(1) := m, \\
        f_m(r) := r m
    \end{aligned}
    \right\}
    \implies
    \begin{cases}
        f_m(r+r') = f_m(r) + f_m(r') \\
        f_m(sr) = sf(r)
    \end{cases}
\]
بنابراین\( f_m \in \mathrm{Hom}_R(R,M) \) و \(\psi(f_m) = m\).
