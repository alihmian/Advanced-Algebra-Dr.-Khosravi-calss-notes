\section{\lr{09 Free modules}}
\section*{مدول‌های آزاد}


\textbf{تعریف.} اگر $M$ یک $R$-مدول و $\mathcal{X} \subseteq M$ باشد،
$\mathcal{X}$ را \textit{مستقل خطی} گوییم هرگاه
\[
    \forall x_1, x_2, \dots, x_n \in \mathcal{X}, \quad r_1 x_1 + \dots + r_n x_n = 0 \ \implies\ r_1 = r_2 = \dots = r_n = 0.
\]
در این حالت گوییم $\mathcal{X}$ نسبت به حلقه $R$ مستقل خطی است.
$\mathcal{X}$ را وابستهٔ خطی گوییم، هرگاه مستقل خطی نباشد.

\textbf{تعریف.} زیرمجموعه $\mathcal{X}$ از $M$ را یک \textit{پایه} برای $M$ گوییم هرگاه $M = \langle \mathcal{X} \rangle$ و $\mathcal{X}$ مستقل خطی باشد.

\textbf{تعریف.} $R$-مدول $M$ را \textit{آزاد} (\lr{Free}) نامیم هرگاه دارای یک پایه باشد.

\textbf{مثال.} $M = \{0\}$ آزاد است زیرا $M = \langle \varnothing \rangle$ و $\varnothing$ مستقل خطی است.

\textbf{مثال.} حلقه‌ی یک‌دار $R$ به عنوان $R$-مدول $R = \langle 1 \rangle$ آزاد است.

\textbf{مثال.} به عنوان $\mathbb{Z}$-مدول، $\mathbb{Z}_n$ آزاد نیست
چرا که
\[
    \begin{cases}
        \mathbb{Z}_n = \langle \overline{1} \rangle              \\
        \mathbb{Z}_n = \langle \overline{a} \rangle & (a, n) = 1
    \end{cases}
\]
پس آزاد نیست (چرا؟).

\textbf{مثال.} $\mathbb{Q}$ به عنوان $\mathbb{Z}$-مدول آزاد نیست.

\textbf{فرض.} $Q = \langle X \rangle$


اگر $|X| \ge 2$ آنگاه $\frac{a}{b}, \frac{c}{d} \in X$، اما در این صورت:
\[
    bc \left( \frac{a}{b} \right) + (-da) \left( \frac{c}{d} \right) = 0
\]
که تناقض است.
حال اگر
$|\mathcal{X}| = 1$
آنگاه
\(\mathcal{X} = \{\frac{M}{n}\}\).
ولی در اینصورت
\(\frac{m}{2n}\notin \langle X \rangle\).

\hrulefill

\textbf{تذکر.} اگر $R$-مدول $M$ با تولید متناهی باشد، آنگاه الزاما هر زیرمدول آن با تولید متناهی نیست.

\textbf{مثال.} حلقهٔ $R = \mathbb{Z}[x_1, x_2, x_3, \dots]$ را در نظر بگیرید. $R$ به عنوان $R$-مدول با تولید متناهی است ($R = \langle 1 \rangle$) ولی ایده‌آل
\[
    I = \langle x_1, x_2, x_3, \dots \rangle \lneq R
\]
که یک $R$-زیرمدول است با تولید متناهی نیست (چرا؟) (راهنمایی: چندجمله‌ای ثابت صفر و هر چندجمله‌ای متناهی متغیر.).
    [اگر حلقه نوتری باشد هر زیرمدول یک $R$-مدول با تولید متناهی است. پس باید حلقه را غیر نوتری بگیریم.]
\[
    \text{}
\]

\textbf{تذکر.} واضح است اگر $M$ با تولید متناهی یاشد، آنگاه $\frac{M}{N}$ نیز با تولید متناهی است.

\textbf{قضیه.} اگر $R$-مدول $M$ نوتری باشد، آنگاه $N$ و $\frac{M}{N}$ با تولید متناهی باشند، آنگاه $M$ نیز با تولید متناهی است.

\textbf{اثبات.} فرض کنید
\[
    N = \langle n_1, n_2, \dots, n_k \rangle, \quad
    \frac{M}{N} = \langle m_1 + N, \dots, m_t + N \rangle.
\]
اگر $m \in M$ دلخواه باشد، می‌توان نوشت:
\[
    \begin{aligned}
        m + N & = r_1 (m_1 + N) + \dots + r_t (m_t + N) = r_1 m_1 + \dots + r_t m_t + N \\
              & \implies m - (r_1 m_1 + \dots + r_t m_t) \in N                          \\
              & \implies m - (r_1 m_1 + \dots + r_t m_t) = r_1' n_1 + \dots + r_k' n_k  \\
              & \implies m \in \langle m_1, \dots, m_t, n_1, \dots, n_k \rangle
    \end{aligned}
\]



\hrulefill

\textbf{تذکر.} $R$-مدول $M$ می‌تواند با تولید متناهی باشد ولی   پایه‌ای متناهی نداشته باشد. (چرا؟)

\textbf{مثال.} $\mathbb{Z}_n$ به عنوان $\mathbb{Z}$-مدول (مستقل خطی؟).

\hrulefill

\textbf{تذکر.} اگر $M$ آزاد باشد و $N \leqslant M$ آنگاه لزوماً $\frac{M}{N}$ آزاد نیست.

\textbf{مثال.}
\begin{itemize}
    \item $\mathbb{Z}$ به عنوان $\mathbb{Z}$-مدول ،
    \item $n\mathbb{Z}$ به عنوان $\mathbb{Z}$-مدول ، و
    \item
          \[
              \frac{\mathbb{Z}}{n\mathbb{Z}} \cong \mathbb{Z}_n
          \]
\end{itemize}


\hrulefill

\textbf{تذکر.} اگر $M$ آزاد باشد و $N \leqslant M$ آنگاه لزوماً $N$ آزاد نیست.

\textbf{مثال.} $\mathbb{Z}_6$ به عنوان $\mathbb{Z}$-مدول آزاد است، اما
\[
    \langle \overline{2} \rangle = \{ \overline{0}, \overline{2} , \overline{4} \}
\]
به عنوان \(\mathcal{Z}_6\)مدول آزاد نیست زیرا $\overline{3} \cdot \overline{2} =\overline{3} \cdot \overline{4} = \overline{0}$.

بعداً در بحث دنباله‌های دقیق
خواهیم دید
که اگر
$N$ و $\frac{M}{N}$ آزاد باشند، آنگاه $M$ هم آزاد است.

\textbf{مثال.} $R = \mathbb{R}[x,y]$، در این صورت $R$ به عنوان $R$-مدول آزاد است. نشان دهید $N = \langle x, y \rangle$ آزاد نیست.

\textbf{حل:} $R = \langle 1 \rangle$

اگر $N$ آزاد باشد و فرض کنیم $A$ پایه آن باشد،
در این صورت
بایستی
$|A| > 1$
چرا که
$x, y$
نمی‌توانند
با تنها یک چندجمله‌ای تولید شوند.
حال اگر
\[
    f(x), g(x) \in A \ \implies \ g(x) f(x) + (-f(x)) g(x) = 0
\]
پس مستقل خطی نیستند. پس $N$ به عنوان $R$-مدول آزاد نیست.
