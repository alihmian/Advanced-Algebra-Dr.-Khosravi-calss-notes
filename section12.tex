\section{\lr{12 Noetherian and Artinian rings}}

\subsection*{حلقه‌های نوتری و آرتینی}

\textbf{تعریف.} فرض کنید $R$ یک حلقه باشد. در این صورت $R$ را \textbf{نوتری چپ (آرتینی چپ)} گوییم هرگاه هر زنجیر صعودی (نزولی) از ایده‌آل‌های چپ متوقف شود.

\textbf{حلقه $R$ را نوتری (آرتینی) گوییم هرگاه هم نوتری (آرتینی) چپ و هم نوتری (آرتینی) راست باشد.}

\textbf{قضیه.} فرض کنید $R$ حلقه‌ای یک‌دار باشد. در این صورت $R$ نوتری چپ (آرتینی چپ) است اگر و تنها اگر برای هر $n \geqslant 1$، $M_n(R)$ نوتری چپ (آرتینی چپ) باشد.

\textbf{اثبات.}
$M_n(R)$ با تولید متناهی به عنوان $R$-مدول چپ (چرا؟) پس نوتری (آرتینی) است.

اما هر ایده‌آل چپ $M_n(R)$ خود یک $R$-مدول چپ هم هست (چرا؟) پس هر زنجیر از ایده‌آل‌های چپ $M_n(R)$ زنجیری از $R$-زیرمدول‌های $M_n(R)$ است و لذا متوقف می‌شود.

\textbf{برعکس.}
فرض کنید
\(n = 1\).

\hrulefill

\textbf{تذکر.} اگر حلقه $M_n(R)$ حتی برای یک $n$ هم نوتری (آرتینی) چپ باشد،  نتیجه می‌شود که $R$ نوتری (آرتینی) چپ است. چرا که
\[
    I_1 \subsetneq I_2 \subsetneq I_3 \subsetneq \cdots \Rightarrow M_n(I_1) \subsetneq M_n(I_2) \subsetneq M_n(I_3) \subsetneq \cdots
\]


\hrulefill

\textbf{قضیه اساسی هیلبرت}

\begin{center}
    حلقهٔ $R$ نوتری و یک‌دار $\iff$ حلقهٔ $R[x]$ نوتری چپ است.
\end{center}

\textbf{اثبات.}  فرض کنید که $R$ نوتری چپ باشد و زنجیر
\[
    A_1 \subsetneq A_2 \subsetneq A_3 \subsetneq \cdots
\]
از ایده‌آل‌های $R[x]$ را در نظر بگیرید.

ابتدا برای هر ایده‌آل $I$ از $R[x]$ و هر $n \in \mathbb{N}$ تعریف می‌کنیم:
\[
    \varphi_n(I) = \{ a \in R \mid a_n x^n + a_{n-1} x^{n-1} + \dots + a_0 \in I \ \text{و} \ a = a_n \neq 0 \} \cup \{0\}.
\]

واضح است که $\varphi_n(I)$ یک ایده‌آلِ $R$ است.
بعلاوه چون
$I$
ایده‌آل
$R[x]$
است، پس
\[
    \varphi_n(I) \subseteq \varphi_{n+1}(I).
\]
بعلاوه اگر
\(I\)
و
\(J\)
دو ایده‌آل
$R[x]$
باشند که
\(I \subseteq J\)
و بعلاوه به‌ازای هر
\(n \in N\)
\[
    \varphi_n(I) = \varphi_n(J)
\]
آنگاه
\(I = J\).
برای اثبات
این مطلب فرض کنید
\(f(x) \in J \setminus I\).
در این صورت اگر درجهٔ
\(f(x)\)
برابر با
\(n\)
باشد و چون
\(\varphi_n(I) = \varphi_n(J)\)
پس
\(g_n(x) \in I\)
هست که ضریب
\(x^n\)
آن با
\(f(x)\)
یکسان است. (چرا؟) و لذا
\(f(x) - g_n(x)\)
یا صفر است یا درجهٔ آن کمتر یا مساوی
\(x^{n-1}\)
می‌باشد. نتیجه ایکنه چون
\(\varphi_{n-1}(I) = \varphi_{n-1}(J)\)
با استدلال مشابه
\(g_{n-1}(x)\)
در
\(I\)
هست که
\(f(x) - g_n(x) - g_{n-1}(x)\)
یا صفر است یا درجهٔ حداکثر
\(n-2\)
و با ادامه دادن این فرآیند نهایتاً به صفر می‌رسیم(چرا؟) پس
\(f(x)\)
مجموعی از عناصر
\(I\)
است و لذا
\(I = J\). که تناقض است.



حال به اثبات برمی‌گردیم:
\[
    A_1 \subsetneq A_2 \subsetneq A_3 \subsetneq \cdots
\]
ابتدا تمام ایده‌آل‌ها به شکل $\varphi_t(A_s)$ را که $t,s \in \mathbb{N}$ در یک مجموعه قرار می‌دهیم. بنابر فرض چون $R$ نوتری است، این مجموعه عناصر ماکسیمالی مانند $\varphi_k(A_l)$ دارد که $k,q \in \mathbb{N}$. پس برای هر $t,s \in \mathbb{N}$،
\[
    \varphi_t(A_l) \subseteq \varphi_k(A_l)
\]
و اما چون $\varphi_k(A_q) \subseteq \varphi_k(A_{q+i})$ برای هر $i \in \mathbb{N}$، پس
\[
    \forall k' \geq k \quad \varphi_{k'}(A_l) = \varphi_k(A_l).
\]
یعنی
\[
    \forall i \in \mathbb{N} \quad \varphi_k(A_l) = \varphi_{k+i}(A_l).
\]

بعلاوه $\forall j \in \mathbb{N}$ داریم $A_l \subseteq A_{l+j}$ و در نتیجه
\[
    \varphi_{k}(A_l) \subseteq \varphi_{k+i}(A_{l+j}) \quad \forall i,j \in \mathbb{N}.
\]


بنابراین فقط
$\varphi_{k-1},  \dots, \varphi_2, \varphi_{1}$
بررسی نشده‌است.
اما
\[
    \varphi_1(A_l) \subseteq \varphi_1(A_{l+1}) \subseteq \varphi_1(A_{l+2}) \subseteq \dots
\]
و از نوتری بودن $R$ مرتبه‌ای مانند $l+t_1$ هست که
\[
    \varphi_1(A_{l+t_1}) = \varphi_1(A_{l+t_1+i}) \quad \forall i \in \mathbb{N}.
\]

به همین صورت برای $\varphi_2$ مقداری مانند $l+t_2$ هست که
\[
    \varphi_2(A_{l+t_2}) = \varphi_2(A_{l+t_2+i}) \quad \forall i \in \mathbb{N},
\]
و همین‌طور برای $\varphi_{k-1}$ یک مرتبهٔ $q+t_{k-1}$ وجود دارد.

حال با در نظر گرفتن
\[
    m \geq \max \{  l+t_1, l+t_2, \dots, l+t_{k-1} \}
\]
نتیجه می‌شود که
\[
    \forall n \in \mathbb{N} \quad \varphi_n(A_m) = \varphi_n(A_{m+i}) \quad \forall i.
\]
و بنابراین ادعا ثابت شده:
\[
    A_m = A_{m+i} \quad \forall i \in \mathbb{N}.
\]
یعنی $R[x]$ نوتری چپ است.

\textbf{تذکر.} حکم برای آرتینی صحیح نیست.

\textbf{قضیه:} اگر $M = \bigoplus_{i=1}^n M_i$ که $M_i$ها $R$-مدول می‌باشند، در این‌صورت
\[
    \operatorname{Hom}_R(M,M) \ringcong
    \begin{bmatrix}
        \operatorname{Hom}_R(M_1,M_1) & \operatorname{Hom}_R(M_2,M_1) & \cdots & \operatorname{Hom}_R(M_k,M_1) \\
        \operatorname{Hom}_R(M_1,M_2) & \operatorname{Hom}_R(M_2,M_2) & \cdots & \operatorname{Hom}_R(M_k,M_2) \\
        \vdots                        & \vdots                        & \ddots & \vdots                        \\
        \operatorname{Hom}_R(M_1,M_k) & \operatorname{Hom}_R(M_2,M_k) & \cdots & \operatorname{Hom}_R(M_k,M_k)
    \end{bmatrix}
\]

\textit{اثبات:}
\[
    \sigma : \operatorname{Hom}_R(M,M) \longrightarrow \left[ \dots \right]
\]
به‌طوری که
\[
    \sigma(\varphi) = \left[ \pi_i \varphi \iota_j \right]_{(i,j)}
\]
\[
    \sigma(\varphi_1 + \varphi_2) = \left[ \pi_i (\varphi_1 + \varphi_2) \iota_j \right]
    = \sigma(\varphi_1) + \sigma(\varphi_2)
\]
\[
    \sigma(\varphi_1) \, \sigma(\varphi_2) =
    \left[ \pi_i \varphi_1 \iota_j \right] \left[ \pi_i \varphi_2 \iota_j \right]
    = \left[ \sum_{k=1}^n \pi_i \varphi_1 \iota_k \, \pi_k \varphi_2 \iota_j \right]
\]
\[
    = \left[ \pi_i \varphi_1 \left( \sum_{k=1}^n \iota_k \pi_k \right) \varphi_2 \iota_j \right]
    = \left[ \pi_i \varphi_1 \varphi_2 \iota_j \right]
    = \sigma(\varphi_1 \varphi_2)
\]
پس $\sigma$ همریختی حلقه‌ای است.

\textbf{یک به یک بودن:}
\[
    \sigma(\varphi) = 0 \quad \Longleftrightarrow \quad \varphi \in \ker\sigma
\]
اگر $\varphi \neq 0$، آنگاه
\[
    \exists (m_1,\dots,m_n) \in M \quad s.t. \quad \varphi(m_1,\dots,m_n) \neq 0
\]
پس
\[
    \exists j; \quad \varphi(0,\dots,0,m_j,0,\dots,0) \neq 0
\]
یعنی این عبارت که یک بردار است مولفهٔ $j$ آن ناصفر است. بنابراین
\[
    \exists i \quad \pi_i \varphi \iota_j(m_j) \neq 0
\]
و لذا $\pi_i \varphi \iota_j \neq 0$ که با $\sigma(\varphi) = [0]$ تناقض دارد. پس $\sigma$ یک‌به‌یک است.

\textbf{پوشا بودن}: فرض کنید $\left[ f_{ij} \right]$ عنصری در طرف راست عبارت بالا باشد، تعریف می‌کنیم:
\[
    \varphi = \sum_{r} \sum_{s} \iota_r f_{rs} \pi_s
\]
که عنصری از $\operatorname{Hom}_R(M,M)$ است. آنگاه:
\[
    \sigma(\varphi) =
    \left[ \pi_i \left( \sum_{r,s} \iota_r f_{rs} \pi_s \right) \iota_j \right]
\]
حال می‌دانیم که اگر $i \neq j$ آنگاه $\pi_i \iota_j = 0$ و $\pi_i \iota_i = \mathrm{id}$.
پس
\[
    \sigma(\varphi) = \left[ f_{ij} \right].
\]

\textbf{حلقهٔ متضاد:}

اگر $R$ یک حلقه باشد، در این صورت
\[
    R^{\mathrm{op}}
\]
همان مجموعهٔ اعضای $R$ است، جمع همان جمع حلقهٔ $R$ است،
اما ضرب به صورت
\[
    a \ast b := ba \quad (\text{ضرب در $R$})
\]
تعریف می‌شود که مجدداً یک حلقه می‌سازد.

\textbf{لم.} اگر $R$ یک حلقه باشد
\[
    M_n(R)^{\mathrm{op}} \ringcong M_n(R^{\mathrm{op}})
\]
به‌صورت حلقه‌ای.

\textbf{اثبات:} تعریف می‌کنیم
\[
    \varphi : M_n(R)^{\mathrm{op}} \longrightarrow M_n(R^{\mathrm{op}}),
    \quad
    A \longmapsto A^t.
\]


از نظر مجموعه
\(R^{\mathrm{op}} = R\)
پس
$M_n(R^{\mathrm{op}}) = M_n(R)^{\mathrm{op}}$
به عنوان مجموعه یا گروه جمعی. داریم:
\[
    \varphi(A+B) = (A+B)^t = A^t + B^t = \varphi(A) + \varphi(B).
\]

\textit{  ضرب:}
در $M_n(R)^{\mathrm{op}}$ داریم
\[
    \varphi(A \ast B) = \varphi(BA) = (BA)^t = A^t B^t = \varphi(A) \varphi(B).
\]
و به وضوح $\varphi$ یک به یک و پوشا است.


\textbf{قضیه.} اگر $R$ یک‌دار باشد
\[
    R^{\mathrm{op}} \ringcong \operatorname{Hom}_R(R,R)
\]
به‌صورت حلقه‌ای.

\textbf{اثبات:}
تعریف می‌کنیم
\[
    \varphi : R^{\mathrm{op}} \longrightarrow \operatorname{Hom}_R(R,R),
    \quad
    a \longmapsto f_a : R \to R,\quad f_a(r) := ra.
\]

واضح است که
\[
    f_a(r_1 + r_2) = f_a(r_1) + f_a(r_2),
\]
\[
    f_a(rr') = r r' a = r (r'a) = r f_a(r'),
\]
پس $f_a \in \operatorname{Hom}_R(R,R)$.

بعلاوه $\varphi$ همریختی حلقه‌ای است:
\[
    \varphi(a_1 + a_2) = f_{a_1 + a_2} = f_{a_1} + f_{a_2} = \varphi(a_1) + \varphi(a_2),
\]
\[
    \varphi(a_1 \ast a_2) = \varphi(a_2 a_1) = f_{a_2 a_1} = f_{a_2} f_{a_1} = \varphi(a_1) \varphi(a_2),
\]
زیرا
\[
    f_{a_2} f_{a_1}(r) = f_{a_2}(r a_1) = (r a_1) a_2 = r a_1 a_2.
\]

$\varphi$ یک‌به‌یک و پوشا است:\\
یک‌به‌یک بودن: $\ker\varphi = \{0\}$ واضح است.\\
پوشا بودن: اگر $g \in \operatorname{Hom}_R(R,R)$، آنگاه
\[
    g(r) = r g(1) \quad \implies \quad g = f_{g(1)}.
\]
پس یک‌ریخت است.

\textbf{قضیه آرتین–ودربرن.}
فرض کنید $R$ حلقه‌ای نیم‌ساده باشد. در این صورت اعداد طبیعی
\(n_1, \dots, n_k\)
و حلقه‌های تقسیم $D_k, \dots, D_2, D_1$ موجودند به‌طوری‌که
\[
    R \ringcong M_{n_1}(D_1) \times \cdots \times M_{n_k}(D_k)
\]

\textbf{اثبات.}
$R$ حلقه‌ای نیم‌ساده است، پس جمع مستقیم تعدادی متناهی زیرمدول ساده (ایده‌آل مینیمال) خودش است:
\[
    R = \bigoplus_{i=1}^n M_i.
\]
می‌نویسیم
\[
    R = \bigoplus_{i=1}^k n_i M_i
\]
که $M_i \not\cong M_j$ هرگاه $i \neq j$. و
\[
    n_i M_i = \bigoplus_{t=1}^{n_i} M_i.
\]
حال می‌توان نوشت
\[
    R^{\mathrm{op}} \ringcong \operatorname{Hom}_R(R,R)
    \ringcong \operatorname{Hom}_R\left( \bigoplus_{i=1}^k n_i M_i , \bigoplus_{i=1}^k n_i M_i \right)
\]
\[
    \cong
    \begin{bmatrix}
        \operatorname{Hom}_R(n_1 M_1 , n_1 M_1) & \cdots & \operatorname{Hom}_R(n_k M_k , n_1 M_1) \\
        \vdots                                  & \ddots & \vdots                                  \\
        \operatorname{Hom}_R(n_1 M_1 , n_k M_k) & \cdots & \operatorname{Hom}_R(n_k M_k , n_k M_k)
    \end{bmatrix}
\]

اگر $i \neq j$ آنگاه
\[
    \operatorname{Hom}_R(n_i M_i , n_j M_j) \cong \bigoplus_{t=1}^{n_i n_j} \operatorname{Hom}_R(M_i , M_j) = 0
\]
که در آن همریختی ها
\(\mathbb{Z}\)-مدولی
هستند. از آنجا که به‌عنوان
\(\mathbb{Z}\)-مدول
برابر با صفر شد، پس به‌عنوان
\(R\)-مدولی
هم صفر ‌می‌شود.

و دقت کنید که
\[
    \operatorname{Hom}_R(n_i M_i , n_i M_i)
    \cong
    \begin{bmatrix}
        \operatorname{Hom}_R( M_i ,  M_i) & \cdots & \\
        \vdots                            & \ddots &
    \end{bmatrix}
\]
در تمام درایه‌ها $\operatorname{Hom}_R(M_i, M_i)$ ظاهر می‌شود که چون $M_i$ ساده است، بنابر لم شُور $\operatorname{Hom}_R(M_i, M_i) = D_i$ یک حلقهٔ تقسیم می‌شود. پس
\[
    \operatorname{Hom}_R(n_i M_i , n_i M_i) \cong M_{n_i}(D_i).
\]

با جمع‌بندی داریم:
\[
    R^{\mathrm{op}}
    \cong
    \begin{bmatrix}
        M_{n_1}(D_1) & 0            & \cdots & 0            \\
        0            & M_{n_2}(D_2) & \cdots & 0            \\
        \vdots       & \vdots       & \ddots & \vdots       \\
        0            & 0            & \cdots & M_{n_k}(D_k)
    \end{bmatrix}
    \cong M_{n_1}(D_1) \times \cdots \times M_{n_k}(D_k).
\]
آخرین گام: اگر $\mathrm{op}$ بگیریم و دقت کنیم که $D_i^{\mathrm{op}}$ خودش هم یک حلقهٔ تقسیم است، حکم ثابت می‌شود.
